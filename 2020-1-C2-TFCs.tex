% (find-LATEX "2020-1-C2-TFCs.tex")
% (defun c () (interactive) (find-LATEXsh "lualatex -record 2020-1-C2-TFCs.tex" :end))
% (defun D () (interactive) (find-pdf-page      "~/LATEX/2020-1-C2-TFCs.pdf"))
% (defun d () (interactive) (find-pdftools-page "~/LATEX/2020-1-C2-TFCs.pdf"))
% (defun e () (interactive) (find-LATEX "2020-1-C2-TFCs.tex"))
% (defun u () (interactive) (find-latex-upload-links "2020-1-C2-TFCs"))
% (defun v () (interactive) (find-2a '(e) '(d)) (g))
% (find-pdf-page   "~/LATEX/2020-1-C2-TFCs.pdf")
% (find-sh0 "cp -v  ~/LATEX/2020-1-C2-TFCs.pdf /tmp/")
% (find-sh0 "cp -v  ~/LATEX/2020-1-C2-TFCs.pdf /tmp/pen/")
%   file:///home/edrx/LATEX/2020-1-C2-TFCs.pdf
%               file:///tmp/2020-1-C2-TFCs.pdf
%           file:///tmp/pen/2020-1-C2-TFCs.pdf
% http://angg.twu.net/LATEX/2020-1-C2-TFCs.pdf
% (find-LATEX "2019.mk")

% «.title»			(to "title")
% «.formadas-por-segmentos»	(to "formadas-por-segmentos")
% «.exercicio-1»		(to "exercicio-1")
% «.exercicio-2»		(to "exercicio-2")
% «.na-direcao-errada»		(to "na-direcao-errada")
% «.exercicio-3»		(to "exercicio-3")
% «.exercicio-4»		(to "exercicio-4")
% «.exercicio-5»		(to "exercicio-5")
% «.exercicio-6»		(to "exercicio-6")

\documentclass[oneside,12pt]{article}
\usepackage[colorlinks,citecolor=DarkRed,urlcolor=DarkRed]{hyperref} % (find-es "tex" "hyperref")
\usepackage{amsmath}
\usepackage{amsfonts}
\usepackage{amssymb}
\usepackage{pict2e}
\usepackage[x11names,svgnames]{xcolor} % (find-es "tex" "xcolor")
\usepackage{colorweb}                  % (find-es "tex" "colorweb")
%\usepackage{tikz}
%
% (find-dn6 "preamble6.lua" "preamble0")
%\usepackage{proof}   % For derivation trees ("%:" lines)
%\input diagxy        % For 2D diagrams ("%D" lines)
%\xyoption{curve}     % For the ".curve=" feature in 2D diagrams
%
\usepackage{edrx15}               % (find-LATEX "edrx15.sty")
\input edrxaccents.tex            % (find-LATEX "edrxaccents.tex")
\input edrxchars.tex              % (find-LATEX "edrxchars.tex")
\input edrxheadfoot.tex           % (find-LATEX "edrxheadfoot.tex")
\input edrxgac2.tex               % (find-LATEX "edrxgac2.tex")
%
%\usepackage[backend=biber,
%   style=alphabetic]{biblatex}            % (find-es "tex" "biber")
%\addbibresource{catsem-slides.bib}        % (find-LATEX "catsem-slides.bib")
%
% (find-es "tex" "geometry")
\usepackage[a6paper, landscape,
            top=1.5cm, bottom=.25cm, left=1cm, right=1cm, includefoot
           ]{geometry}
%
\begin{document}

\catcode`\^^J=10
\directlua{dofile "dednat6load.lua"}  % (find-LATEX "dednat6load.lua")

%L dofile "edrxtikz.lua"  -- (find-LATEX "edrxtikz.lua")
%L dofile "edrxpict.lua"  -- (find-LATEX "edrxpict.lua")
\pu

% «defs»  (to ".defs")
% (find-LATEX "edrx15.sty" "colors-2019")
\long\def\ColorRed   #1{{\color{Red1}#1}}
\long\def\ColorViolet#1{{\color{MagentaVioletLight}#1}}
\long\def\ColorViolet#1{{\color{Violet!50!black}#1}}
\long\def\ColorGreen #1{{\color{SpringDarkHard}#1}}
\long\def\ColorGreen #1{{\color{SpringGreenDark}#1}}
\long\def\ColorGreen #1{{\color{SpringGreen4}#1}}
\long\def\ColorGray  #1{{\color{GrayLight}#1}}
\long\def\ColorGray  #1{{\color{black!30!white}#1}}
\long\def\ColorBrown #1{{\color{Brown}#1}}
\long\def\ColorBrown #1{{\color{brown}#1}}

\long\def\ColorShort #1{{\color{SpringGreen4}#1}}
\long\def\ColorLong  #1{{\color{Red1}#1}}

\def\frown{\ensuremath{{=}{(}}}
\def\True {\mathbf{V}}
\def\False{\mathbf{F}}

\def\drafturl{http://angg.twu.net/LATEX/2020-1-C2.pdf}
\def\drafturl{http://angg.twu.net/2020.1-C2.html}
\def\draftfooter{\tiny \href{\drafturl}{\jobname{}} \ColorBrown{\shorttoday{} \hours}}

\unitlength=10pt

% (find-angg ".emacs" "c2q192")


%  _____ _ _   _                               
% |_   _(_) |_| | ___   _ __   __ _  __ _  ___ 
%   | | | | __| |/ _ \ | '_ \ / _` |/ _` |/ _ \
%   | | | | |_| |  __/ | |_) | (_| | (_| |  __/
%   |_| |_|\__|_|\___| | .__/ \__,_|\__, |\___|
%                      |_|          |___/      
%
% «title»  (to ".title")
% (c2m201tfcsp 1 "title")
% (c2m201tfcs    "title")

\thispagestyle{empty}

\begin{center}

\vspace*{1.2cm}

{\bf \Large Cálculo 2 - 2020.1}

\bsk

Aula 9: TFC1

(o primeiro Teorema Fundamental do Cálculo)

\bsk

Eduardo Ochs - RCN/PURO/UFF

\url{http://angg.twu.net/2020.1-C2.html}

\end{center}

\newpage

% «formadas-por-segmentos»  (to ".formadas-por-segmentos")
% (c2m201tfcsp 2 "formadas-por-segmentos")
% (c2m201tfcs    "formadas-por-segmentos")

{\bf Funções formadas por segmentos}

\ssk

Uma \ColorRed{função escada} é uma função cujo gráfico é formado por
um número finito de segmentos horizontais e de pontos isolados. Essa
terminologia é padrão.

Nesta aula vamos usar algumas funções como a abaixo, cujo gráfico é
formado por um número finito de segmentos {\sl não necessariamente
  horizontais} e de pontos isolados. Acho que não há um termo padrão
para funções deste tipo, então vou chamá-las de \ColorRed{funções
  formadas por segmentos}.
%
$$
 \unitlength=8pt
 %
 f(x) \;\; = \;\;
 \vcenter{\hbox{%
 \beginpicture(0,-5)(5,5)
   \pictgrid%
   \pictpiecewise{(0,0)c--(1,1)c
                  (1,-1)o--(2,-2)o (2,1)c
                  (2,2)o--(3,3)c
                  (3,-3)c--(4,-4)c
                  (4,4)o--(5,5)c}%
   \pictaxes%
 \end{picture}%
 }}
$$

\newpage



% (find-books "__analysis/__analysis.el" "apex-calculus")
% (find-apexcalculuspage (+ 10 236) "5.4 The Fundamental Theorem of Calculus")
% (c2m201escadasp 9 "spoiler")
% (c2m201escadas    "spoiler")

Na aula passada nós conseguimos calcular a função
%
$$G(b) = \Intx{1}{b}{g(x)}$$

para a função $g(x)$ abaixo. Obtivemos:
%
$$
 g(x) \;\; = \;\;
 \vcenter{\hbox{%
 \beginpicture(0,-2)(10,6)
   \pictgrid%
   \pictpiecewise{(0,1)--(2,1)o (2,0)c
                  (2,2)o--(4,2)o
                  (4,0)c--(6,0)o
                  (6,-1)c--(8,-1)c
                  (8,1)o--(10,1)}%
   \pictaxes%
 \end{picture}%
 }}
 %
 \qquad
 %
 G(b) \;\; = \;\;
 \vcenter{\hbox{%
 \beginpicture(0,-2)(10,6)
   \pictgrid%
   \pictpiecewise{(1,0)c--(2,1)c--(4,5)c--(6,5)c--(8,3)c--(10,5)}%
   \pictaxes%
 \end{picture}%
 }}
$$

\newpage

...e vimos que:

\begin{enumerate}

\item $G(1) = \Intx{1}{1}{g(x)} = 0$.

\item $G'(x) = g(x)$ ``sempre que isto faz sentido'', isto é, para
  todos os valores de $x$ nos quais $G(x)$ é derivável.

\item $G'(x) = g(x)$ em todo $x$ no qual $g(x)$ é contínua. Isto é uma
  condição mais forte que a 2.

\item $G(x)$ é contínua --- inclusive nos valores de $x$ nos quais as
  condições 2 e 3 não se aplicam.

\end{enumerate}

\newpage

% «exercicio-1»  (to ".exercicio-1")
% (c2m201tfcsp 5 "exercicio-1")
% (c2m201tfcs    "exercicio-1")

{\bf Exercício 1.}

\ssk

a) Faça uma cópia no papel do gráfico da $g(x)$ do slide 3 e
represente nele as áreas $\Intx{1}{3}{g(x)}$ e $\Intx{1}{4}{g(x)}$.

b) Agora visualize a diferença $\Intx{1}{4}{g(x)} -
\Intx{1}{3}{g(x)}$, cancelando na região $\Intx{1}{4}{g(x)}$ a região
$\Intx{1}{3}{g(x)}$. Represente graficamente esta diferença
$\Intx{1}{4}{g(x)} - \Intx{1}{3}{g(x)}$ e verifique que ela é
exatamente a área $\Intx{3}{4}{g(x)}$.

c) Calcule no olhômetro \ColorRed{sem fazer nenhuma conta no papel} o
valor de cada subexpressão das igualdades abaixo e verifique que as
duas igualdades são verdade. Lembre que você pode obter $G(3)$ e
$G(4)$ pelo gráfico da $G$ do slide 3.
%
$$\begin{array}{rcl}
  \Intx{3}{4}{g(x)} &=& \Intx{1}{4}{g(x)} - \Intx{1}{3}{g(x)} \\[5pt]
                    &=& G(4) - G(3)
  \end{array}
$$

\newpage

% «exercicio-2»  (to ".exercicio-2")
% (c2m201tfcsp 6 "exercicio-2")
% (c2m201tfcs    "exercicio-2")

{\bf Exercício 2.}

\ssk

Dá pra generalizar o ``$\Intx{3}{4}{g(x)} = G(4) - G(3)$'' do final do
exercício 1. O caso geral vai ser:

\msk

Se $F(b) = \Intx{a}{b}{f(x)}$ então $\Intx{c}{d}{f(x)} = F(d) - F(c)$,

\msk

e isto vai valer para qualquer função integrável $f$ e quaisquer
valores de $a$, $c$ e $d$. Isto é um pouco mais difícil de interpretar
quando a função $f$ assume valores negativos em alguns trechos. Neste
exercício você vai tentar interpretar isto nas nossas funções $g$ e
$G$ em alguns casos complicados.

a) Visualize cada subexpressão de 
%
$$\begin{array}{rcl}
  \Intx{7}{8}{g(x)} &=& \Intx{1}{8}{g(x)} - \Intx{1}{7}{g(x)} \\[5pt]
                    &=& G(8) - G(7)
  \end{array}
$$

e descubra como lidar com a parte abaixo do eixo horizontal.

\newpage

% «na-direcao-errada»  (to ".na-direcao-errada")
% (c2m201tfcsp 7 "na-direcao-errada")
% (c2m201tfcs    "na-direcao-errada")

{\bf Integrais ``na direção errada''}

\ssk

Até agora nós só lidamos com integrais da forma $\Intx{a}{b}{f(x)}$
nas quais tínhamos $a≤b$... mas os matemáticos acham estas duas
propriedades aqui

\msk

1. Se $F(b) = \Intx{a}{b}{f(x)}$ então $\Intx{c}{d}{f(x)} = F(d) - F(c)$

2. $\Intx{a}{c}{f(x)} = \Intx{a}{b}{f(x)} + \Intx{b}{c}{f(x)}$

\msk

TÃO legais que eles decidiram que elas têm que valer sempre... e pra
elas valerem sempre a gente precisa encontrar o significado ``certo''
para integrais da forma $\Intx{a}{b}{f(x)}$ onde $a>b$, isto é,
integrais em que o intervalo de integração está ``expresso na ordem
errada''.

\newpage

% «exercicio-3»  (to ".exercicio-3")
% (c2m201tfcsp 8 "exercicio-3")
% (c2m201tfcs    "exercicio-3")

{\bf Exercício 3.}

\ssk

Descubra o \ColorRed{valor certo} e depois a \ColorRed{interpretação
  geométrica} da integral ``na direção errada'' em cada uma das
igualdades abaixo. Obs: ainda estamos usando as funções $g$ e $G$ do
slide 3.

\msk

a) $\Intx{1}{4}{g(x)} + \Intx{4}{3}{g(x)} = \Intx{1}{3}{g(x)}$

\msk

b) $\Intx{1}{8}{g(x)} + \Intx{8}{7}{g(x)} = \Intx{1}{7}{g(x)}$

\newpage



Agora leia as duas primeiras seções desta página:

\ssk

{\footnotesize
% https://pt.wikipedia.org/wiki/Teorema_fundamental_do_c%C3%A1lculo
\url{https://pt.wikipedia.org/wiki/Teorema_fundamental_do_c\%C3\%A1lculo}
}

\ssk

e veja a figura da Parte 2 da demonstração.

\msk

Como nós estamos usando principalmente funções formadas por segmentos
e que podem ser descontínuas nós vamos precisar de uma versão do TFC1
um pouco mais geral do que a dessa página da Wikipedia.

\msk

Digamos que $f:[a,b]→\R$ seja uma função integrável.

Uma \ColorRed{antiderivada} para $f$ é uma função $F:[a,b]→\R$ tal que
$F'(x)=f(x)$ em todo $x∈[a,b]$ no qual a $f(x)$ é contínua.

Uma \ColorRed{antiderivada contínua} para $f$ é uma função
$F:[a,b]→\R$ tal que $F'(x)=f(x)$ em todo $x∈[a,b]$ no qual a $f(x)$ é
contínua --- e que além disso esta $F$ é \ColorRed{contínua}. Ou seja,
nos pontos $x$ em que $f(x)$ não é contínua a $F(x)$ não precisa ser
derivável, mas precisa ser contínua.

\newpage

% «exercicio-4»  (to ".exercicio-4")
% (c2m201tfcsp 10 "exercicio-4")
% (c2m201tfcs     "exercicio-4")


{\bf Exercício 4.}

Seja $f:[0,5]→\R$ esta função:

$$
 \unitlength=10pt
 %
 f(x) \;\; = \;\;
 \vcenter{\hbox{%
 \beginpicture(0,-2)(5,2)
   \pictgrid%
   \pictpiecewise{(0,1)c--(2,1)c
                  (2,-1)o--(5,-1)c}%
   \pictaxes%
 \end{picture}%
 }}
$$

a) Encontre uma antiderivada contínua para $f$.

b) Encontre uma antiderivada para $f$ que não é contínua.

c) Encontre uma antiderivada contínua para $f$ que

obedeça $F(0)=0$.

d) Encontre uma antiderivada contínua para $f$ que

obedeça $F(1)=0$.

\newpage

A versão do TFC1 que vamos usar é esta aqui:

\bsk

Digamos que $f:[a,c]→\R$ seja integrável.

Seja $F:[a,c]→\R$ \ColorRed{uma} antiderivada contínua de $f$

que obedeça $F(a)=0$.

Seja $G: [a,c] → \R$ a função $G(b) := \Intx{a}{b}{f(x)}$.

\ColorRed{Então as funções $F$ e $G$ são iguais.}

\bsk

Ou seja, dá pra encontrar a função $G(b) := \Intx{a}{b}{f(x)}$

só encontrando uma antiderivada contínua!

Na aula passada nós levamos horas pra encontra a $G$,

mas agora quando a $f$ é uma função escada simples

você deve ser capaz de encontrar uma antiderivada

contínua dela no olhômetro BEM rápido.



\newpage

% «exercicio-5»  (to ".exercicio-5")
% (c2m201tfcsp 12 "exercicio-5")
% (c2m201tfcs     "exercicio-5")

{\bf Exercício 5.}

\ssk

Seja $f:\R→\R$ esta função:

$$
 \unitlength=10pt
 %
 f(x) \;\; = \;\;
 \vcenter{\hbox{%
 \beginpicture(0,-2)(6,3)
   \pictgrid%
   \pictpiecewise{(0,2)--(2,2)c
                  (2,1)o--(3,1)c
                  (3,0)o--(4,0)c
                  (4,-1)o--(6,-1)}%
   \pictaxes%
 \end{picture}%
 }}
$$

\msk

Seja $F:\R→\R$ uma antiderivada contínua de $f$

que obedeça $F(1)=0$.

\msk


a) Desenhe o gráfico da $F$ \ColorRed{no olhômetro, sem fazer contas}.

\ssk

Agora você tem dois jeitos de calcular $\Intx{1}{6}{f(x)}$,

que tem que dar o mesmo resultado...

\ssk

b) Calcule $\Intx{1}{6}{f(x)}$ pelo gráfico da $f$.

c) Calcule $\Intx{1}{6}{f(x)}$ pelo gráfico da $F$.


\newpage

% «exercicio-6»  (to ".exercicio-6")
% (c2m201tfcsp 13 "exercicio-6")
% (c2m201tfcs     "exercicio-6")

{\bf Exercício 6.}

\ssk

Seja $f:\R→\R$ a nossa função preferida das primeiras aulas:

$$\begin{array}{rcl}
  f(x) &=& 4 - (x-2)^2 \\
       &=& 4 - (x^2 -4x + 4) \\
       &=& -x^2 +4x \\
  \end{array}
$$

a) Encontre uma antiderivada para $f$.

b) Encontre uma antiderivada para $f$

\phantom{!!!!} \ColorRed{que obedeça $F(0)=0$.}

c) Use a sua resposta do item anterior para calcular
%
$$\Intx{0}{4}{f(x)}.$$




%\printbibliography

\end{document}

%  __  __       _        
% |  \/  | __ _| | _____ 
% | |\/| |/ _` | |/ / _ \
% | |  | | (_| |   <  __/
% |_|  |_|\__,_|_|\_\___|
%                        
% <make>

 (eepitch-shell)
 (eepitch-kill)
 (eepitch-shell)
# (find-LATEXfile "2019planar-has-1.mk")
make -f 2019.mk STEM=2020-1-C2-TFCs veryclean
make -f 2019.mk STEM=2020-1-C2-TFCs pdf

% Local Variables:
% coding: utf-8-unix
% ee-tla: "c2m201tfcs"
% End:
