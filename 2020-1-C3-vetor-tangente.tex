% (find-LATEX "2020-1-C3-vetor-tangente.tex")
% (defun c () (interactive) (find-LATEXsh "lualatex -record 2020-1-C3-vetor-tangente.tex" :end))
% (defun D () (interactive) (find-pdf-page      "~/LATEX/2020-1-C3-vetor-tangente.pdf"))
% (defun d () (interactive) (find-pdftools-page "~/LATEX/2020-1-C3-vetor-tangente.pdf"))
% (defun e () (interactive) (find-LATEX "2020-1-C3-vetor-tangente.tex"))
% (defun u () (interactive) (find-latex-upload-links "2020-1-C3-vetor-tangente"))
% (defun v () (interactive) (find-2a '(e) '(d)) (g))
% (find-pdf-page   "~/LATEX/2020-1-C3-vetor-tangente.pdf")
% (find-sh0 "cp -v  ~/LATEX/2020-1-C3-vetor-tangente.pdf /tmp/")
% (find-sh0 "cp -v  ~/LATEX/2020-1-C3-vetor-tangente.pdf /tmp/pen/")
%   file:///home/edrx/LATEX/2020-1-C3-vetor-tangente.pdf
%               file:///tmp/2020-1-C3-vetor-tangente.pdf
%           file:///tmp/pen/2020-1-C3-vetor-tangente.pdf
% http://angg.twu.net/LATEX/2020-1-C3-vetor-tangente.pdf
% (find-LATEX "2019.mk")

% «.defs»		(to "defs")
% «.title»		(to "title")
% «.exercicio-3»	(to "exercicio-3")
% «.exercicio-4»	(to "exercicio-4")

\documentclass[oneside,12pt]{article}
\usepackage[colorlinks,citecolor=DarkRed,urlcolor=DarkRed]{hyperref} % (find-es "tex" "hyperref")
\usepackage{amsmath}
\usepackage{amsfonts}
\usepackage{amssymb}
\usepackage{pict2e}
\usepackage[x11names,svgnames]{xcolor} % (find-es "tex" "xcolor")
%\usepackage{colorweb}                 % (find-es "tex" "colorweb")
%\usepackage{tikz}
%
% (find-dn6 "preamble6.lua" "preamble0")
%\usepackage{proof}   % For derivation trees ("%:" lines)
%\input diagxy        % For 2D diagrams ("%D" lines)
%\xyoption{curve}     % For the ".curve=" feature in 2D diagrams
%
\usepackage{edrx15}               % (find-LATEX "edrx15.sty")
\input edrxaccents.tex            % (find-LATEX "edrxaccents.tex")
\input edrxchars.tex              % (find-LATEX "edrxchars.tex")
\input edrxheadfoot.tex           % (find-LATEX "edrxheadfoot.tex")
\input edrxgac2.tex               % (find-LATEX "edrxgac2.tex")
%
%\usepackage[backend=biber,
%   style=alphabetic]{biblatex}            % (find-es "tex" "biber")
%\addbibresource{catsem-slides.bib}        % (find-LATEX "catsem-slides.bib")
%
% (find-es "tex" "geometry")
\usepackage[a6paper, landscape,
            top=1.5cm, bottom=.25cm, left=1cm, right=1cm, includefoot
           ]{geometry}
%
\begin{document}

\catcode`\^^J=10
\directlua{dofile "dednat6load.lua"}  % (find-LATEX "dednat6load.lua")

% %L dofile "edrxtikz.lua"  -- (find-LATEX "edrxtikz.lua")
% %L dofile "edrxpict.lua"  -- (find-LATEX "edrxpict.lua")
% \pu

%\printbibliography


% «defs»  (to ".defs")
% (find-LATEX "edrx15.sty" "colors-2019")
\long\def\ColorRed   #1{{\color{Red1}#1}}
\long\def\ColorViolet#1{{\color{MagentaVioletLight}#1}}
\long\def\ColorViolet#1{{\color{Violet!50!black}#1}}
\long\def\ColorGreen #1{{\color{SpringDarkHard}#1}}
\long\def\ColorGreen #1{{\color{SpringGreenDark}#1}}
\long\def\ColorGreen #1{{\color{SpringGreen4}#1}}
\long\def\ColorGray  #1{{\color{GrayLight}#1}}
\long\def\ColorGray  #1{{\color{black!30!white}#1}}
\long\def\ColorBrown #1{{\color{Brown}#1}}
\long\def\ColorBrown #1{{\color{brown}#1}}

\long\def\ColorShort #1{{\color{SpringGreen4}#1}}
\long\def\ColorLong  #1{{\color{Red1}#1}}

\def\frown{\ensuremath{{=}{(}}}
\def\True {\mathbf{V}}
\def\False{\mathbf{F}}

\def\drafturl{http://angg.twu.net/LATEX/2020-1-C2.pdf}
\def\drafturl{http://angg.twu.net/2020.1-C2.html}
\def\draftfooter{\tiny \href{\drafturl}{\jobname{}} \ColorBrown{\shorttoday{} \hours}}


%  _____ _ _   _                               
% |_   _(_) |_| | ___   _ __   __ _  __ _  ___ 
%   | | | | __| |/ _ \ | '_ \ / _` |/ _` |/ _ \
%   | | | | |_| |  __/ | |_) | (_| | (_| |  __/
%   |_| |_|\__|_|\___| | .__/ \__,_|\__, |\___|
%                      |_|          |___/      
%
% «title»  (to ".title")
% (c3m201vtp 1 "title")
% (c3m201vt    "title")

\thispagestyle{empty}

\begin{center}

\vspace*{1.2cm}

{\bf \Large Cálculo 3 - 2020.1}

\bsk

Aula 2: Vetores tangentes em $\R^2$

\bsk

Eduardo Ochs - RCN/PURO/UFF

\url{http://angg.twu.net/2020.1-C3.html}

\end{center}

\newpage



{\bf Introdução}

Leia as páginas 187 a 199 do capítulo 6 do Bortolossi.

Nesta aula nós vamos representar curvas parametrizadas pelo
\ColorRed{traço} delas (p.188) com algumas anotações extras -- como
`$t=0$', `$t=1$', `$f(π)$' -- sobre pontos delas... além disso também
vamos desenhar vetores (vetores tangentes!) apoiados em alguns pontos,
fazer anotações neles também, e vamos usar tudo isso pra tentar
adivinhar (ééééé!) o comportamento de uma curva esquisita.

% (find-bortolossi6page (+ -186 188)   "traço")
% (find-bortolossi6page (+ -186 199)   "limite de vetores secantes")

\newpage

{\bf Exercício 1}

Sejam $P(t) = (4,0) + t\VEC{0,1}$ e $Q(u) = (0,3) + u\VEC{2,0}$.

Represente num gráfico só o traço de $P(t)$ e o de $Q(u)$.

Marque o ponto $P(0)$ e escreva `$t=0$' do lado dele.

Faça o mesmo para os pontos $P(1)$ (`$t=1$') e $Q(0)$ e $Q(1)$

(`$u=0$' e `$u=1$'). 

\msk

Seja $r$ o traço de $P(t)$ e $s$ o traço de $Q(u)$.

Seja $X$ o ponto de interseção de $r$ e $s$.

Quais são as coordenadas de $X$?

\msk

Cada ponto de $r$ está ``associado'' a um valor de $t$ e cada ponto de
$s$ a um valor de $u$. Quais são os valores de $t$ e $u$ associados ao
ponto $X$? Chame-os de $t_0$ e $u_0$ e indique-os no seu gráfico --
por exemplo, se $t_0=99$ e $u_0=200$ você vai escrever `$t=99$' e
`$u=200$' do lado do ponto $X$.

\newpage

{\bf Exercício 1 (continuação)}

Faça o desenho sozinho -- talvez você gaste alguns minutos pra
decifrar todas as instruções -- e depois compare o seu desenho com o
dos seus colegas.


\newpage

{\bf Exercício 2}

Seja $P(t) = (\cos t, \sen t)$.

Represente num gráfico só:

1) o traço de $P(t)$,

2) $P(\frac{π}{2}) + P'(\frac{π}{2})$, escrevendo `$P(\frac{π}{2})$'
ao lado do ponto

e `$P'(\frac{π}{2})$' ao lado da seta,

3) Idem para estes outros valores de $t$: $0, \frac14π, \frac34π, π$.

4) Seja $Q(u) = P(π) + uP'(π)$. Desenhe o traço de $Q(u)$ e anote
`$Q(0)$' e `$Q(1)$' nos pontos adequados.

\msk

5) O traço de $Q(u)$ é uma reta tangente ao traço de $P(t)$ no ponto
$P(π)$? Encontre no livro ou no resto da internet uma definição formal
de reta tangente e descubra se isto é verdade ou não.


\newpage

% «exercicio-3»  (to ".exercicio-3")
% (c3m201vtp 6 "exercicio-3")
% (c3m201vt    "exercicio-3")

{\bf Exercício 3}

Seja $P(t) = (\cos \ColorRed{2}t, \sen t)$.

Represente graficamente $P(t)+P'(t)$ para os seguintes valores de $t$:

$0, \frac14π, \frac24π, \frac34π, \ldots, 2π$.

Faça as anotações adequadas nos seu pontos e vetores pra lembrar qual
é o $t$ associado a cada um.

\msk

\ColorRed{Tente} usar as informações deste gráfico pra desenhar o
traço de $P(t)$. Isto não é nada óbvio -- se inspire nas figuras das
páginas 208 e 209 do capítulo 6 do Bortolossi e tente conseguir uma
hipótese razoável.

Você pode pensar que $P(t)$ é a posição do Super Mario Kart no
instante $t$ e $P'(t)$ é o {\sl vetor velocidade} dele no instante $t$
(lembre que um vetor tem ``direção'', ``orientação'' e ``módulo''!)...
você só sabe a posição e a velocidade dele em alguns instantes, isto
é, em alguns valores de $t$, e você vai ter que encontrar uma
aproximação razoável, olhométrica, pra pista onde ele está correndo.




\newpage

% «exercicio-4»  (to ".exercicio-4")
% (c3m201vtp 8 "exercicio-4")
% (c3m201vt    "exercicio-4")

{\bf Exercício 4}

Seja $P(t) = (\cos t, \sen \ColorRed{2}t)$.

Represente graficamente $P(t)+P'(t)$ para os seguintes valores de $t$:

$0, \frac14π, \frac24π, \frac34π, \ldots, 2π$.

Faça as anotações adequadas nos seu pontos e vetores pra lembrar qual
é o $t$ associado a cada um.

\msk

\ColorRed{Tente} usar as informações deste gráfico pra desenhar o
traço de $P(t)$. Isto não é nada óbvio -- se inspire nas figuras das
páginas 208 e 209 do capítulo 6 do Bortolossi e tente conseguir uma
hipótese razoável.

\msk

\ColorRed{Obs: este exercício é bem mais fácil do que o 3! Eu deveria
  ter apresentado ele antes do outro, mas acabei trocando a ordem por
  um erro de digitação...}




\end{document}

%  __  __       _        
% |  \/  | __ _| | _____ 
% | |\/| |/ _` | |/ / _ \
% | |  | | (_| |   <  __/
% |_|  |_|\__,_|_|\_\___|
%                        
% <make>

 (eepitch-shell)
 (eepitch-kill)
 (eepitch-shell)
# (find-LATEXfile "2019planar-has-1.mk")
make -f 2019.mk STEM=2020-1-C3-vetor-tangente veryclean
make -f 2019.mk STEM=2020-1-C3-vetor-tangente pdf

% Local Variables:
% coding: utf-8-unix
% ee-tla: "c3m201vt"
% End:
