% (find-LATEX "2020-1-C2-int-subst.tex")
% (defun c () (interactive) (find-LATEXsh "lualatex -record 2020-1-C2-int-subst.tex" :end))
% (defun D () (interactive) (find-pdf-page      "~/LATEX/2020-1-C2-int-subst.pdf"))
% (defun d () (interactive) (find-pdftools-page "~/LATEX/2020-1-C2-int-subst.pdf"))
% (defun e () (interactive) (find-LATEX "2020-1-C2-int-subst.tex"))
% (defun u () (interactive) (find-latex-upload-links "2020-1-C2-int-subst"))
% (defun v () (interactive) (find-2a '(e) '(d)) (g))
% (find-pdf-page   "~/LATEX/2020-1-C2-int-subst.pdf")
% (find-sh0 "cp -v  ~/LATEX/2020-1-C2-int-subst.pdf /tmp/")
% (find-sh0 "cp -v  ~/LATEX/2020-1-C2-int-subst.pdf /tmp/pen/")
%   file:///home/edrx/LATEX/2020-1-C2-int-subst.pdf
%               file:///tmp/2020-1-C2-int-subst.pdf
%           file:///tmp/pen/2020-1-C2-int-subst.pdf
% http://angg.twu.net/LATEX/2020-1-C2-int-subst.pdf
% (find-LATEX "2019.mk")
% (find-C2-aula-links "2020-1-C2-int-subst" "12" "isub")

% «.videos»		(to "videos")
%
% «.defs»		(to "defs")
% «.title»		(to "title")
% «.exemplo-com-gambs»	(to "exemplo-com-gambs")
% «.exercicio-1»	(to "exercicio-1")
% «.int-partes»		(to "int-partes")


% «videos»  (to ".videos")
% (find-ssr-links     "c2m201isub1" "2020_int_subst_1")
% (code-eevvideo      "c2m201isub1" "2020_int_subst_1")
% (code-eevlinksvideo "c2m201isub1" "2020_int_subst_1")
% (find-c2m201isub1video "0:00")
% (find-c2m201isub1video "8:00" "Exercicio")

% (find-ssr-links     "c2m201ipartes1" "2020_int_partes_1")
% (code-eevvideo      "c2m201ipartes1" "2020_int_partes_1")
% (code-eevlinksvideo "c2m201ipartes1" "2020_int_partes_1")
% (find-c2m201ipartes1video "0:00")

% (find-ssr-links     "c2m201derivln" "2020_deriv_ln")
% (code-eevvideo      "c2m201derivln" "2020_deriv_ln")
% (code-eevlinksvideo "c2m201derivln" "2020_deriv_ln")
% (find-c2m201derivlnvideo "0:00")



\documentclass[oneside,12pt]{article}
\usepackage[colorlinks,citecolor=DarkRed,urlcolor=DarkRed]{hyperref} % (find-es "tex" "hyperref")
\usepackage{amsmath}
\usepackage{amsfonts}
\usepackage{amssymb}
\usepackage{pict2e}
\usepackage[x11names,svgnames]{xcolor} % (find-es "tex" "xcolor")
\usepackage{colorweb}                 % (find-es "tex" "colorweb")
%\usepackage{tikz}
%
% (find-dn6 "preamble6.lua" "preamble0")
%\usepackage{proof}   % For derivation trees ("%:" lines)
%\input diagxy        % For 2D diagrams ("%D" lines)
%\xyoption{curve}     % For the ".curve=" feature in 2D diagrams
%
\usepackage{edrx15}               % (find-LATEX "edrx15.sty")
\input edrxaccents.tex            % (find-LATEX "edrxaccents.tex")
\input edrxchars.tex              % (find-LATEX "edrxchars.tex")
\input edrxheadfoot.tex           % (find-LATEX "edrxheadfoot.tex")
\input edrxgac2.tex               % (find-LATEX "edrxgac2.tex")
%
%\usepackage[backend=biber,
%   style=alphabetic]{biblatex}            % (find-es "tex" "biber")
%\addbibresource{catsem-slides.bib}        % (find-LATEX "catsem-slides.bib")
%
% (find-es "tex" "geometry")
\usepackage[a6paper, landscape,
            top=1.5cm, bottom=.25cm, left=1cm, right=1cm, includefoot
           ]{geometry}
%
\begin{document}

\catcode`\^^J=10
\directlua{dofile "dednat6load.lua"}  % (find-LATEX "dednat6load.lua")

% %L dofile "edrxtikz.lua"  -- (find-LATEX "edrxtikz.lua")
% %L dofile "edrxpict.lua"  -- (find-LATEX "edrxpict.lua")
% \pu

% «defs»  (to ".defs")
% (find-LATEX "edrx15.sty" "colors-2019")
\long\def\ColorRed   #1{{\color{Red1}#1}}
\long\def\ColorViolet#1{{\color{MagentaVioletLight}#1}}
\long\def\ColorViolet#1{{\color{Violet!50!black}#1}}
\long\def\ColorGreen #1{{\color{SpringDarkHard}#1}}
\long\def\ColorGreen #1{{\color{SpringGreenDark}#1}}
\long\def\ColorGreen #1{{\color{SpringGreen4}#1}}
\long\def\ColorGray  #1{{\color{GrayLight}#1}}
\long\def\ColorGray  #1{{\color{black!30!white}#1}}
\long\def\ColorBrown #1{{\color{Brown}#1}}
\long\def\ColorBrown #1{{\color{brown}#1}}

\long\def\ColorShort #1{{\color{SpringGreen4}#1}}
\long\def\ColorLong  #1{{\color{Red1}#1}}

\def\frown{\ensuremath{{=}{(}}}
\def\True {\mathbf{V}}
\def\False{\mathbf{F}}
\def\Subst#1{\bmat{#1}}
\def\D{\displaystyle}

\def\drafturl{http://angg.twu.net/LATEX/2020-1-C2.pdf}
\def\drafturl{http://angg.twu.net/2020.1-C2.html}
\def\draftfooter{\tiny \href{\drafturl}{\jobname{}} \ColorBrown{\shorttoday{} \hours}}

% (find-angg ".emacs" "c2q192")


%  _____ _ _   _                               
% |_   _(_) |_| | ___   _ __   __ _  __ _  ___ 
%   | | | | __| |/ _ \ | '_ \ / _` |/ _` |/ _ \
%   | | | | |_| |  __/ | |_) | (_| | (_| |  __/
%   |_| |_|\__|_|\___| | .__/ \__,_|\__, |\___|
%                      |_|          |___/      
%
% «title»  (to ".title")

\thispagestyle{empty}

\begin{center}

\vspace*{1.2cm}

{\bf \Large Cálculo 2 - 2020.1}

\bsk

Aula 13: Integração por substituição

\bsk

Eduardo Ochs - RCN/PURO/UFF

\url{http://angg.twu.net/2020.1-C2.html}

\end{center}

\newpage

% «exemplo-com-gambs»  (to ".exemplo-com-gambs")
% (c2m201isubp 2 "exemplo-com-gambs")
% (c2m201isub    "exemplo-com-gambs")

{\bf Exemplo (VERSÃO COM GAMBIARRAS)}

$$\begin{array}{l}
  \D \intx{2 \cos(3x+4)} \\
  = \;\; \D \intu {2 (\cos u) · \frac13} \\
  = \;\; \D \frac23 \intu{\cos u} \\
  = \;\; \D \frac23 \sen u \\
  = \;\; \D \frac23 \sen (3x+4) \\
  \end{array}
  \qquad
  \Subst{
    u = 3x+4 \\ \frac{du}{dx} = 3 \\ du = 3 \, dx \\ dx = \frac13 \, du \\
  }
$$

\newpage

O ``bloquinho de substituições'' à direita é \ColorRed{parecido} com
as substituições da primeira aula no sentido de que ``algumas linhas
são consequências das linhas anteriores e estão lá só pra ajudar a
gente a se enrolar menos'' (veja os slides da primeira aula!) mas ele
é bem menos formal do que as substituições com `$:=$', e ele tem
várias gambiarras pesadas... por exemplo, o ``$dx = \frac13 \, du$'' é
algo que até é fácil de aprender a usar, mas que a gente vai demorar
pra conseguir formalizar.

% (c2m201introp 12 "acrescentamos")
% (c2m201intro     "acrescentamos")

\bsk

% «exercicio-1»  (to ".exercicio-1")
% (c2m201isubp 3 "exercicio-1")
% (c2m201isub    "exercicio-1")
{\bf Exercício 1.}

Assista este videozinho sobre como {\sl usar} estas gambiarras,

\ssk

\url{http://angg.twu.net/eev-videos/2020_int_subst_1.mp4}

% (find-ssr-links     "c2m201isub1" "2020_int_subst_1" "{hash}")
% (code-eevvideo      "c2m201isub1" "2020_int_subst_1" "{hash}")
% (code-eevlinksvideo "c2m201isub1" "2020_int_subst_1" "{hash}")
% (find-c2m201isub1video "0:00")
% (find-c2m201isub1video "8:00" "Exercicio")

\ssk

% \noindent
e faça o exercício do final dele: $\intx {(2x+3)^{10}} = ?$


\newpage

{\bf Mais diferenças}

\ssk

Repare que esse bloquinho de substituição com `$=$'s ao invés de
`$:=$'s fica solto à direita, longe das contas, ao invés de ficar
colado numa expressão específica... e ele é usado duas vezes, no
primeiro `$=$' e no último.

Além disso nós usamos ele pra transformar `$3x+4$' em `$u$' no
primeiro passo do exemplo. Os bloquinhos de substituição com `$:=$'s
têm uma sintaxe super rígida e eles só substituem {\sl variáveis}.
Veja:

\ssk

\url{http://angg.twu.net/LATEX/2020-1-C2-intro.pdf\#page=7}
% (c2m201introp 7 "subst-zoomed")
% (c2m201intro    "subst-zoomed")



\newpage

% «int-partes»  (to ".int-partes")
% (c2m201isubp 4 "int-partes")
% (c2m201isub    "int-partes")

{\bf Integração por partes}

Video:

\ssk

\url{http://angg.twu.net/eev-videos/2020_int_partes_1.mp4}
% (code-eevvideo "ipartes1" "2020_int_partes_1")
% (code-video "ipartes1video" "$S/http/angg.twu.net/eev-videos/2020_int_partes_1.mp4")
% (find-ipartes1video "6:00")

\msk

Definições:

$$\begin{array}{rcl}
  \relax [IP1] &=& \left( \D fg = \intx{f'g} + \intx{fg'} \right) \\
  \relax [IP2] &=& \left( \D \intx{f'g} = fg - \intx{fg'} \right) \\
  \relax [IP3] &=& \left( \D \intx{fg'} = fg - \intx{f'g} \right) \\
  \end{array}
$$

\newpage

{\bf Exercício 2:}

Calcule o resultado destas substituições:

a) $[IP1] \bmat{ f:=2x \\ g:= e^{3x} \\ f':= 2 \\ g':=3e^{3x}}$

\ssk

b) $[IP1] \bmat{ f:=2x \\ g:= e^{3x} \\}$

\ssk

c) $[IP2] \bmat{ f:=2x \\ g:= e^{3x} \\}$

\ssk

d) $[IP3] \bmat{ f:=2x \\ g:= e^{3x} \\}$

\newpage

{\bf Exercício 3.}

\ssk

a) Calcule $\intx {(2x) e^{3x}}$ usando o mesmo tipo de anotações sob
as expressões que eu usei no vídeo.

b) Verifique que só uma das regras IP2 e IP3 do vídeo funcionam pra
resolver o item anterior --- uma transforma aquela integral em algo
mais simples e a outra transforma ela em algo mais complicado.

c) Use o método do final do meu vídeo pra verificar se a sua resposta
esta certa.


%\printbibliography

\end{document}

 (eepitch-shell)
 (eepitch-kill)
 (eepitch-shell)
cd /tmp/
cp -v  ~/LATEX/2020-1-C2-int-subst.pdf /tmp/
xournalpp      2020-1-C2-int-subst.pdf

cd /tmp/
xournalpp



% (find-sh0 "cp -v  ~/LATEX/2020-1-C2-int-subst.pdf /tmp/")


%  __  __       _        
% |  \/  | __ _| | _____ 
% | |\/| |/ _` | |/ / _ \
% | |  | | (_| |   <  __/
% |_|  |_|\__,_|_|\_\___|
%                        
% <make>

 (eepitch-shell)
 (eepitch-kill)
 (eepitch-shell)
# (find-LATEXfile "2019planar-has-1.mk")
make -f 2019.mk STEM=2020-1-C2-int-subst veryclean
make -f 2019.mk STEM=2020-1-C2-int-subst pdf

% Local Variables:
% coding: utf-8-unix
% ee-tla: "c2m201isub"
% End:
