% (find-LATEX "2020-1-C3-derivs-alta-ordem.tex")
% (defun c () (interactive) (find-LATEXsh "lualatex -record 2020-1-C3-derivs-alta-ordem.tex" :end))
% (defun D () (interactive) (find-pdf-page      "~/LATEX/2020-1-C3-derivs-alta-ordem.pdf"))
% (defun d () (interactive) (find-pdftools-page "~/LATEX/2020-1-C3-derivs-alta-ordem.pdf"))
% (defun e () (interactive) (find-LATEX "2020-1-C3-derivs-alta-ordem.tex"))
% (defun u () (interactive) (find-latex-upload-links "2020-1-C3-derivs-alta-ordem"))
% (defun v () (interactive) (find-2a '(e) '(d)) (g))
% (find-pdf-page   "~/LATEX/2020-1-C3-derivs-alta-ordem.pdf")
% (find-sh0 "cp -v  ~/LATEX/2020-1-C3-derivs-alta-ordem.pdf /tmp/")
% (find-sh0 "cp -v  ~/LATEX/2020-1-C3-derivs-alta-ordem.pdf /tmp/pen/")
%   file:///home/edrx/LATEX/2020-1-C3-derivs-alta-ordem.pdf
%               file:///tmp/2020-1-C3-derivs-alta-ordem.pdf
%           file:///tmp/pen/2020-1-C3-derivs-alta-ordem.pdf
% http://angg.twu.net/LATEX/2020-1-C3-derivs-alta-ordem.pdf
% (find-LATEX "2019.mk")

\documentclass[oneside,12pt]{article}
\usepackage[colorlinks,citecolor=DarkRed,urlcolor=DarkRed]{hyperref} % (find-es "tex" "hyperref")
\usepackage{amsmath}
\usepackage{amsfonts}
\usepackage{amssymb}
\usepackage{pict2e}
\usepackage[x11names,svgnames]{xcolor} % (find-es "tex" "xcolor")
%\usepackage{colorweb}                 % (find-es "tex" "colorweb")
%\usepackage{tikz}
%
% (find-dn6 "preamble6.lua" "preamble0")
%\usepackage{proof}   % For derivation trees ("%:" lines)
%\input diagxy        % For 2D diagrams ("%D" lines)
%\xyoption{curve}     % For the ".curve=" feature in 2D diagrams
%
\usepackage{edrx15}               % (find-LATEX "edrx15.sty")
\input edrxaccents.tex            % (find-LATEX "edrxaccents.tex")
\input edrxchars.tex              % (find-LATEX "edrxchars.tex")
\input edrxheadfoot.tex           % (find-LATEX "edrxheadfoot.tex")
\input edrxgac2.tex               % (find-LATEX "edrxgac2.tex")
%
%\usepackage[backend=biber,
%   style=alphabetic]{biblatex}            % (find-es "tex" "biber")
%\addbibresource{catsem-slides.bib}        % (find-LATEX "catsem-slides.bib")
%
% (find-es "tex" "geometry")
\usepackage[a6paper, landscape,
            top=1.5cm, bottom=.25cm, left=1cm, right=1cm, includefoot
           ]{geometry}
%
\begin{document}

\catcode`\^^J=10
\directlua{dofile "dednat6load.lua"}  % (find-LATEX "dednat6load.lua")

% «defs»  (to ".defs")
% (find-LATEX "edrx15.sty" "colors-2019")
\long\def\ColorRed   #1{{\color{Red1}#1}}
\long\def\ColorViolet#1{{\color{MagentaVioletLight}#1}}
\long\def\ColorViolet#1{{\color{Violet!50!black}#1}}
\long\def\ColorGreen #1{{\color{SpringDarkHard}#1}}
\long\def\ColorGreen #1{{\color{SpringGreenDark}#1}}
\long\def\ColorGreen #1{{\color{SpringGreen4}#1}}
\long\def\ColorGray  #1{{\color{GrayLight}#1}}
\long\def\ColorGray  #1{{\color{black!30!white}#1}}
\long\def\ColorBrown #1{{\color{Brown}#1}}
\long\def\ColorBrown #1{{\color{brown}#1}}

\long\def\ColorShort #1{{\color{SpringGreen4}#1}}
\long\def\ColorLong  #1{{\color{Red1}#1}}

\def\frown{\ensuremath{{=}{(}}}
\def\True {\mathbf{V}}
\def\False{\mathbf{F}}

\def\drafturl{http://angg.twu.net/LATEX/2020-1-C2.pdf}
\def\drafturl{http://angg.twu.net/2020.1-C2.html}
\def\draftfooter{\tiny \href{\drafturl}{\jobname{}} \ColorBrown{\shorttoday{} \hours}}


%  _____ _ _   _                               
% |_   _(_) |_| | ___   _ __   __ _  __ _  ___ 
%   | | | | __| |/ _ \ | '_ \ / _` |/ _` |/ _ \
%   | | | | |_| |  __/ | |_) | (_| | (_| |  __/
%   |_| |_|\__|_|\___| | .__/ \__,_|\__, |\___|
%                      |_|          |___/      
%
% «title»  (to ".title")
% (c3m201Fxyp 1 "title")
% (c3m201Fxy    "title")

\thispagestyle{empty}

\begin{center}

\vspace*{1.2cm}

{\bf \Large Cálculo 3 - 2020.1}

\bsk

Aula 18: Derivadas parciais de ordem mais alta

\bsk

Eduardo Ochs - RCN/PURO/UFF

\url{http://angg.twu.net/2020.1-C3.html}

\end{center}

\newpage

% (c3m192p 5 "ponto-base")
% (c3m192    "ponto-base")

Às vezes você vai ver esse aviso aqui...

\ssk

\ColorRed{Contas fora do ponto base zeram a questão!}

\ssk

Se o nosso ponto base é $p_0 = (x_0,y_0)$ isso quer dizer que você

vai ter que evitar ao máximo fazer expansões como:
%
$$h(x,y) (x-x_0) \;\; \squigto \;\; h(x,y)·x + h(x,y)·(x_0)$$

E você vai ter que derivar esse $h(x,y) (x-x_0)$ assim:

$$\begin{array}{rcl}
  \frac{∂}{∂x} (h(x,y) (x-x_0)) &=& (\frac{∂}{∂x}h(x,y)) (x-x_0) + h(x,y) \frac{∂}{∂x}(x-x_0) \\
                                &=& h_x(x,y) (x-x_0) + h(x,y). \\
  \end{array}
$$

O mini-teste vai ter um aviso desses.

Isto vale também para $(y-y_0)$, $(x-x_0)^k$, e $(y-y_0)^k$.

\newpage


{\bf Exercício 0.}

Sejam:

$F(x,y) \; =\;  (x-x_0)^4 (y-y_0)^7$

e $(x_0,y_0) = (2,3)$.

\msk

Calcule:
%
$\begin{array}[t]{cccccccc}
 F(x,y),      & F_x(x,y),     & F_{xx}(x,y), \\
 F_y(x,y),    & F_{xy}(x,y),  & F_{xxy}(x,y), \\
 F_{yy}(x,y), & F_{xyy}(x,y), & F_{xxyy}(x,y), \\[5pt]
 %
 F(x_0,y_0),      & F_x(x_0,y_0),     & F_{xx}(x_0,y_0), \\
 F_y(x_0,y_0),    & F_{xy}(x_0,y_0),  & F_{xxy}(x_0,y_0), \\
 F_{yy}(x_0,y_0), & F_{xyy}(x_0,y_0), & F_{xxyy}(x_0,y_0), \\
 \end{array}
$

\msk

Dica: \ColorRed{não} substitua, por exemplo, $3^3 · 7^2$ por 1323 --

se você deixar como ``$3^3 · 7^2$'' vai dar pra ver os padrões,

e se você trocar isso por 1323 só alguém MUITO bom de conta

vai conseguir vê-los.

\newpage


{\bf Exercício 1.}

Seja:

$\scalebox{0.8}{$
 \begin{array}[t]{cccccccc}
 F(x,y) &=& a_{00}           &+& a_{10} (x-x_0)           &+& a_{20} (x-x_0)^2         \\
        &+& a_{01} (y-y_0)   &+& a_{11} (x-x_0) (y-y_0)   &+& a_{21} (x-x_0)^2 (y-y_0) \\
        &+& a_{02} (y-y_0)^2 &+& a_{12} (x-x_0) (y-y_0)^2 &+& a_{22} (x-x_0)^2 (y-y_0)^2. \\
 \end{array}
 $}
$

\bsk

Calcule:
%
$\begin{array}[t]{cccccccc}
 F(x,y),      & F_x(x,y),     & F_{xx}(x,y), \\
 F_y(x,y),    & F_{xy}(x,y),  & F_{xxy}(x,y), \\
 F_{yy}(x,y), & F_{xyy}(x,y), & F_{xxyy}(x,y), \\[5pt]
 %
 F(x_0,y_0),      & F_x(x_0,y_0),     & F_{xx}(x_0,y_0), \\
 F_y(x_0,y_0),    & F_{xy}(x_0,y_0),  & F_{xxy}(x_0,y_0), \\
 F_{yy}(x_0,y_0), & F_{xyy}(x_0,y_0), & F_{xxyy}(x_0,y_0), \\
 \end{array}
$

\bsk

Dica: dá pra fazer essas contas de cabeça depois que você descobrir
certos truques padrões. Faça as primeiras contas explicitamente no
papel, e depois descubra esses padrões.

% (c3q192 30 "20191108" "Exercícios sobre ponto base; regra da cadeia no Bortolossi")

\newpage

{\bf Exercício 2.}

Seja 

\msk

$\begin{array}[t]{cccccccc}
 G(x,y) &=& 4         &+& 5 (x-2)         &+& 6 (x-2)^2         \\
        &+& 7 (y-3)   &+& 8 (x-2) (y-3)   &+& 9 (x-2)^2 (y-3) \\
        &+& 10 (y-3)^2 &+& 11 (x-2) (y-3)^2 &+& 12 (x-2)^2 (y-3)^2. \\
 \end{array}
$

\msk

Calcule:
%
$\begin{array}[t]{cccccccc}
 G(2,3),      & G_x(2,3),     & G_{xx}(2,3), \\
 G_y(2,3),    & G_{xy}(2,3),  & G_{xxy}(2,3), \\
 G_{yy}(2,3), & G_{xyy}(2,3), & G_{xxyy}(2,3), \\
 \end{array}
$

\msk

(Dica: qual é o ponto base aqui?)

\newpage

Isso vai ter montes de aplicações -- por exemplo, os capítulos 11 e 12
do Bortolossi, que são sobre otimização, usam derivadas parciais de
ordem maior que 1 e aproximações de Taylor em $R^2$ a beça...

O que a gente está fazendo hoje é começar a entender quais são as
funções que são bem aproximadas pelas suas aproximações de Taylor (que
vamos ver em breve!) -- e a gente vai comecar por funções polinomiais.


\newpage

(Ainda não revisei a partir daqui...)

\msk


3) Leia a seção sobre Teorema de Young no Bortolossi. Dá pra aplicar o
teorema de Young nas funções $F$ e $G$?

\msk

4) Calcule todas as derivadas de 2ª ordem da função $F$. (Dica:
procure no Bortolossi a definição de "derivadas de 2ª ordem!)

\msk

5) Calcule todas as derivadas de 3ª ordem da função $H(x,y) = x^2 y_2$.

\msk

6) Especialize o Teorema 7.7 do Bortolossi para o caso $l=1$, $m=2$,
$n=1$. Obs: o livro tem alguns erros de digitação nesse teorema, e às
vezes ele troca `$l$'s por `$k$'s e `$k$'s por `$l$'s; considere que
todas as funções são de classe $C^k$. {\sl Escreva o seu resultado
  como um corolário.} Dica: leia as páginas 252 a 263 se precisar
tirar dúvidas sobre matriz jacobiana.

\msk

7) Use o seu corolário para calcular $\frac{d}{dt} F(g(t),h(t))$.

\msk

8) Use o que você obteve no (7) para calcular $\frac{d}{dt}
F(g(t),h(t))$ no caso em que $F(x,y) = x^2y^3$, $g(t) = \sen t$, $h(t)
= e^{4t}$.

\msk

9) Calcule $\frac{d}{dt} ((\sen t)^2(e^{4t})^3)$ usando métodos de
Cálculo 1.


%\printbibliography

\end{document}

%  __  __       _        
% |  \/  | __ _| | _____ 
% | |\/| |/ _` | |/ / _ \
% | |  | | (_| |   <  __/
% |_|  |_|\__,_|_|\_\___|
%                        
% <make>

 (eepitch-shell)
 (eepitch-kill)
 (eepitch-shell)
# (find-LATEXfile "2019planar-has-1.mk")
make -f 2019.mk STEM=2020-1-C3-derivs-alta-ordem veryclean
make -f 2019.mk STEM=2020-1-C3-derivs-alta-ordem pdf

% Local Variables:
% coding: utf-8-unix
% ee-tla: "c3m201Fxy"
% End:
