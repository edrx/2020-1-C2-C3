% (find-LATEX "2020-1-C3-dicas-p1.tex")
% (defun c () (interactive) (find-LATEXsh "lualatex -record 2020-1-C3-dicas-p1.tex" :end))
% (defun D () (interactive) (find-pdf-page      "~/LATEX/2020-1-C3-dicas-p1.pdf"))
% (defun d () (interactive) (find-pdftools-page "~/LATEX/2020-1-C3-dicas-p1.pdf"))
% (defun e () (interactive) (find-LATEX "2020-1-C3-dicas-p1.tex"))
% (defun u () (interactive) (find-latex-upload-links "2020-1-C3-dicas-p1"))
% (defun v () (interactive) (find-2a '(e) '(d)) (g))
% (find-pdf-page   "~/LATEX/2020-1-C3-dicas-p1.pdf")
% (find-sh0 "cp -v  ~/LATEX/2020-1-C3-dicas-p1.pdf /tmp/")
% (find-sh0 "cp -v  ~/LATEX/2020-1-C3-dicas-p1.pdf /tmp/pen/")
%   file:///home/edrx/LATEX/2020-1-C3-dicas-p1.pdf
%               file:///tmp/2020-1-C3-dicas-p1.pdf
%           file:///tmp/pen/2020-1-C3-dicas-p1.pdf
% http://angg.twu.net/LATEX/2020-1-C3-dicas-p1.pdf
% (find-LATEX "2019.mk")
% (find-C3-aula-links "2020-1-C3-dicas-p1" "dp1" "dp1")

% «.dicas»		(to "dicas")
% «.aulas-7-e-8»	(to "aulas-7-e-8")
% «.aula-13»		(to "aula-13")

\documentclass[oneside,12pt]{article}
\usepackage[colorlinks,citecolor=DarkRed,urlcolor=DarkRed]{hyperref} % (find-es "tex" "hyperref")
\usepackage{amsmath}
\usepackage{amsfonts}
\usepackage{amssymb}
\usepackage{pict2e}
\usepackage[x11names,svgnames]{xcolor} % (find-es "tex" "xcolor")
%\usepackage{colorweb}                 % (find-es "tex" "colorweb")
%\usepackage{tikz}
%
% (find-dn6 "preamble6.lua" "preamble0")
%\usepackage{proof}   % For derivation trees ("%:" lines)
%\input diagxy        % For 2D diagrams ("%D" lines)
%\xyoption{curve}     % For the ".curve=" feature in 2D diagrams
%
\usepackage{edrx15}               % (find-LATEX "edrx15.sty")
\input edrxaccents.tex            % (find-LATEX "edrxaccents.tex")
\input edrxchars.tex              % (find-LATEX "edrxchars.tex")
\input edrxheadfoot.tex           % (find-LATEX "edrxheadfoot.tex")
\input edrxgac2.tex               % (find-LATEX "edrxgac2.tex")
%
%\usepackage[backend=biber,
%   style=alphabetic]{biblatex}            % (find-es "tex" "biber")
%\addbibresource{catsem-slides.bib}        % (find-LATEX "catsem-slides.bib")
%
% (find-es "tex" "geometry")
\usepackage[a6paper, landscape,
            top=1.5cm, bottom=.25cm, left=1cm, right=1cm, includefoot
           ]{geometry}
%
\begin{document}

\catcode`\^^J=10
\directlua{dofile "dednat6load.lua"}  % (find-LATEX "dednat6load.lua")

% %L dofile "edrxtikz.lua"  -- (find-LATEX "edrxtikz.lua")
% %L dofile "edrxpict.lua"  -- (find-LATEX "edrxpict.lua")
% \pu

% «defs»  (to ".defs")
% (find-LATEX "edrx15.sty" "colors-2019")
\long\def\ColorRed   #1{{\color{Red1}#1}}
\long\def\ColorViolet#1{{\color{MagentaVioletLight}#1}}
\long\def\ColorViolet#1{{\color{Violet!50!black}#1}}
\long\def\ColorGreen #1{{\color{SpringDarkHard}#1}}
\long\def\ColorGreen #1{{\color{SpringGreenDark}#1}}
\long\def\ColorGreen #1{{\color{SpringGreen4}#1}}
\long\def\ColorGray  #1{{\color{GrayLight}#1}}
\long\def\ColorGray  #1{{\color{black!30!white}#1}}
\long\def\ColorBrown #1{{\color{Brown}#1}}
\long\def\ColorBrown #1{{\color{brown}#1}}

\long\def\ColorShort #1{{\color{SpringGreen4}#1}}
\long\def\ColorLong  #1{{\color{Red1}#1}}

\def\frown{\ensuremath{{=}{(}}}
\def\True {\mathbf{V}}
\def\False{\mathbf{F}}

\def\drafturl{http://angg.twu.net/LATEX/2020-1-C2.pdf}
\def\drafturl{http://angg.twu.net/2020.1-C2.html}
\def\draftfooter{\tiny \href{\drafturl}{\jobname{}} \ColorBrown{\shorttoday{} \hours}}


%  _____ _ _   _                               
% |_   _(_) |_| | ___   _ __   __ _  __ _  ___ 
%   | | | | __| |/ _ \ | '_ \ / _` |/ _` |/ _ \
%   | | | | |_| |  __/ | |_) | (_| | (_| |  __/
%   |_| |_|\__|_|\___| | .__/ \__,_|\__, |\___|
%                      |_|          |___/      
%
% «title»  (to ".title")
% (c3m201dp1p 1 "title")
% (c3m201dp1a   "title")

\thispagestyle{empty}

\begin{center}

\vspace*{1.2cm}

{\bf \Large Cálculo 3 - 2020.1}

\bsk

Dicas pra estudar pra P1

\bsk

Eduardo Ochs - RCN/PURO/UFF

\url{http://angg.twu.net/2020.1-C3.html}

\end{center}

\newpage

% «dicas»  (to ".dicas")
% (c3m201dp1p 2 "dicas")
% (c3m201dp1    "dicas")

{\bf Dicas}

% (c3m201introp)
% (c3m201intro)

\ssk

Aula 2: Vetores tangentes em $\R^2$

Refaça o exercício 4.

% (c3m201vtp)
% (c3m201vt)
% (c3m201vtp 8 "exercicio-4")
% (c3m201vt    "exercicio-4")

\msk

Aulas 3 e 4: Aproximações de 1ª e 2ª ordem

Refaça o exercício 5.

% (c3m201taylor1p)
% (c3m201taylor1)
% (c3m201taylor1p 18 "exercicio-5")
% (c3m201taylor1     "exercicio-5")

\msk

Aula 5 e 6: Aproximações de 1ª e 2ª ordem: algumas aplicações

Refaça os exercícios 1 até 4.

Tem um scan da resposta no próximo PDF.

% (c3m201taylor2p)
% (c3m201taylor2)

\msk

% «aulas-7-e-8»  (to ".aulas-7-e-8")
% (c3m201dp1p 2 "aulas-7-e-8")
% (c3m201dp1    "aulas-7-e-8")

Aulas 7 e 8: dx, $Δx$ e série de Taylor

Refaça os exercícios 1 e 4.

% (c3m201taylor3p)
% (c3m201taylor3)
% (c3m201taylor3p 5 "exercicio-1")
% (c3m201taylor3    "exercicio-1")
% (c3m201taylor3p 9 "exercicio-4")
% (c3m201taylor3    "exercicio-4")

\msk

Aulas 9 e 10: introdução a superfícies e curvas de nível

Refaça os exercícios 1a, 1b, 1c.

Refaça os exercícios 2a e 2b.

% (c3m201sups1p)
% (c3m201sups1)

\newpage

Aulas 11 e 12: Alguns truques para visualizar superfícies

Tente fazer os exercícios 1b--1f sem fazer o diagrama de

numerozinhos.

% (c3m201sups2p)
% (c3m201sups2)

\msk

% «aula-13»  (to ".aula-13")
% (c3m201dp1p 3 "aula-13")
% (c3m201dp1    "aula-13")
% (c3m201derpsp)
% (c3m201derps)

Aula 13: Derivadas parciais

Marque só estes pontos no diagram de numerozinhos:

$$\begin{array}{lll}
              & F(2,5) = 0 & \\
F(1,4) ≈ 2.83 & F(2,4) ≈ 2.24 & F(3,4) = 0 \\
              & F(2,3) ≈ 3.46 & \\
\end{array}
$$

e use-os para fazer o exercício 2.

Releia o exercício 3, sobre tipar tudo --- ele é muito importante.

% (eepitch-lua51)
% (eepitch-kill)
% (eepitch-lua51)
% f = function (x,y)
%     local z = math.sqrt(25 - x*x - y*y)
%     print(x, y, z)
%   end
% f(2,4)
% f(2,4+1)
% f(2,4-1)
% f(2+1,4)
% f(2-1,4)

\msk


% (c3m201pltanp)
% (c3m201pltan)
% (c3m201pltanp 4 "exercicio-2")
% (c3m201pltan    "exercicio-2")
% (c3m201pltanp 9 "exercicio-4")
% (c3m201pltan    "exercicio-4")

Aula 14 e 15: Introdução a planos tangentes e à derivada

Refaça os exercícios 2a, 2g--2k, 4e, 4f.

\newpage

Aula 18: Derivadas parciais de ordem mais alta

Já revisamos no mini-teste.

% (c3m201Fxyp)
% (c3m201Fxy)

\msk

% (c3m201aprox2aop)
% (c3m201aprox2ao)

Aula 19: aproximações de 2ª ordem em $\R^2$

Faça no exercício 2 o item a e o caso 5 do item b.

\msk

% (c3m201dp1p)
% (c3m201dp1)




\end{document}

%  __  __       _        
% |  \/  | __ _| | _____ 
% | |\/| |/ _` | |/ / _ \
% | |  | | (_| |   <  __/
% |_|  |_|\__,_|_|\_\___|
%                        
% <make>

 (eepitch-shell)
 (eepitch-kill)
 (eepitch-shell)
# (find-LATEXfile "2019planar-has-1.mk")
make -f 2019.mk STEM=2020-1-C3-dicas-p1 veryclean
make -f 2019.mk STEM=2020-1-C3-dicas-p1 pdf

% Local Variables:
% coding: utf-8-unix
% ee-tla: "c3m201dp1"
% End:
