% (find-LATEX "2020-1-C3-P2.tex")
% (defun c () (interactive) (find-LATEXsh "lualatex -record 2020-1-C3-P2.tex" :end))
% (defun C () (interactive) (find-LATEXsh "lualatex 2020-1-C3-P2.tex" "Success!!!"))
% (defun D () (interactive) (find-pdf-page      "~/LATEX/2020-1-C3-P2.pdf"))
% (defun d () (interactive) (find-pdftools-page "~/LATEX/2020-1-C3-P2.pdf"))
% (defun e () (interactive) (find-LATEX "2020-1-C3-P2.tex"))
% (defun u () (interactive) (find-latex-upload-links "2020-1-C3-P2"))
% (defun v () (interactive) (find-2a '(e) '(d)))
% (defun cv () (interactive) (C) (ee-kill-this-buffer) (v) (g))
% (find-pdf-page   "~/LATEX/2020-1-C3-P2.pdf")
% (find-sh0 "cp -v  ~/LATEX/2020-1-C3-P2.pdf /tmp/")
% (find-sh0 "cp -v  ~/LATEX/2020-1-C3-P2.pdf /tmp/pen/")
%   file:///home/edrx/LATEX/2020-1-C3-P2.pdf
%               file:///tmp/2020-1-C3-P2.pdf
%           file:///tmp/pen/2020-1-C3-P2.pdf
% http://angg.twu.net/LATEX/2020-1-C3-P2.pdf
% (find-LATEX "2019.mk")
% (find-C3-aula-links "2020-1-C3-P2" "p2" "p2")

% «.regras»		(to "regras")
% «.questao-1»		(to "questao-1")
% «.questao-2»		(to "questao-2")
%   «.estudo-de-sinal»	(to "estudo-de-sinal")
% «.questao-3»		(to "questao-3")

\documentclass[oneside,12pt]{article}
\usepackage[colorlinks,citecolor=DarkRed,urlcolor=DarkRed]{hyperref} % (find-es "tex" "hyperref")
\usepackage{amsmath}
\usepackage{amsfonts}
\usepackage{amssymb}
\usepackage{pict2e}
\usepackage[x11names,svgnames]{xcolor} % (find-es "tex" "xcolor")
%\usepackage{colorweb}                 % (find-es "tex" "colorweb")
%\usepackage{tikz}
%
% (find-dn6 "preamble6.lua" "preamble0")
%\usepackage{proof}   % For derivation trees ("%:" lines)
%\input diagxy        % For 2D diagrams ("%D" lines)
%\xyoption{curve}     % For the ".curve=" feature in 2D diagrams
%
\usepackage{edrx15}               % (find-LATEX "edrx15.sty")
\input edrxaccents.tex            % (find-LATEX "edrxaccents.tex")
\input edrxchars.tex              % (find-LATEX "edrxchars.tex")
\input edrxheadfoot.tex           % (find-LATEX "edrxheadfoot.tex")
\input edrxgac2.tex               % (find-LATEX "edrxgac2.tex")
%
%\usepackage[backend=biber,
%   style=alphabetic]{biblatex}            % (find-es "tex" "biber")
%\addbibresource{catsem-slides.bib}        % (find-LATEX "catsem-slides.bib")
%
% (find-es "tex" "geometry")
\usepackage[%a6paper, landscape,
            %paperheight=105mm, paperwidth=148mm,
            paperheight=105mm, paperwidth=155mm,
            top=1.5cm, bottom=.25cm, left=1cm, right=1cm, includefoot
           ]{geometry}
%
\begin{document}

\catcode`\^^J=10
\directlua{dofile "dednat6load.lua"}  % (find-LATEX "dednat6load.lua")

% %L dofile "edrxtikz.lua"  -- (find-LATEX "edrxtikz.lua")
% %L dofile "edrxpict.lua"  -- (find-LATEX "edrxpict.lua")
% \pu

% «defs»  (to ".defs")
% (find-LATEX "edrx15.sty" "colors-2019")
\long\def\ColorRed   #1{{\color{Red1}#1}}
\long\def\ColorViolet#1{{\color{MagentaVioletLight}#1}}
\long\def\ColorViolet#1{{\color{Violet!50!black}#1}}
\long\def\ColorGreen #1{{\color{SpringDarkHard}#1}}
\long\def\ColorGreen #1{{\color{SpringGreenDark}#1}}
\long\def\ColorGreen #1{{\color{SpringGreen4}#1}}
\long\def\ColorGray  #1{{\color{GrayLight}#1}}
\long\def\ColorGray  #1{{\color{black!30!white}#1}}
\long\def\ColorBrown #1{{\color{Brown}#1}}
\long\def\ColorBrown #1{{\color{brown}#1}}

\long\def\ColorShort #1{{\color{SpringGreen4}#1}}
\long\def\ColorLong  #1{{\color{Red1}#1}}

\def\frown{\ensuremath{{=}{(}}}
\def\True {\mathbf{V}}
\def\False{\mathbf{F}}

\def\drafturl{http://angg.twu.net/LATEX/2020-1-C2.pdf}
\def\drafturl{http://angg.twu.net/2020.1-C2.html}
\def\draftfooter{\tiny \href{\drafturl}{\jobname{}} \ColorBrown{\shorttoday{} \hours}}

\setlength{\parindent}{0em}
\def\T(Total: #1 pts){{\bf(Total: #1 pts)}}
\def\T(Total: #1 pts){{\bf(Total: #1)}}
\def\B       (#1 pts){{\bf(#1 pts)}}
% Usage:
% 1) \T(Total: 2.34 pts) Foo
% a) \B(0.45 pts) Bar


%  _____ _ _   _                               
% |_   _(_) |_| | ___   _ __   __ _  __ _  ___ 
%   | | | | __| |/ _ \ | '_ \ / _` |/ _` |/ _ \
%   | | | | |_| |  __/ | |_) | (_| | (_| |  __/
%   |_| |_|\__|_|\___| | .__/ \__,_|\__, |\___|
%                      |_|          |___/      
%
% «title»  (to ".title")
% (c3m201p2p 1 "title")
% (c3m201p2a   "title")

\thispagestyle{empty}

\begin{center}

\vspace*{1.2cm}

{\bf \Large Cálculo 3 - 2020.1}

\bsk

P2 (Segunda prova)

\bsk

Eduardo Ochs - RCN/PURO/UFF

\url{http://angg.twu.net/2020.1-C3.html}

\end{center}

\newpage

% «regras»  (to ".regras")
% (c3m201p2p 2 "regras")
% (c3m201p2    "regras")
% (c3m201p1p 2 "regras")
% (c3m201p1    "regras")

{\bf Regras para a P2:}

\ssk

As questões da P2 serão disponibilizadas às 18:00 da
quarta-feira 02/dezv/2020 e você deverá entregar as respostas
\ColorRed{escritas à mão} até as 18:00 da sexta 04/dez/2020 na
plataforma Classroom. Se o Classroom der algum problema mande também
para este endereço de e-mail:


\ssk

\ColorRed{eduardoochs@gmail.com}

\ssk

Provas entregues após este horário não serão considerados.

Durante as 24 horas do mini-teste o professor não responderá perguntas
sobre os assuntos do mini-teste, mas você pode discutir com os seus
colegas... \ColorRed{só que as respostas devem ser individuais}.

\newpage

% «questao-1»  (to ".questao-1")
% (c3m201p2p 3 "questao-1")
% (c3m201p2    "questao-1")

{\bf Questão 1}

\T(Total: 5.0 pts)

\ssk

Sejam:
%
$$\def\r#1{\ColorRed{\,#1}}
  \begin{array}{rcl}
  A &=& \setofxyst{0≤x\r{≤5}, \; 0≤y\r{≤5}, \; d((x,y),(5,5))>5}, \\
  B &=& \setofxyst{0≤x\r{≤5}, \; 0≤y\r{≤5}, \; d((x,y),(5,5))≥5}, \\
  F(x,y) &=& d((x,y),(1,2))^2 \\
         &=& (x-1)^2 + (y-2)^2. \\
  \end{array}
$$


a) \B(0.5 pts) Represente $A$ graficamente.

b) \B(0.5 pts) Represente $B$ graficamente.

c) \B(1.0 pts) Dos dois conjuntos $A$ e $B$ um é compacto e outro
não.

Descubra qual é qual e explique porquê.

\newpage

{\bf Questão 1 (continuação)}

\ssk

A partir daqui $C$ é o conjunto compacto do

item anterior e $N$ é o conjunto não compacto.

\msk

d) \B(0.5 pts) Represente graficamente as curvas de nível de $F(x,y)$ em $\R^2$.

e) \B(0.5 pts) Represente graficamente as curvas de nível de $F(x,y)$ em $C$.

f) \B(1.0 pts) Represente graficamente as curvas de nível de $F(x,y)$ em $N$.

(Aqui $N$ é o ``conjunto admissível da $F$''; veja o cap.10 do Bortolossi.)

\msk

g) \B(1.0 pts) Use as curvas de nível pra mostrar que a função $F$
assume valores positivos arbitrariamente próximos de zero em $N$ mas
não assume o valor zero em $N$; use isto pra mostrar que esta $F$ não
tem mínimo global em $N$. Use desenhos e português pra explicar as
suas idéias.

% (find-books "__analysis/__analysis.el" "bortolossi")
% (find-bortolossi10page (+ -350 351) "10. Máximos e mínimos de funções de várias variáveis")


\newpage

% «questao-2»  (to ".questao-2")
% (c3m201p2p 5 "questao-2")
% (c3m201p2    "questao-2")

{\bf Questão 2}

\T(Total: 3.0 pts)

\ssk

Sejam
%
$$\begin{array}{rcl}
         (x_0,y_0) &=& (5,2), \\
  F(x_0+Δx,y_0+Δx) &=& Δx^2 + ΔxΔy - 6Δy^2, \quad \text{isto é}, \\
            F(x,y) &=& (x-x_0)^2 + (x-x_0)(y-y_0) - 6(y-y_0)^2, \quad \text{e} \\
               H_k &=& \setofxyst{y=y_0+k}.
  \end{array}
$$

a) \B(0.5 pts) Represente graficamente os conjuntos $H_1, H_{0.1},
H_0, H_{-0.1}, H_{-1}$.

b) \B(1.0 pts) Faça o ``estudo de sinal'' da função $F$ no conjunto
$H_1$. Dica:

\ssk

% «estudo-de-sinal»  (to ".estudo-de-sinal")
{\footnotesize
\url{http://www.matematica.pucminas.br/profs/web_fabiano/calculo1/sinal.pdf}
}

\msk

\newpage

{\bf Questão 2 (continuação)}

(Dica: refaça os exercícios da aula 19!)

\ssk



% (c3m201aprox2aop 6 "exercicio-2")
% (c3m201aprox2ao    "exercicio-2")


c) \B(0.5 pts) Transporte o que você descobriu sobre a $F$ em $H_1$
para $H_{-1}$.

d) \B(0.5 pts) Transporte o que você descobriu sobre a $F$ em $H_1$
para $H_{0.1}$.

e) \B(0.5 pts) Esta função $F$ não tem mínimo global em $\R^2$.
Explique por quê.

\newpage

% «questao-3»  (to ".questao-3")
% (c3m201p2p 7 "questao-3")
% (c3m201p2    "questao-3")

{\bf Questão 3}

\T(Total: 3.0 pts)

\ssk

Sejam
%
$$\begin{array}{rcl}
         (x_0,y_0) &=& (5,2), \\
            F(x,y) &=& (x-x_0)^2 + 4(x-x_0)(y-y_0) + 5 \ColorRed{(y-y_0)^2}, \quad \text{e} \\
               H_k &=& \setofxyst{y=y_0+k}.
  \end{array}
$$

a) \B(0.5 pts) Faça o ``estudo de sinal'' da função $F$ no conjunto
$H_1$.

b) \B(0.5 pts) Encontre o mínimo global da $F$ em $H_1$.

c) \B(0.5 pts) Transporte o que você descobriu para $H_{-1}$.

d) \B(0.5 pts) Transporte o que você descobriu para $H_{0.1}$.

e) \B(0.5 pts) Encontre o mínimo global da $F$ em $H_0$.

f) \B(0.5 pts) Esta função $F$ tem mínimo global em $\R^2$. Explique
por quê.





\GenericWarning{Success:}{Success!!!}

\end{document}

%  __  __       _        
% |  \/  | __ _| | _____ 
% | |\/| |/ _` | |/ / _ \
% | |  | | (_| |   <  __/
% |_|  |_|\__,_|_|\_\___|
%                        
% <make>

 (eepitch-shell)
 (eepitch-kill)
 (eepitch-shell)
# (find-LATEXfile "2019planar-has-1.mk")
make -f 2019.mk STEM=2020-1-C3-P2 veryclean
make -f 2019.mk STEM=2020-1-C3-P2 pdf

% Local Variables:
% coding: utf-8-unix
% ee-tla: "c3m201p2"
% End:
