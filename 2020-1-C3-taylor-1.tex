% (find-LATEX "2020-1-C3-taylor-1.tex")
% (defun c () (interactive) (find-LATEXsh "lualatex -record 2020-1-C3-taylor-1.tex" :end))
% (defun D () (interactive) (find-pdf-page      "~/LATEX/2020-1-C3-taylor-1.pdf"))
% (defun d () (interactive) (find-pdftools-page "~/LATEX/2020-1-C3-taylor-1.pdf"))
% (defun e () (interactive) (find-LATEX "2020-1-C3-taylor-1.tex"))
% (defun u () (interactive) (find-latex-upload-links "2020-1-C3-taylor-1"))
% (defun v () (interactive) (find-2a '(e) '(d)) (g))
% (find-pdf-page   "~/LATEX/2020-1-C3-taylor-1.pdf")
% (find-sh0 "cp -v  ~/LATEX/2020-1-C3-taylor-1.pdf /tmp/")
% (find-sh0 "cp -v  ~/LATEX/2020-1-C3-taylor-1.pdf /tmp/pen/")
%   file:///home/edrx/LATEX/2020-1-C3-taylor-1.pdf
%               file:///tmp/2020-1-C3-taylor-1.pdf
%           file:///tmp/pen/2020-1-C3-taylor-1.pdf
% http://angg.twu.net/LATEX/2020-1-C3-taylor-1.pdf
% (find-LATEX "2019.mk")

% «.defs»		(to "defs")
% «.title»		(to "title")
% «.polis-e-fpolis»	(to "polis-e-fpolis")
% «.taylor-1»		(to "taylor-1")
% «.taylor-2»		(to "taylor-2")
% «.taylor-3»		(to "taylor-3")
% «.taylor-4»		(to "taylor-4")
% «.taylor-5»		(to "taylor-5")
% «.taylor-6»		(to "taylor-6")
% «.taylor-7»		(to "taylor-7")
% «.taylor-8»		(to "taylor-8")
% «.taylors-famosas»	(to "taylors-famosas")
% «.exercicio-4»	(to "exercicio-4")
% «.exercicio-5»	(to "exercicio-5")

\documentclass[oneside,12pt]{article}
\usepackage[colorlinks,citecolor=DarkRed,urlcolor=DarkRed]{hyperref} % (find-es "tex" "hyperref")
\usepackage{amsmath}
\usepackage{amsfonts}
\usepackage{amssymb}
\usepackage{pict2e}
\usepackage[x11names,svgnames]{xcolor} % (find-es "tex" "xcolor")
%\usepackage{colorweb}                 % (find-es "tex" "colorweb")
%\usepackage{tikz}
%
% (find-dn6 "preamble6.lua" "preamble0")
%\usepackage{proof}   % For derivation trees ("%:" lines)
%\input diagxy        % For 2D diagrams ("%D" lines)
%\xyoption{curve}     % For the ".curve=" feature in 2D diagrams
%
\usepackage{edrx15}               % (find-LATEX "edrx15.sty")
\input edrxaccents.tex            % (find-LATEX "edrxaccents.tex")
\input edrxchars.tex              % (find-LATEX "edrxchars.tex")
\input edrxheadfoot.tex           % (find-LATEX "edrxheadfoot.tex")
\input edrxgac2.tex               % (find-LATEX "edrxgac2.tex")
%
%\usepackage[backend=biber,
%   style=alphabetic]{biblatex}            % (find-es "tex" "biber")
%\addbibresource{catsem-slides.bib}        % (find-LATEX "catsem-slides.bib")
%
% (find-es "tex" "geometry")
\usepackage[a6paper, landscape,
            top=1.5cm, bottom=.25cm, left=1cm, right=1cm, includefoot
           ]{geometry}
%
\begin{document}

\catcode`\^^J=10
\directlua{dofile "dednat6load.lua"}  % (find-LATEX "dednat6load.lua")

% %L dofile "edrxtikz.lua"  -- (find-LATEX "edrxtikz.lua")
% %L dofile "edrxpict.lua"  -- (find-LATEX "edrxpict.lua")
% \pu

%\printbibliography


% «defs»  (to ".defs")
% (find-LATEX "edrx15.sty" "colors-2019")
\long\def\ColorRed   #1{{\color{Red1}#1}}
\long\def\ColorViolet#1{{\color{MagentaVioletLight}#1}}
\long\def\ColorViolet#1{{\color{Violet!50!black}#1}}
\long\def\ColorGreen #1{{\color{SpringDarkHard}#1}}
\long\def\ColorGreen #1{{\color{SpringGreenDark}#1}}
\long\def\ColorGreen #1{{\color{SpringGreen4}#1}}
\long\def\ColorGray  #1{{\color{GrayLight}#1}}
\long\def\ColorGray  #1{{\color{black!30!white}#1}}
\long\def\ColorBrown #1{{\color{Brown}#1}}
\long\def\ColorBrown #1{{\color{brown}#1}}

\long\def\ColorShort #1{{\color{SpringGreen4}#1}}
\long\def\ColorLong  #1{{\color{Red1}#1}}

\def\frown{\ensuremath{{=}{(}}}
\def\derivs{\mathsf{derivs}}
\def\True {\mathbf{V}}
\def\False{\mathbf{F}}

\def\drafturl{http://angg.twu.net/LATEX/2020-1-C2.pdf}
\def\drafturl{http://angg.twu.net/2020.1-C2.html}
\def\draftfooter{\tiny \href{\drafturl}{\jobname{}} \ColorBrown{\shorttoday{} \hours}}


%  _____ _ _   _                               
% |_   _(_) |_| | ___   _ __   __ _  __ _  ___ 
%   | | | | __| |/ _ \ | '_ \ / _` |/ _` |/ _ \
%   | | | | |_| |  __/ | |_) | (_| | (_| |  __/
%   |_| |_|\__|_|\___| | .__/ \__,_|\__, |\___|
%                      |_|          |___/      
%
% «title»  (to ".title")
% (c3m201taylor1p 1 "title")
% (c3m201taylor1    "title")

\thispagestyle{empty}

\begin{center}

\vspace*{1.2cm}

{\bf \Large Cálculo 3 - 2020.1}

\bsk

Aulas 3 e 4: Aproximações de 1ª e 2ª ordem

\bsk

Eduardo Ochs - RCN/PURO/UFF

\url{http://angg.twu.net/2020.1-C3.html}

\end{center}

\newpage

No final da última aula eu passei dois exercícios sobre ``adivinhar''
trajetórias a partir dos valores de $P(t)$ e $P'(t)$ para alguns
valores de $t$. O exercício 3 era mais difícil que o 4 -- no 3 a
velocidade $P'(t)$ era zero em alguns dos pontos fáceis de calcular, e
o melhor modo da gente descobrir o comportamento da trajetória $P(t)$
em torno daqueles pontos é usando o \ColorRed{vetor aceleração},
$P''(t)$, que é um dos assuntos de hoje.

\bsk

{\bf Importante:} por um erro de digitação eu acabei passando o
exercício mais difícil antes do mais fácil, e acabei só mostrando o
enunciado do 4 (mais fácil) pra algumas poucas pessoas por Telegram e
só acrescentei ele ao PDF depois da aula... então: 

\ColorRed{Comece refazendo o exercício 3 da aula passada e fazendo o
  4.}


\newpage

% «polis-e-fpolis»  (to ".polis-e-fpolis")
% (c3m201taylor1p 4 "polis-e-fpolis")
% (c3m201taylor1    "polis-e-fpolis")

{\bf Polinômios e funções polinomiais}

Alguns (poucos) livros distinguem {\sl polinômios} de {\sl funções
  polinomiais}. Um {\sl polinômio de grau $n$ em $x$} é uma expressão
da forma
%
$$ a_n x^n + a_{n-1} x^{n-1} + \ldots + a_1 x + a_0, $$
%
onde $a_n, a_{n-1}, \ldots, a_0$ são constantes, e uma {\sl função
  polinomial em $x$} é uma função $g(x)$ para a qual existe um
polinômio $f(x)$ tal que $g(x)=f(x)$ para todo $x$. Por exemplo,
%
$$ 42 (x-99)^{200} - 12 (x-99)^6 $$
%
é uma função polinomial em $x$ mas não um polinômio em $x$, porque os
coeficientes $a_{200}, \ldots, a_0$ do polinômio não são dados
explicitamente.

\msk

\newpage

{\bf Polinômios e funções polinomiais (2)}

...maaaas repare que $42 (x-99)^{200} - 12 (x-99)^6$ é um polinômio
\ColorRed{em $x-99$}, de grau 200 e com coeficientes $b_{200}=42$,
$b_6=12$, e zero nos outros índices. Mais formalmente,
%
$$\sum_{k=0}^{200} b_k (x - 99)^k = 42 (x-99)^{200} - 12 (x-99)^6$$
%
quando $b_{200}=42$, $b_6=12$, e $b_k=0$ nos outros índices.

\bsk

(Isto vai ser útil para séries de Taylor...)



\newpage

% «taylor-1»  (to ".taylor-1")
% (c3m201taylor1p 5 "taylor-1")
% (c3m201taylor1    "taylor-1")

{\bf Mini-revisão de séries de Taylor}

Nos meus cursos de Cálculo 2 eu costumo fazer uma introdução rápida a
Séries de Taylor pra convencer as pessoas de que a fórmula abaixo é
verdade...
%
$$e^{iθ} = \cosθ + i\senθ \qquad\qquad (*)$$

Se $f:\R → \R$ a \ColorRed{Série de Taylor de $f$ em no ponto 0} é:
%
$$f(x) = \sum_{k=0}^∞ \frac{f^{(k)}(0)}{k!} x^k \qquad\qquad (**)$$ 
%
onde $f^{(0)} = f$, $f^{(1)} = f'$, $f^{(2)} = f''$, etc.

\newpage

% «taylor-2»  (to ".taylor-2")
% (c3m201taylor1p 6 "taylor-2")
% (c3m201taylor1    "taylor-2")

{\bf Mini-revisão de séries de Taylor (2)}

Sejam $\derivs$ e $\derivs_0$ as seguintes operações:

$\derivs(f) = (f, f', f'', f''', \ldots)$

$\derivs_0(f) = (f(0), f'(0), f''(0), f'''(0), \ldots)$

Repare que $\derivs(f)$ retorna uma sequência infinita de funções e
$\derivs(f)$ retorna uma sequência infinita de números.

Um exemplo: se $f(x) = ax^2 + bx + c$, então:
%
$$\begin{array}{rclcrcl}
  f(x)   &=& ax^2 + bx + c,    && f(0)   &=& c, \\
  f'(x)  &=& 2ax + b,          && f'(0)  &=& b, \\
  f''(x) &=& 2a,               && f''(0) &=& 2a, \\
  f'''(x) &=& 0,               && f'''(0) &=& 0, \\
  \end{array}
$$

$\derivs(f) = (ax^2 + bx + c, \; 2ax + b, \; 2a, \; 0, 0, 0, \ldots)$

$\derivs_0(f) = (c, b, 2a, 0, 0, 0, \ldots)$

\newpage

{\bf Mini-revisão de séries de Taylor (3)}

% «taylor-3»  (to ".taylor-3")
% (c3m201taylor1p 7 "taylor-3")
% (c3m201taylor1    "taylor-3")

...e neste caso os termos do somatório são todos zero

a partir de $k=3$:
%
$$\begin{array}{rcl}
  f(x) &=& \displaystyle \sum_{k=0}^∞ \frac{f^{(k)}(0)}{k!} x^k \\[15pt]
       &=& \displaystyle
           \frac{f(0)}{0!} x^0 +
           \frac{f(0)'}{1!} x^1 +
           \frac{f''(0)}{2!} x^2 +
           \frac{f'''(0)}{3!} x^3 + \ldots \\[10pt]
       &=& c + bx + ax^2 + 0 + \ldots \\
  \end{array}
$$ 

E neste caso a igualdade da fórmula $(**)$ é verdade.

\newpage

% «taylor-4»  (to ".taylor-4")
% (c3m201taylor1p 8 "taylor-4")
% (c3m201taylor1    "taylor-4")

{\bf Mini-revisão de séries de Taylor (4)}

\ssk

{\bf Exercício 1} (pra você se convencer de que a fórmula $(**)$ vale
sempre que a função $f$ for um polinômio).

Seja $f(x) = a_4x^4 + a_3x^3 + a_2x^2 + a_1x^1 + a_0x^0$.

\ssk

a) Calcule $\derivs(f)$.

b) Calcule $\derivs_0(f)$. 

c) Expanda o somatório $\sum_{k=0}^∞ \frac{f^{(k)}(0)}{k!} x^k$ e
verifique que neste caso a igualdade $(**)$ é verdade (como no slide
anterior).



\newpage

% «taylor-5»  (to ".taylor-5")
% (c3m201taylor1p 9 "taylor-5")
% (c3m201taylor1    "taylor-5")

{\bf Mini-revisão de séries de Taylor (5)}

\ssk

No caso geral -- em que a $f$ não é polinomial -- a expansão do
somatório na fórmula $(**)$ dá uma soma com infinitos termos
não-zero... e isto às vezes é formalizado desta forma:
%
$$f(x) = \displaystyle \lim_{N→∞} \left(\sum_{k=0}^N \frac{f^{(k)}(0)}{k!} x^k\right)$$

À medida que o $N$ cresce a expressão $\sum_{k=0}^N
\frac{f^{(k)}(0)}{k!} x^k$ -- a \ColorRed{série de Taylor de $f$ em
  $x=0$ truncada até grau $N$} -- vira um polinômio com mais termos, e
cada polinômio novo com mais termos que o anterior é uma aproximação
melhor para a função $f$.

\msk

A série de Taylor truncada até grau $N$ às vezes vai ser chamada de
\ColorRed{aproximação de grau $N$} ou de \ColorRed{polinômio de Taylor
  de grau $N$}.



\newpage

% «taylor-6»  (to ".taylor-6")
% (c3m201taylor1p 9 "taylor-6")
% (c3m201taylor1    "taylor-6")

{\bf Mini-revisão de séries de Taylor (6)}

\ssk

Os detalhes são \ColorRed{bem} complicados -- você vai ver todas as
contas horríveis que demonstram as estimativas de erro numa matéria do
Fábio -- mas deve dar pra entender a idéia geral a partir dos desenhos
e animações das páginas da Wikipedia.

Dê uma olhada em:

\ssk

\url{https://pt.wikipedia.org/wiki/S\%C3\%A9rie_de_Taylor}

\url{https://en.wikipedia.org/wiki/Taylor_series}

\url{https://en.wikipedia.org/wiki/Taylor_series\#Approximation_error_and_convergence}

\url{https://en.wikipedia.org/wiki/Taylor\%27s_theorem}

\ssk

principalmente nas figuras que comparam aproximações de grau 1, 2, 3,
etc. As páginas da Wikipedia em português têm menos figuras que as em
inglês, então eu pus os links pras páginas em inglês também.


% (find-es "tex" "hyperref")
% (find-hyperrefmanualpage 25 "5.8    \\unichar")
% (find-hyperrefmanualtext 25 "5.8    \\unichar")


\newpage

% «taylor-7»  (to ".taylor-7")
% (c3m201taylor1p 9 "taylor-7")
% (c3m201taylor1    "taylor-7")

{\bf Mini-revisão de séries de Taylor (7)}

\ssk

O que vai importar pra gente agora é isto:

$$\begin{array}{rcl}
  f(x) &≈& f(0) + f'(0)x \\
  f(x) &≈& f(0) + f'(0)x + \frac{f''(0)}{2} x^2 \\[5pt]
  \end{array}
$$

A ``aproximação de grau 1'', $f(x) ≈ f(0) + f'(0)x$, dá uma
aproximação bem razoável pro valor de $f(x)$ quando $x$ é pequeno, e a
``aproximação de grau 2'', $f(x) ≈ f(0) + f'(0)x + \frac{f''(0)}{2}
x^2$ dá uma aproximação melhor...

...só que daqui a pouco nós vamos adaptar isto para funções $f$ que
são ``trajetórias'', ou seja, que vão de $\R$ e $\R^2$. E além disso...



\newpage

% «taylor-8»  (to ".taylor-8")
% (c3m201taylor1p 12 "taylor-8")
% (c3m201taylor1     "taylor-8")

{\bf Séries de Taylor em torno de pontos que não são o 0}

...e além disso vamos querer trabalhar com séries de Taylor em torno
de pontos que não são o ponto 0 -- e isso eu nunca mostro em Cálculo
2. As fórmulas são:
%
$$\begin{array}{rcl}
  f(x) &=& \displaystyle \sum_{k=0}^∞ \frac{f^{(k)}(a)}{k!} (x-a)^k \\[10pt]
  f(x) &≈& f(a) + f'(a)(x-a) \\
  f(x) &≈& f(a) + f'(a)(x-a) + \frac{f''(a)}{2} (x-a)^2 \\[5pt]
  \end{array}
$$

Repare que as expressões à direita do `$=$' e dos `$≈$' são
\ColorRed{polinômios em $(x-a)$!} Uma definição nova:
%
$$\derivs_a(f) = (f(a), f'(a), f''(a), f'''(a), \ldots).$$


\newpage

% «taylors-famosas»  (to ".taylors-famosas")

{\bf Algumas séries de Taylor famosas}

\ssk

{\bf Exercício 2.} Calcule os primeiros termos de $\derivs(f)$ e
$\derivs_0(f)$ para as funções abaixo:

a) $f(x) = e^x$

b) $f(x) = e^{2x}$

c) $f(x) = e^{ix}$

d) $f(x) = \cos x$

e) $f(x) = \sen x$

f) $f(x) = i\sen x$

g) $f(x) = \cos x + i\sen x$ \qquad (só o $\derivs_0$)

e compare os itens (c) e (g).

\newpage

{\bf Exercício 3.} Descubra a série de Taylor de $f(x) = e^{2x}$ no
ponto 0. Dica: veja a página da Wikipedia em português sobre Séries de
Taylor...

\ssk

\url{https://pt.wikipedia.org/wiki/S\%C3\%A9rie_de_Taylor}

\ssk

Ela tem alguns exemplos numa seção chamada ``Lista de série de Taylor
de algumas funções comuns ao redor de $a=0$ (Série de Maclaurin)''.



\newpage

{\bf Voltando a Cálculo 3...}

Nas aulas anteriores nós aprendemos a desenhar {\sl retas
  parametrizadas} e {\sl parábolas parametrizadas}, e entendemos o que
são o {\sl vetor velocidade} e o {\sl vetor aceleração} de uma
trajetória...

Agora vamos ver como usar retas parametrizadas como uma aproximação de
grau 1 -- ou ``de primeira ordem'' -- para uma trajetória e parábolas
parametrizadas como aproximações de grau 2 -- ou ``de segunda ordem''.

\msk

Repare que no Bortolossi essas idéias estão espalhadas pelo livro...

O vetor tangente aparece no cap.6, p.197,

o vetor aceleração aparece no cap.6, p.217,

polinômios de Taylor de ordem 2 aparecem no cap.11, p.371

polinômios de Taylor de ordem $k$ aparecem no cap.11, p.376.

\newpage

{\bf Aproximações de 1ª e 2ª ordem para trajetórias}

\ssk

Seja $f:\R→\R^2$ uma trajetória.

Seja $t_0∈\R$.

Definições (temporárias, vão ser melhoradas depois):

\msk

A {\sl aproximação de 1ª ordem para $f$ em $t=t_0$} é a reta
  parametrizada:
%
$$g(t) = f(t_0) + f'(t_0)(t-t_0)$$

A {\sl aproximação de 2ª ordem para $f$ em $t=t_0$} é a reta
parametrizada:
%
$$h(t) = f(t_0) + f'(t_0)(t-t_0) + \frac{f''(t_0)}{2} (t-t_0)^2$$

Vamos fazer alguns execícios pra aprender a desenhar e a visualizar
essas aproximações e depois vamos ver que propriedades elas obedecem.


\newpage

% «exercicio-4»  (to ".exercicio-4")
% (c3m201taylor1p 17 "exercicio-4")
% (c3m201taylor1     "exercicio-4")

{\bf Exercício 4.}

Sejam $f(t)=(t,\cos t)$ e $t_0=0$.

Represente graficamente num gráfico só:

a) O traço de $f(t)$ e os pontos $f(0)$, $f(\frac{\pi}{2})$, $f(\pi)$,

b) $f(t_0) + f'(t_0)$, 

c) $f(t_0) + f''(t_0)$, 

d) O traço de $g(t)$ -- lembre que $g(t) = f(t_0) + f'(t_0)(t-t_0)$ --
e os pontos $g(t_0)$ e $g(t_0+1)$,


\newpage

% «exercicio-5»  (to ".exercicio-5")
% (c3m201taylor1p 18 "exercicio-5")
% (c3m201taylor1     "exercicio-5")

{\bf Exercício 5.}

Sejam $f(t)=(t,\cos t)$ e $t_0=\ColorRed{π}$.

Represente graficamente num gráfico só:

a) O traço de $f(t)$ e os pontos $f(0)$, $f(\frac{\pi}{2})$, $f(\pi)$,

b) $f(t_0) + f'(t_0)$, 

c) $f(t_0) + f''(t_0)$, 

d) O traço de $g(t)$ -- lembre que $g(t) = f(t_0) + f'(t_0)(t-t_0)$ --
e os pontos $g(t_0)$ e $g(t_0+1)$,

e) Lembre que $h(t) = f(t_0) + f'(t_0)(t-t_0) + \frac{f''(t_0)}{2}
(t-t_0)^2$. Represente graficamente os pontos $h(t_0)$, $h(t_0+1)$,
$h(t_0-1)$, $h(t_0+2)$, $h(t_0-2)$.

f) \ColorRed{(Mais difícil)} Use o que você descobriu no item (e) para
representar graficamente o traço de $h(t)$ -- que vai ser uma parábola
parametrizada.

\msk

\ColorRed{Dica: o seu desenho não precisa ficar muito preciso -- use
  $π≈3$.}.







% (find-bortolossi6page (+ -186 197) "6.2 O vetor tangente a uma curva parametrizada")
% (find-bortolossi6page (+ -186 199)   "limite de vetores secantes")
% (find-bortolossi6page (+ -186 215)   "A ciclóide")
% (find-bortolossi6page (+ -186 217)   "O vetor aceleração")
% (find-bortolossi11page (+ -364 371) "11.2. Polinômios de Taylor")
% (find-bortolossi11page (+ -364 376)   "Polinômios de Taylor de ordem k")





\ssk





\end{document}

%  __  __       _        
% |  \/  | __ _| | _____ 
% | |\/| |/ _` | |/ / _ \
% | |  | | (_| |   <  __/
% |_|  |_|\__,_|_|\_\___|
%                        
% <make>

 (eepitch-shell)
 (eepitch-kill)
 (eepitch-shell)
# (find-LATEXfile "2019planar-has-1.mk")
make -f 2019.mk STEM=2020-1-C3-taylor-1 veryclean
make -f 2019.mk STEM=2020-1-C3-taylor-1 pdf

% Local Variables:
% coding: utf-8-unix
% ee-tla: "c3m201taylor1"
% End:
