% (find-LATEX "2020-1-C2-TFC2-2.tex")
% (defun c () (interactive) (find-LATEXsh "lualatex -record 2020-1-C2-TFC2-2.tex" :end))
% (defun D () (interactive) (find-pdf-page      "~/LATEX/2020-1-C2-TFC2-2.pdf"))
% (defun d () (interactive) (find-pdftools-page "~/LATEX/2020-1-C2-TFC2-2.pdf"))
% (defun e () (interactive) (find-LATEX "2020-1-C2-TFC2-2.tex"))
% (defun u () (interactive) (find-latex-upload-links "2020-1-C2-TFC2-2"))
% (defun v () (interactive) (find-2a '(e) '(d)) (g))
% (find-pdf-page   "~/LATEX/2020-1-C2-TFC2-2.pdf")
% (find-sh0 "cp -v  ~/LATEX/2020-1-C2-TFC2-2.pdf /tmp/")
% (find-sh0 "cp -v  ~/LATEX/2020-1-C2-TFC2-2.pdf /tmp/pen/")
%   file:///home/edrx/LATEX/2020-1-C2-TFC2-2.pdf
%               file:///tmp/2020-1-C2-TFC2-2.pdf
%           file:///tmp/pen/2020-1-C2-TFC2-2.pdf
% http://angg.twu.net/LATEX/2020-1-C2-TFC2-2.pdf
% (find-LATEX "2019.mk")
% (find-C2-aula-links "2020-1-C2-TFC2-2" "11" "tfc22")

% «.title»			(to "title")
% «.intro»			(to "intro")
% «.integral-indefinida»	(to "integral-indefinida")
% «.exercicio-1»		(to "exercicio-1")
% «.exercicio-2»		(to "exercicio-2")
% «.subst-defs»			(to "subst-defs")
% «.TFC2-TFC2I-S3-S3I»		(to "TFC2-TFC2I-S3-S3I")
% «.S1»				(to "S1")
% «.dica-importantissima»	(to "dica-importantissima")
% «.exercicio-3»		(to "exercicio-3")
% «.exercicio-4»		(to "exercicio-4")
% «.exercicio-5»		(to "exercicio-5")
%
% «.exerc-jose-victor»		(to "exerc-jose-victor")

\documentclass[oneside,12pt]{article}
\usepackage[colorlinks,citecolor=DarkRed,urlcolor=DarkRed]{hyperref} % (find-es "tex" "hyperref")
\usepackage{amsmath}
\usepackage{amsfonts}
\usepackage{amssymb}
\usepackage{pict2e}
\usepackage[x11names,svgnames]{xcolor} % (find-es "tex" "xcolor")
%\usepackage{colorweb}                 % (find-es "tex" "colorweb")
%\usepackage{tikz}
%
% (find-dn6 "preamble6.lua" "preamble0")
%\usepackage{proof}   % For derivation trees ("%:" lines)
%\input diagxy        % For 2D diagrams ("%D" lines)
%\xyoption{curve}     % For the ".curve=" feature in 2D diagrams
%
\usepackage{edrx15}               % (find-LATEX "edrx15.sty")
\input edrxaccents.tex            % (find-LATEX "edrxaccents.tex")
\input edrxchars.tex              % (find-LATEX "edrxchars.tex")
\input edrxheadfoot.tex           % (find-LATEX "edrxheadfoot.tex")
\input edrxgac2.tex               % (find-LATEX "edrxgac2.tex")
%
%\usepackage[backend=biber,
%   style=alphabetic]{biblatex}            % (find-es "tex" "biber")
%\addbibresource{catsem-slides.bib}        % (find-LATEX "catsem-slides.bib")
%
% (find-es "tex" "geometry")
\usepackage[a6paper, landscape,
            top=1.5cm, bottom=.25cm, left=1cm, right=1cm, includefoot
           ]{geometry}
%
\begin{document}

\catcode`\^^J=10
\directlua{dofile "dednat6load.lua"}  % (find-LATEX "dednat6load.lua")

% %L dofile "edrxtikz.lua"  -- (find-LATEX "edrxtikz.lua")
% %L dofile "edrxpict.lua"  -- (find-LATEX "edrxpict.lua")
% \pu

% «defs»  (to ".defs")
% (find-LATEX "edrx15.sty" "colors-2019")
\long\def\ColorRed   #1{{\color{Red1}#1}}
\long\def\ColorViolet#1{{\color{MagentaVioletLight}#1}}
\long\def\ColorViolet#1{{\color{Violet!50!black}#1}}
\long\def\ColorGreen #1{{\color{SpringDarkHard}#1}}
\long\def\ColorGreen #1{{\color{SpringGreenDark}#1}}
\long\def\ColorGreen #1{{\color{SpringGreen4}#1}}
\long\def\ColorGray  #1{{\color{GrayLight}#1}}
\long\def\ColorGray  #1{{\color{black!30!white}#1}}
\long\def\ColorBrown #1{{\color{Brown}#1}}
\long\def\ColorBrown #1{{\color{brown}#1}}

\long\def\ColorShort #1{{\color{SpringGreen4}#1}}
\long\def\ColorLong  #1{{\color{Red1}#1}}

\def\frown{\ensuremath{{=}{(}}}
\def\True {\mathbf{V}}
\def\False{\mathbf{F}}
\def\Subst#1{\bmat{#1}}

\def\drafturl{http://angg.twu.net/LATEX/2020-1-C2.pdf}
\def\drafturl{http://angg.twu.net/2020.1-C2.html}
\def\draftfooter{\tiny \href{\drafturl}{\jobname{}} \ColorBrown{\shorttoday{} \hours}}

% (find-angg ".emacs" "c2q192")


%  _____ _ _   _                               
% |_   _(_) |_| | ___   _ __   __ _  __ _  ___ 
%   | | | | __| |/ _ \ | '_ \ / _` |/ _` |/ _ \
%   | | | | |_| |  __/ | |_) | (_| | (_| |  __/
%   |_| |_|\__|_|\___| | .__/ \__,_|\__, |\___|
%                      |_|          |___/      
%
% «title»  (to ".title")
% (c2m201tfc22p 1 "title")
% (c2m201tfc22a   "title")

\thispagestyle{empty}

\begin{center}

\vspace*{1.2cm}

{\bf \Large Cálculo 2 - 2020.1}

\bsk

Aula 11: Integração por TFC2 e chutar e testar

\bsk

Eduardo Ochs - RCN/PURO/UFF

\url{http://angg.twu.net/2020.1-C2.html}

\end{center}

\newpage

% «integral-indefinida»  (to ".integral-indefinida")
% (c2m201tfc22p 2 "integral-indefinida")
% (c2m201tfc22a   "integral-indefinida")

{\bf A integral indefinida}

\ssk

Na aula passada nós vimos que pra dá pra calcular integrais de uma
função $f(x)$ usando uma antiderivada contínua de $f(x)$. Se $F(x)$ é
uma antiderivada contínua de $f(x)$ então:
%
$$\Intx{a}{b}{f(x)} \;\; = \;\; F(b) - F(a)$$

Nós vamos interpretar a notação da integral indefinida, que é
$\intx{\ldots}$ sem os limites de integração, como uma espécie de
inversa da operação $\frac{d}{dx}$: pra nós as duas expressões abaixo
vão ser exatamente equivalentes:
%
$$\begin{array}{ccr}
  \displaystyle \int {f(x)} \, dx &=& F(x) \\
                f(x)  &=& \frac{d}{dx} F(x) \\
  \end{array}
$$

\newpage

Depois eu vou explicar porque é que a maioria dos livros prefere
escrever sempre o `$+\;C$' em
%
$$\int {f(x)} \, dx \;\; = \;\; F(x) \ColorRed{\;+\;C}$$
%
e porque é que eu prefiro não usá-lo.

\msk

Outra coisa: a partir de agora \ColorRed{quase} todas as nossas
funções vão ser contínuas e deriváveis, e por isso eu \ColorRed{em
  geral} vou falar só de ``antiderivadas'' ao invés de ``antiderivadas
contínuas''.


\newpage

% «exercicio-1»  (to ".exercicio-1")
% (c2m201tfc22p 4 "exercicio-1")
% (c2m201tfc22a   "exercicio-1")

{\bf Exercício 1.}

\ssk

Encontre as seguintes antiderivadas por chutar \ColorRed{e testar}:

\ssk

a) $\intx {\cos x}$

b) $\intx {\sen x}$

c) $\intx {\cos 3x}$

d) $\intx {\cos (3x+4)}$

e) $\intx {2 \cos (3x+4)}$

f) $\intx {a \cos (bx+c)}$

g) $\intx {\sen x + 2 \cos (3x+4)}$

h) $\intx {e^x}$

i) $\intx {e^{x+4}}$

j) $\intx {e^{3x+4}}$

k) $\intx {2e^{3x+4}}$

l) $\intx {ae^{bx+c}}$



\newpage


% «exercicio-2»  (to ".exercicio-2")
% (c2m201tfc22p 5 "exercicio-2")
% (c2m201tfc22a   "exercicio-2")

{\bf Exercício 2.}

\ssk

Calcule:

\bsk

$$\begin{array}{rl}
  a) & \displaystyle \Intx{10}{20}{2\cos(3x+4)} \\[25pt]
  b) & \displaystyle \Intx{d}{e}{a\cos(bx+c)} \\
  \end{array}
$$


\newpage


{\bf Integração por substituição}

\ssk

\def\und#1#2{#1}

A fórmula mais útil pra encontrar antiderivadas difíceis é exatamente
uma das mais difíceis de entender... ela é fácil de decorar na versão
com integrais indefinidas, que é:
%
$$\displaystyle \intx{
    f(\und{g(x)}{u}) \und{g'(x)}{\frac{du}{dx}}
  } = \intu {f(u)}
$$

\def\und#1#2{\underbrace{\textstyle #2}_{#1}}

Se definirmos que à direita do `$=$' o $u$ é uma variável mas à esquerda ele é uma abreviação para $g(x)$ então podemos reescrever a fórmula acima como:
%
$$\displaystyle \intx{
    f(u) \frac{du}{dx}
  } = \intu {f(u)}
$$

\newpage


{\bf Integração por substituição (2)}

\ssk

...só que a fórmula da página anterior tem várias gambiarras
brabíssimas que nós só vamos conseguir formalizar daqui a bastante
tempo, então nós vamos começar entendendo a versão dela que usa
integrais definidas, que é:
%
$$\displaystyle \Intx{a}{b}{f(g(x))g'(x)} = \Intu{g(a)}{g(b)}{f(u)}
$$

\newpage

% (c2mo)
% (c2mop)

% «subst-defs»  (to ".subst-defs")
% (c2m192p 2 "subst-defs-formulas")
% (c2m192    "subst-defs-formulas")

\def\pfo#1{\mathsf{[#1]}}
\def\D{\displaystyle}

% Difference with mathstrut
\def\Difms #1#2#3{\left. \mathstrut #3 \right|_{s=#1}^{s=#2}}
\def\Difmu #1#2#3{\left. \mathstrut #3 \right|_{u=#1}^{u=#2}}
\def\Difmx #1#2#3{\left. \mathstrut #3 \right|_{x=#1}^{x=#2}}
\def\Difmth#1#2#3{\left. \mathstrut #3 \right|_{θ=#1}^{θ=#2}}

\def\iequationbox#1#2{
    \left(
    \begin{array}{rcl}
    \D{ #1 } &=& \D{ #2 } \\
    \end{array}
    \right)
  }
\def\isubstbox#1#2#3#4#5{{
    \def\veq{\rotatebox{90}{$=$}}
    \def\ph{\phantom}
    \left(
    \begin{array}{rcl}
    \D{ #1 } &=& \D{ #2 } \\
    {\veq#3} \\
    \D{ #4 } &=& \D{ #5 } \\
    \end{array}
    \right)
  }}
\def\isubstboxT#1#2#3#4#5#6{{
    \def\veq{\rotatebox{90}{$=$}}
    \def\ph{\phantom}
    \left(
    \begin{array}{rcl}
    \multicolumn{3}{l}{\text{#6}} \\%[5pt]
    \D{ #1 } &=& \D{ #2 } \\
    {\veq#3} \\
    \D{ #4 } &=& \D{ #5 } \\
    \end{array}
    \right)
  }}

% Definição das fórmulas para integração por substituição.
% Algumas são pmatrizes 3x3 usando isubstbox.

\def\TFCtwo{
  \iequationbox {\Intx{a}{b}{F'(x)}}
                {\Difmx{a}{b}{F(x)}}
}
\def\TFCtwoI{
  \iequationbox {\intx{F'(x)}}
                {F(x)}
}

\def\Sone{
  \isubstbox
    {\Difmx{a}{b}{f(g(x))}}  {\Intx{a}{b}{f'(g(x))g'(x)}}
    {\ph{mmm}}
    {\Difmu{g(a)}{g(b)}{f(u)}} {\Intu{g(a)}{g(b)}{f'(u)}}
}
\def\SoneI{
  \isubstbox
    {f(g(x))} {\intx{f'(g(x))g'(x)}}
    {\ph{m}}
    {f(u)}    {\intu{f'(u)}}
}

\def\Stwo{
  \isubstboxT
    {\Difmx{a}{b}{F(g(x))}}   {\Intx{a}{b}{f(g(x))g'(x)}}
    {\ph{mmm}}
    {\Difmu{g(a)}{g(b)}{F(u)}}  {\Intu{g(a)}{g(b)}{f(u)}}
    {Se $F'(x)=f(x)$ então:}
}
\def\StwoI{
  \isubstboxT
    {F(g(x))}  {\intx{f(g(x))g'(x)}}
    {\ph{m}}
    {F(u)}     {\intu{f(u)}}
    {Se $F'(x)=f(x)$ então:}
}

\def\Sthree{
  \iequationbox {\Intx{a}{b}{f(g(x))g'(x)}}
                {\Intu{g(a)}{g(b)}{f(u)}}
}
\def\SthreeI{
  \iequationbox {\intx{f(g(x))g'(x)}}
                {\intu{f(u)}
                 \qquad [u=g(x)]
                }
  % [u=g(x)]
}

% \newpage

% «TFC2-TFC2I-S3-S3I»  (to ".TFC2-TFC2I-S3-S3I")
% (c2m201tfc22p 8 "TFC2-TFC2I-S3-S3I")
% (c2m201tfc22a   "TFC2-TFC2I-S3-S3I")

{\bf Integração por substituição (3)}

\ssk

Vamos dar nomes para algumas estas fórmulas:
%
$$\begin{array}[t]{rcl}
  %\text{Fórmulas}:        \\[5pt]
  \pfo{TFC2} &=& \TFCtwo  \\
 \pfo{TFC2I} &=& \TFCtwoI \\
   %\pfo{S1} &=& \Sone    \\
   %\pfo{S2} &=& \Stwo    \\
    \pfo{S3} &=& \Sthree  \\
  %\pfo{S1I} &=& \SoneI   \\
  %\pfo{S2I} &=& \StwoI   \\
   \pfo{S3I} &=& \SthreeI
  \end{array}
$$

A notação ``$\Difmx{a}{b}{F(x)}$'' se pronuncia ``a diferença do valor
de $F(x)$ entre $x=a$ e $x=b$'' e é definida por: $\Difmx{a}{b}{F(x)}
= F(b) - F(a)$.

\newpage

% «S1»  (to ".S1")
% (c2m201tfc22p 9 "S1")
% (c2m201tfc22a   "S1")

{\bf Integração por substituição (4)}

\ssk

As fórmulas com ``I'' no nome usam integrais indefinidas e são mais
abstratas do que as sem ``I''. A fórmula $\pfo{S3}$ é consequência
desta aqui, que é uma demonstração composta de três igualdades fáceis
de provar:

$$\begin{array}[t]{rcl}
   \pfo{S1} &=& \Sone    \\
   %\pfo{S2} &=& \Stwo    \\
  \end{array}
$$

\newpage

% «dica-importantissima»  (to ".dica-importantissima")
% (c2m201tfc22p 10 "dica-importantissima")
% (c2m201tfc22a    "dica-importantissima")

{\bf Dica importantíssima:}

Se estivermos fazendo as coisas bem passo a passo,

isto está certo:
%
$$(f(2+3)) \Subst{f(x):=\sen x} = \sen(2+3)$$

e isto está \ColorRed{errado},
%
$$(f(2+3)) \Subst{f(x):=\sen x} = \sen(5)$$

porque aqui além da substituição indicada no ``$\Subst{f(x):=\sen
  x}$''

a gente substituiu o $2+3$ por $5$.



\newpage

% «exercicio-3»  (to ".exercicio-3")
% (c2m201tfc22p 11 "exercicio-3")
% (c2m201tfc22a    "exercicio-3")

{\bf Exercício 3.}

Encontre o resultado das substituições:

\msk

a) $\pfo{TFC2}\Subst{F(x) := f(g(x))}$

b) $\pfo{TFC2}\Subst{x:=u}$

c) $\pfo{TFC2}\Subst{x:=u}\subst{a:=g(a) \\ b:=g(b) \\ F(u):=f(u)\\}$

E verifique que os itens (a) e (c) provam

as duas igualdades horizontais da $\pfo{S1}$.

\msk

Além disso verifique que:
%
$$\Difmx{a}{b}{f(g(x))}
  \;\;=\;\; f(g(b)) - f(g(a))
  \;\;=\;\;
  \Difmu{g(a)}{g(b)}{f(u)}
$$

\newpage

% «exercicio-4»  (to ".exercicio-4")
% (c2m201tfc22p 12 "exercicio-4")
% (c2m201tfc22a    "exercicio-4")

{\bf Exercício 4.}

\ssk

Encontre o resultado das substituições:

\msk

a) $\pfo{S1} \Subst{a:=10 \\ b:=20}$

b) $\pfo{S1} \Subst{f(u) := \sen(u) \\ g(x) := 3x+4 \\ a:=10 \\ b:=20}$

\bsk

Obs: em cada uma delas o resultado da substituição é uma série de três
igualdades arrumada num formato bem parecido com o da $\pfo{S1}$
original.




\newpage

No exercício 1 você encontrou um monte de antiderivadas, e

algumas delas servem como {\sl fórmulas de integração.}

Por exemplo:


$$\begin{array}{ccr}
                      a \cos(bx+c) &=& \frac{d}{dx} (\frac{a}{b}\sen(bx+c)) \\
  \displaystyle \int {a \cos(bx+c)} \, dx &=&        \frac{a}{b}\sen(bx+c) \\
  \end{array}
$$

Em geral pra resolver integrais a gente vai ter que encontrar alguma
substituição que transforme uma fórmula de integração que conhecemos
em numa fórmula que resolve a integral que queremos resolver... por
exemplo, sabemos que $\frac{d}{dx}f(g(x)) = f'(g(x))g'(x)$ sempre
vale, e portanto:
%
$$ \intx {f'(g(x))g'(x)} \;\; = \;\; f(g(x)) $$

% exemplo, sabemos que $\frac{d}{dx}(a \, f(g(x))) = a \, f'(g(x))g'(x)$, e
% portanto
% %
% $$ \intx {a f'(g(x))g'(x)} \;\; = \;\; a \, f(g(x)) $$

\newpage

% «exercicio-5»  (to ".exercicio-5")
% (c2m201tfc22p 14 "exercicio-5")
% (c2m201tfc22a    "exercicio-5")

{\bf Exercício 5.}

\ssk

\def\rq{\ColorRed{?}}

Descubra como completar a substituição abaixo pra

que os dois `$=$'s sejam verdade. O primeiro `$=$' é uma

substituição bem passo a passo, como na

``Dica Importantíssima'' do slide 10.

$$\begin{array}{rcl}
  \left( f'(g(x))g'(x) \right)
  \Subst{f(u) := \rq \\ f'(u) := \rq \\ g(x) := 3x+4 \\ g'(x) := \rq}
    &=& \rq \\
    &=& 2 \cos(3x+4) \\
  \end{array}
$$

\newpage

{\bf Dica:} as linhas ``$f'(u) := \rq$'' e ``$g'(x) := \rq$'' na
substituição

da página anterior são um truque pra ajudar a gente a se

enrolar menos... a gente usou isso na primeira aula, no

slide 12. Dê uma olhada! Link:

\ssk

\url{http://angg.twu.net/LATEX/2020-1-C2-intro.pdf}

% (c2m201introp 12 "acrescentamos")
% (c2m201intro     "acrescentamos")





% (find-apexcalculuspage (+ 924 43) "Integration")




% Exercícios de somatório pro José Victor:
% «exerc-jose-victor»  (to ".exerc-jose-victor")
% 
% Digamos que $g(x)=6-x$.
% 
% \msk
% 
% a) Calcule BEM passo a passo:
% 
% $$\sum_{i=2}^{4} \left( \frac{g(4)+g(2)}{2} · (g(4)-g(2)) \right)$$
% 
% \bsk
% 
% b) Calcule BEM passo a passo:
% 
% $$\sum_{i=2}^{4} \left( \frac{g(a_i)+g(b_i)}{2} · (b_i - a_i) \right)$$



%\printbibliography

\end{document}

%  __  __       _        
% |  \/  | __ _| | _____ 
% | |\/| |/ _` | |/ / _ \
% | |  | | (_| |   <  __/
% |_|  |_|\__,_|_|\_\___|
%                        
% <make>

 (eepitch-shell)
 (eepitch-kill)
 (eepitch-shell)
# (find-LATEXfile "2019planar-has-1.mk")
make -f 2019.mk STEM=2020-1-C2-TFC2-2 veryclean
make -f 2019.mk STEM=2020-1-C2-TFC2-2 pdf

% Local Variables:
% coding: utf-8-unix
% ee-tla: "c2m201tfc22"
% End:
