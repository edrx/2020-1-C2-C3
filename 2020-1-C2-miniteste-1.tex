% (find-LATEX "2020-1-C2-miniteste-1.tex")
% (defun c () (interactive) (find-LATEXsh "lualatex -record 2020-1-C2-miniteste-1.tex" :end))
% (defun D () (interactive) (find-pdf-page      "~/LATEX/2020-1-C2-miniteste-1.pdf"))
% (defun d () (interactive) (find-pdftools-page "~/LATEX/2020-1-C2-miniteste-1.pdf"))
% (defun e () (interactive) (find-LATEX "2020-1-C2-miniteste-1.tex"))
% (defun u () (interactive) (find-latex-upload-links "2020-1-C2-miniteste-1"))
% (defun v () (interactive) (find-2a '(e) '(d)) (g))
% (find-pdf-page   "~/LATEX/2020-1-C2-miniteste-1.pdf")
% (find-sh0 "cp -v  ~/LATEX/2020-1-C2-miniteste-1.pdf /tmp/")
% (find-sh0 "cp -v  ~/LATEX/2020-1-C2-miniteste-1.pdf /tmp/pen/")
%   file:///home/edrx/LATEX/2020-1-C2-miniteste-1.pdf
%               file:///tmp/2020-1-C2-miniteste-1.pdf
%           file:///tmp/pen/2020-1-C2-miniteste-1.pdf
% http://angg.twu.net/LATEX/2020-1-C2-miniteste-1.pdf
% (find-LATEX "2019.mk")

% «.exercicio-1»	(to "exercicio-1")
% «.exercicio-2»	(to "exercicio-2")
% «.exercicio-3»	(to "exercicio-3")
% «.exercicio-4»	(to "exercicio-4")
% «.miniteste-regras»	(to "miniteste-regras")
% «.mini-teste»		(to "mini-teste")

\documentclass[oneside,12pt]{article}
\usepackage[colorlinks,citecolor=DarkRed,urlcolor=DarkRed]{hyperref} % (find-es "tex" "hyperref")
\usepackage{amsmath}
\usepackage{amsfonts}
\usepackage{amssymb}
\usepackage{pict2e}
\usepackage[x11names,svgnames]{xcolor} % (find-es "tex" "xcolor")
\usepackage{colorweb}                 % (find-es "tex" "colorweb")
%\usepackage{tikz}
%
% (find-dn6 "preamble6.lua" "preamble0")
%\usepackage{proof}   % For derivation trees ("%:" lines)
%\input diagxy        % For 2D diagrams ("%D" lines)
%\xyoption{curve}     % For the ".curve=" feature in 2D diagrams
%
\usepackage{edrx15}               % (find-LATEX "edrx15.sty")
\input edrxaccents.tex            % (find-LATEX "edrxaccents.tex")
\input edrxchars.tex              % (find-LATEX "edrxchars.tex")
\input edrxheadfoot.tex           % (find-LATEX "edrxheadfoot.tex")
\input edrxgac2.tex               % (find-LATEX "edrxgac2.tex")
%
%\usepackage[backend=biber,
%   style=alphabetic]{biblatex}            % (find-es "tex" "biber")
%\addbibresource{catsem-slides.bib}        % (find-LATEX "catsem-slides.bib")
%
% (find-es "tex" "geometry")
\usepackage[a6paper, landscape,
            top=1.5cm, bottom=.25cm, left=1cm, right=1cm, includefoot
           ]{geometry}
%
\begin{document}

\catcode`\^^J=10
\directlua{dofile "dednat6load.lua"}  % (find-LATEX "dednat6load.lua")

%L dofile "edrxtikz.lua"  -- (find-LATEX "edrxtikz.lua")
%L dofile "edrxpict.lua"  -- (find-LATEX "edrxpict.lua")
\pu

% «defs»  (to ".defs")
% (find-LATEX "edrx15.sty" "colors-2019")
\long\def\ColorRed   #1{{\color{Red1}#1}}
\long\def\ColorViolet#1{{\color{MagentaVioletLight}#1}}
\long\def\ColorViolet#1{{\color{Violet!50!black}#1}}
\long\def\ColorGreen #1{{\color{SpringDarkHard}#1}}
\long\def\ColorGreen #1{{\color{SpringGreenDark}#1}}
\long\def\ColorGreen #1{{\color{SpringGreen4}#1}}
\long\def\ColorGray  #1{{\color{GrayLight}#1}}
\long\def\ColorGray  #1{{\color{black!30!white}#1}}
\long\def\ColorBrown #1{{\color{Brown}#1}}
\long\def\ColorBrown #1{{\color{brown}#1}}

\long\def\ColorShort #1{{\color{SpringGreen4}#1}}
\long\def\ColorLong  #1{{\color{Red1}#1}}

\def\frown{\ensuremath{{=}{(}}}
\def\True {\mathbf{V}}
\def\False{\mathbf{F}}

\def\drafturl{http://angg.twu.net/LATEX/2020-1-C2.pdf}
\def\drafturl{http://angg.twu.net/2020.1-C2.html}
\def\draftfooter{\tiny \href{\drafturl}{\jobname{}} \ColorBrown{\shorttoday{} \hours}}

% (find-angg ".emacs" "c2q192")


%  _____ _ _   _                               
% |_   _(_) |_| | ___   _ __   __ _  __ _  ___ 
%   | | | | __| |/ _ \ | '_ \ / _` |/ _` |/ _ \
%   | | | | |_| |  __/ | |_) | (_| | (_| |  __/
%   |_| |_|\__|_|\___| | .__/ \__,_|\__, |\___|
%                      |_|          |___/      
%
% «title»  (to ".title")

\thispagestyle{empty}

\begin{center}

\vspace*{1.2cm}

{\bf \Large Cálculo 2 - 2020.1}

\bsk

Exercícios de preparação para o Miniteste 1

\bsk

Eduardo Ochs - RCN/PURO/UFF

\url{http://angg.twu.net/2020.1-C2.html}

\end{center}

\newpage

% «exercicio-1»  (to ".exercicio-1")
% (c2m201mt1p 2 "exercicio-1")
% (c2m201mt1    "exercicio-1")
% (c2m201tfcsp 2 "formadas-por-segmentos")
% (c2m201tfcs    "formadas-por-segmentos")

{\bf Exercício 1.}

\ssk

Seja $A$ este polígono:

$$
 \unitlength=15pt
 %
 \vcenter{\hbox{%
 \beginpicture(0,0)(5,3)
   \pictgrid%
   \pictpiecewise{(1,2)c--(3,2)c--(4,1)c--(2,1)c--(1,2)c}%
   \pictaxes%
 \end{picture}%
 }}
$$

Seja $h$ a função que dá a diferença entre a borda superior e a borda
inferior de $A$ para cada valor de $x$ em $[1,4]$, e que é 0 para
$x<1$ e para $4<x$.

\msk

a) Faça o gráfico da função $h$.

b) Dê uma definição por casos para a função $h$.

c) Seja $H(b) = \Intx{0}{b}{h(x)}$. Desenhe o gráfico da $H$.

d) Dê uma definição por casos para $H$.

%\printbibliography

\newpage

% «exercicio-2»  (to ".exercicio-2")
% (c2m201mt1p 3 "exercicio-2")
% (c2m201mt1    "exercicio-2")

{\bf Exercício 2.}

\ssk

Agora seja $A$ este polígono:

$$
 \unitlength=15pt
 %
 \vcenter{\hbox{%
 \beginpicture(0,0)(5,4)
   \pictgrid%
   \pictpiecewise{(1,3)c--(1,1)c--(4,1)c--(4,2)c--(3,2)c--(3,3)c--(1,3)c}%
   \pictaxes%
 \end{picture}%
 }}
$$

Seja $h$ a função que dá a diferença entre a borda superior e a borda
inferior de $A$ para cada valor de $x$ em $[1,4]$, e que é 0 para
$x\not∈[1,4]$.

\msk

a) Faça o gráfico da função $h$.

b) Dê uma definição por casos para a função $h$.

c) Seja $H(b) = \Intx{0}{b}{h(x)}$. Desenhe o gráfico da $H$.

d) Dê uma definição por casos para $H$.


\newpage

% «exercicio-3»  (to ".exercicio-3")
% (c2m201mt1p 4 "exercicio-3")
% (c2m201mt1    "exercicio-3")

{\bf Exercício 3.}

\ssk

Agora seja $A$ este polígono:

$$
 \unitlength=15pt
 %
 \vcenter{\hbox{%
 \beginpicture(0,0)(5,4)
   \pictgrid%
   \pictpiecewise{(1,3)c--(1,1)c--(4,1)c--(4,2)c--(3,2)c--(3,3)c--(1,3)c}%
   \pictaxes%
 \end{picture}%
 }}
$$

Seja $h$ a função que dá a diferença entre a borda superior e a borda
inferior de $A$ para cada valor de $x$ em $[1,4]$, e que é 0 para
$x\not∈[1,4]$.

\msk

a) Faça o gráfico da função $h$.

b) Dê uma definição por casos para a função $h$.

c) Seja $H(b) = \Intx{\ColorRed{2}}{b}{h(x)}$. Desenhe o gráfico da $H$.

d) Dê uma definição por casos para $H$.


\newpage

% «exercicio-4»  (to ".exercicio-4")
% (c2m201mt1p 5 "exercicio-4")
% (c2m201mt1    "exercicio-4")

{\bf Exercício 4.}

\ssk

Agora seja $A$ este polígono:

$$
 \unitlength=15pt
 %
 \vcenter{\hbox{%
 \beginpicture(0,0)(6,4)
   \pictgrid%
   \pictpiecewise{(1,3)c--(1,1)c--(5,1)c--(5,2)c--(4,2)c--(3,3)c--(1,3)c}%
   \pictaxes%
 \end{picture}%
 }}
$$

Seja $h$ a função que dá a diferença entre a borda superior e a borda
inferior de $A$ para cada valor de $x$ em $[1,5]$, e que é 0 para
$x\not∈[1,5]$.

\msk

a) Faça o gráfico da função $h$.

b) Dê uma definição por casos para a função $h$.

c) Seja $H(b) = \Intx{2}{b}{h(x)}$. Desenhe o gráfico da $H$.

d) Dê uma definição por casos para $H$.



\newpage


\thispagestyle{empty}

\begin{center}

\vspace*{1.2cm}

{\bf \Large Miniteste 1}


\end{center}

\newpage

% «miniteste-regras»  (to ".miniteste-regras")
% (c2m201mt1p 7 "miniteste-regras")
% (c2m201mt1    "miniteste-regras")

{\bf Regras:}

As questões do mini-teste serão disponibilizadas às 14:00 da
quinta-feira 12/nov/2020 e você deverá entregar as respostas
\ColorRed{escritas à mão} até as 14:00 da sexta-feira 12/nov/2020 na
plataforma Classroom. Se o Classroom der algum problema mande também
para este endereço de e-mail:

\ssk

\ColorRed{eduardoochs@gmail.com}

\ssk

Mini-testes entregues após este horário não serão considerados.

Durante as 24 horas do mini-teste nem o professor nem o monitor
responderão perguntas sobre os assuntos do mini-teste mas você pode
discutir com os seus colegas --- inclusive no grupo da turma, mas não
durante o horário da aula.

Este mini-teste vale 0.5 pontos extras na P1.

\msk

(Obs: alguns alunos entregaram depois porque faltou luz)

\newpage

{\bf Regras (cont.):}

\ssk

Os alunos que cumprirem uma série de condições (ainda não divulguei a
lista delas...) poderão compensar as suas questões erradas na P2
fazendo vídeos explicando passo a passo como resolvê-las na semana
seguinte à prova. \ColorRed{Uma das condições é ter feito todos os
  mini-testes, então não deixe de fazer e entregar este mini-teste!}


\newpage

% «mini-teste»  (to ".mini-teste")
% (c2m201mt1p 9 "mini-teste")
% (c2m201mt1    "mini-teste")

{\bf Mini-teste}

\ssk

Seja $A$ este polígono:

$$
 \unitlength=15pt
 %
 \vcenter{\hbox{%
 \beginpicture(0,0)(6,4)
   \pictgrid%
   \pictpiecewise{(1,3)c--(1,1)c--(2,1)c--(2,2)c--(3,1)c--(4,1)c--(4,2)c--(3,2)c--(3,3)c--(1,3)c}%
   \pictaxes%
 \end{picture}%
 }}
$$

Seja $h$ a função que dá a diferença entre a borda superior e a borda
inferior de $A$ para cada valor de $x$ em $[1,4]$, e que é 0 para
$x\not∈[1,4]$.

\msk

a) Faça o gráfico da função $h$.

b) Dê uma definição por casos para a função $h$.

c) Seja $H(b) = \Intx{2}{b}{h(x)}$. Desenhe o gráfico da $H$.

d) Dê uma definição por casos para $H$.


\newpage

{\bf Mini-gabarito}

(Muito incompleto, só pra me ajudar na correção)

\ssk

a e b) $h(x)=
    \unitlength=10pt
    %
    \vcenter{\hbox{%
    \beginpicture(0,0)(6,4)
    \pictgrid%
    \pictpiecewise{(0,0)--(1,0)o}%
    \pictpiecewise{(1,2)c--(2,2)o}%
    \pictpiecewise{(2,1)o--(3,2)o}%
    \pictpiecewise{(3,1)o--(4,1)c}%
    \pictpiecewise{(4,0)o--(6,0)}%
    \pictaxes%
    \end{picture}%
    }}
    =
    \scalebox{1.0}{$
    \begin{cases}
    0   & \text{quando $x<1$}, \\
    2   & \text{quando $1≤x<2$}, \\
    x-1 & \text{quando $2<x<3$}, \\
    1   & \text{quando $3<x≤4$}, \\
    0   & \text{quando $4<x$}. \\
    \end{cases}
    $}
   $

c e d) $H(x)=
    \unitlength=10pt
    %
    \vcenter{\hbox{%
    \beginpicture(0,-3)(6,4)
    \pictgrid%
    \pictpiecewise{(0,-2)--(1,-2)c--(2,0)c}%
    \pictpiecewise{(3,1.5)c--(4,2.5)c--(6,2.5)}%
    \pictpiecewise{(2,0)--%
                   (2.1,0.105)--%
                   (2.2,0.22)--%
                   (2.3,0.345)--%
                   (2.4,0.48)--%
                   (2.4,0.48)--%
                   (2.5,0.625)--%
                   (2.6,0.78)--%
                   (2.6,0.78)--%
                   (2.7,0.945)--%
                   (2.8,1.12)--%
                   (2.8,1.12)--%
                   (2.9,1.305)--%
                   (3,1.5)}
    \pictaxes%
    \end{picture}%
    }}
    =
    \begin{cases}
    -2     & \text{quando $x≤1$}, \\
    2x-4   & \text{quando $1≤x<2$}, \\
    \frac{x^2}{2}-x & \text{quando  $2≤x<3$}, \\
    x-1.5  & \text{quando $3<x≤4$}, \\
    2.5    & \text{quando $4<x$}. \\
    \end{cases}
   $






\end{document}

%  __  __       _        
% |  \/  | __ _| | _____ 
% | |\/| |/ _` | |/ / _ \
% | |  | | (_| |   <  __/
% |_|  |_|\__,_|_|\_\___|
%                        
% <make>

 (eepitch-shell)
 (eepitch-kill)
 (eepitch-shell)
# (find-LATEXfile "2019planar-has-1.mk")
make -f 2019.mk STEM=2020-1-C2-miniteste-1 veryclean
make -f 2019.mk STEM=2020-1-C2-miniteste-1 pdf

% Local Variables:
% coding: utf-8-unix
% ee-tla: "c2m201mt1"
% End:
