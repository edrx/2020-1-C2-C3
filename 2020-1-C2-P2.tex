% (find-LATEX "2020-1-C2-P2.tex")
% (defun c () (interactive) (find-LATEXsh "lualatex -record 2020-1-C2-P2.tex" :end))
% (defun C () (interactive) (find-LATEXsh "lualatex 2020-1-C2-P2.tex" "Success!!!"))
% (defun D () (interactive) (find-pdf-page      "~/LATEX/2020-1-C2-P2.pdf"))
% (defun d () (interactive) (find-pdftools-page "~/LATEX/2020-1-C2-P2.pdf"))
% (defun e () (interactive) (find-LATEX "2020-1-C2-P2.tex"))
% (defun u () (interactive) (find-latex-upload-links "2020-1-C2-P2"))
% (defun v () (interactive) (find-2a '(e) '(d)))
% (defun cv () (interactive) (C) (ee-kill-this-buffer) (v) (g))
% (find-pdf-page   "~/LATEX/2020-1-C2-P2.pdf")
% (find-sh0 "cp -v  ~/LATEX/2020-1-C2-P2.pdf /tmp/")
% (find-sh0 "cp -v  ~/LATEX/2020-1-C2-P2.pdf /tmp/pen/")
%   file:///home/edrx/LATEX/2020-1-C2-P2.pdf
%               file:///tmp/2020-1-C2-P2.pdf
%           file:///tmp/pen/2020-1-C2-P2.pdf
% http://angg.twu.net/LATEX/2020-1-C2-P2.pdf
% (find-LATEX "2019.mk")
% (find-C2-aula-links "2020-1-C2-P2" "p2" "p2")

% «.dicas»		(to "dicas")
% «.questao-1»		(to "questao-1")
% «.questao-2»		(to "questao-2")
% «.questao-3»		(to "questao-3")
% «.gabarito-1b»	(to "gabarito-1b")
% «.gabarito-1c»	(to "gabarito-1c")

\documentclass[oneside,12pt]{article}
\usepackage[colorlinks,citecolor=DarkRed,urlcolor=DarkRed]{hyperref} % (find-es "tex" "hyperref")
\usepackage{amsmath}
\usepackage{amsfonts}
\usepackage{amssymb}
\usepackage{pict2e}
\usepackage[x11names,svgnames]{xcolor} % (find-es "tex" "xcolor")
%\usepackage{colorweb}                 % (find-es "tex" "colorweb")
%\usepackage{tikz}
%
% (find-dn6 "preamble6.lua" "preamble0")
%\usepackage{proof}   % For derivation trees ("%:" lines)
%\input diagxy        % For 2D diagrams ("%D" lines)
%\xyoption{curve}     % For the ".curve=" feature in 2D diagrams
%
\usepackage{edrx15}               % (find-LATEX "edrx15.sty")
\input edrxaccents.tex            % (find-LATEX "edrxaccents.tex")
\input edrxchars.tex              % (find-LATEX "edrxchars.tex")
\input edrxheadfoot.tex           % (find-LATEX "edrxheadfoot.tex")
\input edrxgac2.tex               % (find-LATEX "edrxgac2.tex")
%
%\usepackage[backend=biber,
%   style=alphabetic]{biblatex}            % (find-es "tex" "biber")
%\addbibresource{catsem-slides.bib}        % (find-LATEX "catsem-slides.bib")
%
% (find-es "tex" "geometry")
\usepackage[a6paper, landscape,
            top=1.5cm, bottom=.25cm, left=1cm, right=1cm, includefoot
           ]{geometry}
%
\begin{document}

\catcode`\^^J=10
\directlua{dofile "dednat6load.lua"}  % (find-LATEX "dednat6load.lua")

% %L dofile "edrxtikz.lua"  -- (find-LATEX "edrxtikz.lua")
% %L dofile "edrxpict.lua"  -- (find-LATEX "edrxpict.lua")
% \pu

%\printbibliography

% «defs»  (to ".defs")
% (find-LATEX "edrx15.sty" "colors-2019")
\long\def\ColorRed   #1{{\color{Red1}#1}}
\long\def\ColorViolet#1{{\color{MagentaVioletLight}#1}}
\long\def\ColorViolet#1{{\color{Violet!50!black}#1}}
\long\def\ColorGreen #1{{\color{SpringDarkHard}#1}}
\long\def\ColorGreen #1{{\color{SpringGreenDark}#1}}
\long\def\ColorGreen #1{{\color{SpringGreen4}#1}}
\long\def\ColorGray  #1{{\color{GrayLight}#1}}
\long\def\ColorGray  #1{{\color{black!30!white}#1}}
\long\def\ColorBrown #1{{\color{Brown}#1}}
\long\def\ColorBrown #1{{\color{brown}#1}}

\long\def\ColorShort #1{{\color{SpringGreen4}#1}}
\long\def\ColorLong  #1{{\color{Red1}#1}}

\def\frown{\ensuremath{{=}{(}}}
\def\True {\mathbf{V}}
\def\False{\mathbf{F}}
\def\D    {\displaystyle}
\def\ph   {\phantom}
\def\veq  {\rotatebox{90}{$=$}}

\def\drafturl{http://angg.twu.net/LATEX/2020-1-C2.pdf}
\def\drafturl{http://angg.twu.net/2020.1-C2.html}
\def\draftfooter{\tiny \href{\drafturl}{\jobname{}} \ColorBrown{\shorttoday{} \hours}}

\setlength{\parindent}{0em}
\def\T(Total: #1 pts){{\bf(Total: #1 pts)}}
\def\T(Total: #1 pts){{\bf(Total: #1)}}
\def\B       (#1 pts){{\bf(#1 pts)}}
% Usage:
% 1) \T(Total: 2.34 pts) Foo
% a) \B(0.45 pts) Bar

% (find-angg ".emacs" "c2q192")


%  _____ _ _   _                               
% |_   _(_) |_| | ___   _ __   __ _  __ _  ___ 
%   | | | | __| |/ _ \ | '_ \ / _` |/ _` |/ _ \
%   | | | | |_| |  __/ | |_) | (_| | (_| |  __/
%   |_| |_|\__|_|\___| | .__/ \__,_|\__, |\___|
%                      |_|          |___/      
%
% «title»  (to ".title")

\thispagestyle{empty}

\begin{center}

\vspace*{1.2cm}

{\bf \Large Cálculo 2 - 2020.1}

\bsk

P2 (Segunda prova)

\bsk

Eduardo Ochs - RCN/PURO/UFF

\url{http://angg.twu.net/2020.1-C2.html}

\end{center}

\newpage

% «regras»  (to ".regras")
% (c2m201p2p 2 "regras")
% (c2m201p2     "regras")
% (c2m201p1p 2 "regras")
% (c2m201p1    "regras")

{\bf Regras para a P2:}

\ssk

As questões da P1 serão disponibilizadas às 18:00 da quarta
02/dez/2020 para uma turma e às 13:00 da quinta 03/dez/2020 para a
outra, e você deverá entregar as suas respostas \ColorRed{escritas à
  mão} até 48 horas depois do momento em que a prova foi
disponibilizada para a sua turma na plataforma Classroom. Se o
Classroom der algum problema mande também para este endereço de
e-mail:

\ssk

\ColorRed{eduardoochs@gmail.com}

\ssk

Provas entregues após este horário não serão considerados.

Durante as 48 horas da prova nem o professor nem o monitor responderão
perguntas sobre os assuntos da prova, mas você pode discutir com os
seus colegas e até com as pessoas do grupo da outra turma...
\ColorRed{só que as respostas devem ser individuais}.


% \newpage
% 
% % «dicas»  (to ".dicas")
% 
% {\bf Dicas}
% 
% \ssk
% 
% Os assuntos desta prova estão



\newpage

% «questao-1»  (to ".questao-1")
% (c2m201p2p 3 "questao-1")
% (c2m201p2    "questao-1")

{\bf Questão 1}

\T(Total: 2.0 pts)

\ssk

a) \B(0.5 pts) Faça o gráfico da função $\arcsen x$. Lembre que o
domínio dela é o conjunto $[-1,1]$.

b) \B(0.5 pts) Reveja o vídeo sobre como provar que $\frac{d}{dx} \ln
x = \frac 1x$ que eu preparei pra aula sobre a integração por frações
parciais. Adapte o método dele para o $\arcsen$: calcule $\frac{d}{dθ}
\arcsen \sen θ$ de dois jeitos diferentes, e use isto pra mostrar que
%
$$\arcsen'(\sen θ)\sqrt{1-(\senθ)^2} = 1.$$

c) \B(1.0 pts) Calcule $\frac{d}{ds} \arcsen s$. Seja bem claro e
detalhado na sua solução.


\newpage

% «questao-2»  (to ".questao-2")
% (c2m201p2p 4 "questao-2")
% (c2m201p2    "questao-2")

{\bf Questão 2}

\T(Total: 2.0 pts)

\ssk

Faça algo parecido para calcular $\frac{d}{dt} \arctan t$. Aqui você
vai precisar das identidades trigonométricas para tangente e secante
do final dos slides sobre substituição trigonométrica.

\newpage

% «questao-3»  (to ".questao-3")
% (c2m201p2p 5 "questao-3")
% (c2m201p2    "questao-3")

{\bf Questão 3}

\T(Total: 6.0 pts)

\ssk

a) \B(2.5 pts) Calcule
%
$$\intt {t^0 \sqrt{1+t^2}^{-2}}$$
%
usando a fórmula do final dos slides sobre substituições
trigonométricas que transforma $\intt {t^α \sqrt{1+t^2}^β}$ em algo
mais fácil de integrar.

b) \B(1.5 pts) Calcule
%
$$\intx{\frac{1}{1+x^2}}.$$

c) \B(2.0 pts) Calcule
%
$$\intx{\frac{1}{4x^2 + 1}}.$$



\newpage

% «gabarito-1b»  (to ".gabarito-1b")
% (c2m201p2p 99 "gabarito-1b")
% (c2m201p2     "gabarito-1b")

{\bf Gabarito parcial}

\def\arrayl #1{\begin{array}{l}#1\end{array}}
\def\parrayl#1{\left(\arrayl{#1}\right)}

% (find-LATEX "2020-1-C2-fracs-parcs.tex" "2020_deriv_ln.mp4")

\ssk

1b) Neste vídeo

\url{http://angg.twu.net/eev-videos/2020_deriv_ln.mp4}

nós vimos esta demonstração:
%
$$\arrayl{
    \text{Se $f(g(x)) = x$ então:} \\
    \ddx f(g(x)) = \ddx x = 1 \\[2pt]
    \ph{mmm}\veq \\
    % \ddx f(g(x)) = 
    f'(g(x))g'(x) \\
  }
$$

substituindo $f(u)$ por $\arcsen u$ e $g(x)$ por $\sen x$ na
demonstração acima obtemos:
%
$$\arrayl{
    \text{Se $\arcsen(\sen(x)) = x$ então:} \\
    \ddx \arcsen(\sen(x)) = \ddx x = 1 \\[2pt]
    \ph{mmm}\veq \\
    % \ddx \arcsen(\sen(x)) =
    \arcsen'(\sen(x))\sen'(x) \\
  }
$$

\newpage

Então temos:
%
$$\text{Se $\arcsen(\sen(θ)) = θ$ então $1 = \arcsen'(\sen(θ))\sen'(θ)$.}
$$

Quando $-\fracπ2 ≤ θ ≤ \fracπ2$ temos $\arcsen(\sen(θ)) = θ$, e aí:
%
$$\begin{array}{rcl}
  1 &=& \arcsen'(\sen(θ))\sen'(θ) \\
    &=& \arcsen'(\sen(θ))\cos(θ) \\
    &=& \arcsen'(\sen(θ))\sqrt{1-\sen^2θ}. \\
  \end{array}
$$

A terceira igualdade acima só vale para certos valores de $θ$... mas
quando $θ$ está no intervalo $[-\fracπ2,\fracπ2]$ temos $\cos(θ)≥0$ e
portanto $\cos(θ) = \sqrt{1-\sen^2θ}$. A maioria dos livros
``básicos'' ignora que precisamos ter $-\fracπ2 ≤ θ ≤ \fracπ2$ --- e
eu não esperava que alguém mencionasse a condição $-\fracπ2 ≤ θ ≤
\fracπ2$ nesta prova.

\newpage

% «gabarito-1c»  (to ".gabarito-1c")
% (c2m201p2p 8 "gabarito-1c")
% (c2m201p2    "gabarito-1c")

1c) Se substituirmos $θ$ por $\arcsen s$ em 
%
$$1 = \arcsen'(\sen(θ))\sqrt{1-\sen^2θ}
$$

obtemos:
%
$$\begin{array}{rcl}
            1 &=& \arcsen'(\sen(\arcsen s))\sqrt{1-(\sen \arcsen s)^2} \\
              &=& \arcsen'(s)\sqrt{1-s^2} \\
  \end{array}
$$
%
e daí:
%
$$\arcsen'(s) = \D \frac{1}{\sqrt{1-s^2}}.$$






% (find-es "ipython" "2020.1-C2-P1")
% (find-es "ipython" "2020.1-C2-P2")
% https://en.wikipedia.org/wiki/Inverse_trigonometric_functions#Derivatives_of_inverse_trigonometric_functions

\GenericWarning{Success:}{Success!!!}


\end{document}

%  __  __       _        
% |  \/  | __ _| | _____ 
% | |\/| |/ _` | |/ / _ \
% | |  | | (_| |   <  __/
% |_|  |_|\__,_|_|\_\___|
%                        
% <make>

 (eepitch-shell)
 (eepitch-kill)
 (eepitch-shell)
# (find-LATEXfile "2019planar-has-1.mk")
make -f 2019.mk STEM=2020-1-C2-P2 veryclean
make -f 2019.mk STEM=2020-1-C2-P2 pdf

% Local Variables:
% coding: utf-8-unix
% ee-tla: "c2m201p2"
% End:
