% (find-LATEX "2020-1-C2-TFC2.tex")
% (defun c () (interactive) (find-LATEXsh "lualatex -record 2020-1-C2-TFC2.tex" :end))
% (defun D () (interactive) (find-pdf-page      "~/LATEX/2020-1-C2-TFC2.pdf"))
% (defun d () (interactive) (find-pdftools-page "~/LATEX/2020-1-C2-TFC2.pdf"))
% (defun e () (interactive) (find-LATEX "2020-1-C2-TFC2.tex"))
% (defun u () (interactive) (find-latex-upload-links "2020-1-C2-TFC2"))
% (defun v () (interactive) (find-2a '(e) '(d)) (g))
% (find-pdf-page   "~/LATEX/2020-1-C2-TFC2.pdf")
% (find-sh0 "cp -v  ~/LATEX/2020-1-C2-TFC2.pdf /tmp/")
% (find-sh0 "cp -v  ~/LATEX/2020-1-C2-TFC2.pdf /tmp/pen/")
%   file:///home/edrx/LATEX/2020-1-C2-TFC2.pdf
%               file:///tmp/2020-1-C2-TFC2.pdf
%           file:///tmp/pen/2020-1-C2-TFC2.pdf
% http://angg.twu.net/LATEX/2020-1-C2-TFC2.pdf
% (find-LATEX "2019.mk")

% «.defs»		(to "defs")
% «.title»		(to "title")
% «.exercicio-1»	(to "exercicio-1")
% «.exercicio-2»	(to "exercicio-2")
% «.TFC2»		(to "TFC2")
% «.exercicio-3»	(to "exercicio-3")

\documentclass[oneside,12pt]{article}
\usepackage[colorlinks,citecolor=DarkRed,urlcolor=DarkRed]{hyperref} % (find-es "tex" "hyperref")
\usepackage{amsmath}
\usepackage{amsfonts}
\usepackage{amssymb}
\usepackage{pict2e}
\usepackage[x11names,svgnames]{xcolor} % (find-es "tex" "xcolor")
%\usepackage{colorweb}                 % (find-es "tex" "colorweb")
%\usepackage{tikz}
%
% (find-dn6 "preamble6.lua" "preamble0")
%\usepackage{proof}   % For derivation trees ("%:" lines)
%\input diagxy        % For 2D diagrams ("%D" lines)
%\xyoption{curve}     % For the ".curve=" feature in 2D diagrams
%
\usepackage{edrx15}               % (find-LATEX "edrx15.sty")
\input edrxaccents.tex            % (find-LATEX "edrxaccents.tex")
\input edrxchars.tex              % (find-LATEX "edrxchars.tex")
\input edrxheadfoot.tex           % (find-LATEX "edrxheadfoot.tex")
\input edrxgac2.tex               % (find-LATEX "edrxgac2.tex")
%
%\usepackage[backend=biber,
%   style=alphabetic]{biblatex}            % (find-es "tex" "biber")
%\addbibresource{catsem-slides.bib}        % (find-LATEX "catsem-slides.bib")
%
% (find-es "tex" "geometry")
\usepackage[a6paper, landscape,
            top=1.5cm, bottom=.25cm, left=1cm, right=1cm, includefoot
           ]{geometry}
%
\begin{document}

\catcode`\^^J=10
\directlua{dofile "dednat6load.lua"}  % (find-LATEX "dednat6load.lua")

% %L dofile "edrxtikz.lua"  -- (find-LATEX "edrxtikz.lua")
% %L dofile "edrxpict.lua"  -- (find-LATEX "edrxpict.lua")
% \pu

% «defs»  (to ".defs")
% (find-LATEX "edrx15.sty" "colors-2019")
\long\def\ColorRed   #1{{\color{Red1}#1}}
\long\def\ColorViolet#1{{\color{MagentaVioletLight}#1}}
\long\def\ColorViolet#1{{\color{Violet!50!black}#1}}
\long\def\ColorGreen #1{{\color{SpringDarkHard}#1}}
\long\def\ColorGreen #1{{\color{SpringGreenDark}#1}}
\long\def\ColorGreen #1{{\color{SpringGreen4}#1}}
\long\def\ColorGray  #1{{\color{GrayLight}#1}}
\long\def\ColorGray  #1{{\color{black!30!white}#1}}
\long\def\ColorBrown #1{{\color{Brown}#1}}
\long\def\ColorBrown #1{{\color{brown}#1}}

\long\def\ColorShort #1{{\color{SpringGreen4}#1}}
\long\def\ColorLong  #1{{\color{Red1}#1}}

\def\frown{\ensuremath{{=}{(}}}
\def\True {\mathbf{V}}
\def\False{\mathbf{F}}

\def\drafturl{http://angg.twu.net/LATEX/2020-1-C2.pdf}
\def\drafturl{http://angg.twu.net/2020.1-C2.html}
\def\draftfooter{\tiny \href{\drafturl}{\jobname{}} \ColorBrown{\shorttoday{} \hours}}

% (find-angg ".emacs" "c2q192")


%  _____ _ _   _                               
% |_   _(_) |_| | ___   _ __   __ _  __ _  ___ 
%   | | | | __| |/ _ \ | '_ \ / _` |/ _` |/ _ \
%   | | | | |_| |  __/ | |_) | (_| | (_| |  __/
%   |_| |_|\__|_|\___| | .__/ \__,_|\__, |\___|
%                      |_|          |___/      
%
% «title»  (to ".title")
% (c2m201tfc2p 1 "title")
% (c2m201tfc2    "title")

\thispagestyle{empty}

\begin{center}

\vspace*{1.2cm}

{\bf \Large Cálculo 2 - 2020.1}

\bsk

Aulas 10 e 11: O TFC2

(O segundo Teorema Fundamental do Cálculo)

\bsk

Eduardo Ochs - RCN/PURO/UFF

\url{http://angg.twu.net/2020.1-C2.html}

\end{center}

\newpage


% «exercicio-1»  (to ".exercicio-1")
% (c2m201tfc2p 2 "exercicio-1")
% (c2m201tfc2    "exercicio-1")

{\bf Introdução}

\ssk

Boa parte do que a gente vai ver hoje é uma versão melhorada das
idéias do exercício 6 da aula passada... então comecem revisando o
exercício 6 da aula passada (e as dicas da página 11). Link:

\ssk

\url{http://angg.twu.net/LATEX/2020-1-C2-TFCs.pdf}

\bsk

{\bf Exercício 1.}

Seja $f(x) = -x^2 +4x$, como na aula passada.

Sejam $a=1$ e $b=2$.

\ssk

a) Encontre uma antiderivada para $f$ e chame-a de $H(x)$.

b) Encontre uma antiderivada $F(x)$ para $f$ que obedeça $F(a)=0$.

c) Use esta $F(x)$ para calcular $\Intx{a}{b}{f(x)}$.

\msk

{\sl Lembre que $a=1$ e $b=2$!}


\newpage

% «exercicio-2»  (to ".exercicio-2")
% (c2m201tfc2p 3 "exercicio-2")
% (c2m201tfc2    "exercicio-2")

{\bf Exercício 2.}

Seja $f(x) = -x^2 +4x$ (de novo).

Sejam $a=2$ e $b=3$.

\ssk

a) Encontre uma antiderivada para $f$ e chame-a de $H(x)$.

b) Encontre uma antiderivada $F(x)$ para $f$ que obedeça $F(a)=0$.

c) Use esta $F(x)$ para calcular $\Intx{a}{b}{f(x)}$.

d) Faça o gráfico da função $D(x) = F(x) - H(x)$. Verifique que ela é
constante. Chame de $C$ o número tal que $D(x) = C$. Verifique que
$F(x) = H(x) + C$.



\newpage

No item (c) da página anterior você calculou $\Intx{a}{b}{f(x)}$ por:
%
$$\Intx{a}{b}{f(x)} = F(b) - F(a).$$

Repare que:
%
$$\begin{array}{rcl}
  \Intx{a}{b}{f(x)} &=& F(b) - F(a) \\
                    &=& (H(b) + C) - (H(a) + C) \\
                    &=& H(b) - H(a) \\
  \end{array}
$$

Ou seja, dá pra calcular $\Intx{a}{b}{f(x)}$ por $H(b) - H(a)$, sem a
gente precisar fazer o item (b), que é trabalhoso...

\newpage

% «TFC2»  (to ".TFC2")
% (c2m201tfc2p 5 "TFC2")
% (c2m201tfc2    "TFC2")

{\bf O TFC2}

\ssk

Digamos que $f(x)$ é uma função integrável

e que queremos calcular $\Intx{a}{b}{f(x)}$.

Digamos que $H(x)$ é alguma antiderivada contínua de $f(x)$.

(Qualquer uma serve, mas tem que ser contínua.)

\ColorRed{Então (Teorema! Este é o TFC2!):}
%
$$\begin{array}{rcl}
  \Intx{a}{b}{f(x)} &=& H(b) - H(a) \\
  \end{array}
$$

\newpage

% «exercicio-3»  (to ".exercicio-3")
% (c2m201tfc2p 6 "exercicio-3")
% (c2m201tfc2    "exercicio-3")

{\bf Exercício 3.}

\ssk

Digamos que $g(x) = 6-x$.

a) Calcule a área $\Intx{2}{4}{g(x)}$ pela fórmula da área do
trapézio.

b) Calcule a área $\Intx{4}{6}{g(x)}$ pela fórmula da área do
trapézio.

c) Use a fórmula da área do trapézio pra encontrar uma fórmula que
calcule a área $\Intx{a}{b}{g(x)}$ para quaisquer $a,b∈[0,6]$. Teste a
sua fórmula nos exercícios (a) e (b).

d) Encontre uma antiderivada para $g(x)$.

e) Use essa antiderivada pra encontrar uma fórmula para calcular
$\Intx{a}{b}{g(x)}$ e verifique que essa fórmula é equivalente à que
você obteve no item (c).

\bsk

\ColorRed{
Agora que você entendeu bem antiderivadas

faça os exercícios 4 e 5 da aula passada.
}



%\printbibliography

\end{document}

%  __  __       _        
% |  \/  | __ _| | _____ 
% | |\/| |/ _` | |/ / _ \
% | |  | | (_| |   <  __/
% |_|  |_|\__,_|_|\_\___|
%                        
% <make>

 (eepitch-shell)
 (eepitch-kill)
 (eepitch-shell)
# (find-LATEXfile "2019planar-has-1.mk")
make -f 2019.mk STEM=2020-1-C2-TFC2 veryclean
make -f 2019.mk STEM=2020-1-C2-TFC2 pdf

% Local Variables:
% coding: utf-8-unix
% ee-tla: "c2m201tfc2"
% End:
