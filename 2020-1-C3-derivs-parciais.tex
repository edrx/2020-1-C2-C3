% (find-LATEX "2020-1-C3-derivs-parciais.tex")
% (defun c () (interactive) (find-LATEXsh "lualatex -record 2020-1-C3-derivs-parciais.tex" :end))
% (defun D () (interactive) (find-pdf-page      "~/LATEX/2020-1-C3-derivs-parciais.pdf"))
% (defun d () (interactive) (find-pdftools-page "~/LATEX/2020-1-C3-derivs-parciais.pdf"))
% (defun e () (interactive) (find-LATEX "2020-1-C3-derivs-parciais.tex"))
% (defun u () (interactive) (find-latex-upload-links "2020-1-C3-derivs-parciais"))
% (defun v () (interactive) (find-2a '(e) '(d)) (g))
% (find-pdf-page   "~/LATEX/2020-1-C3-derivs-parciais.pdf")
% (find-sh0 "cp -v  ~/LATEX/2020-1-C3-derivs-parciais.pdf /tmp/")
% (find-sh0 "cp -v  ~/LATEX/2020-1-C3-derivs-parciais.pdf /tmp/pen/")
%   file:///home/edrx/LATEX/2020-1-C3-derivs-parciais.pdf
%               file:///tmp/2020-1-C3-derivs-parciais.pdf
%           file:///tmp/pen/2020-1-C3-derivs-parciais.pdf
% http://angg.twu.net/LATEX/2020-1-C3-derivs-parciais.pdf
% (find-LATEX "2019.mk")
% (find-C3-aula-links "2020-1-C3-derivs-parciais" "12" "derps")

% «.defs»		(to "defs")
% «.title»		(to "title")
% «.exercicio-1»	(to "exercicio-1")
% «.exercicio-2»	(to "exercicio-2")
% «.tipos»		(to "tipos")
% «.exercicio-3»	(to "exercicio-3")
% «.video»		(to "video")

\documentclass[oneside,12pt]{article}
\usepackage[colorlinks,citecolor=DarkRed,urlcolor=DarkRed]{hyperref} % (find-es "tex" "hyperref")
\usepackage{amsmath}
\usepackage{amsfonts}
\usepackage{amssymb}
\usepackage{pict2e}
\usepackage[x11names,svgnames]{xcolor} % (find-es "tex" "xcolor")
%\usepackage{colorweb}                 % (find-es "tex" "colorweb")
%\usepackage{tikz}
%
% (find-dn6 "preamble6.lua" "preamble0")
%\usepackage{proof}   % For derivation trees ("%:" lines)
%\input diagxy        % For 2D diagrams ("%D" lines)
%\xyoption{curve}     % For the ".curve=" feature in 2D diagrams
%
\usepackage{edrx15}               % (find-LATEX "edrx15.sty")
\input edrxaccents.tex            % (find-LATEX "edrxaccents.tex")
\input edrxchars.tex              % (find-LATEX "edrxchars.tex")
\input edrxheadfoot.tex           % (find-LATEX "edrxheadfoot.tex")
\input edrxgac2.tex               % (find-LATEX "edrxgac2.tex")
%
%\usepackage[backend=biber,
%   style=alphabetic]{biblatex}            % (find-es "tex" "biber")
%\addbibresource{catsem-slides.bib}        % (find-LATEX "catsem-slides.bib")
%
% (find-es "tex" "geometry")
\usepackage[a6paper, landscape,
            top=1.5cm, bottom=.25cm, left=1cm, right=1cm, includefoot
           ]{geometry}
%
\begin{document}

\catcode`\^^J=10
\directlua{dofile "dednat6load.lua"}  % (find-LATEX "dednat6load.lua")

% %L dofile "edrxtikz.lua"  -- (find-LATEX "edrxtikz.lua")
% %L dofile "edrxpict.lua"  -- (find-LATEX "edrxpict.lua")
% \pu

% «defs»  (to ".defs")
% (find-LATEX "edrx15.sty" "colors-2019")
\long\def\ColorRed   #1{{\color{Red1}#1}}
\long\def\ColorViolet#1{{\color{MagentaVioletLight}#1}}
\long\def\ColorViolet#1{{\color{Violet!50!black}#1}}
\long\def\ColorGreen #1{{\color{SpringDarkHard}#1}}
\long\def\ColorGreen #1{{\color{SpringGreenDark}#1}}
\long\def\ColorGreen #1{{\color{SpringGreen4}#1}}
\long\def\ColorGray  #1{{\color{GrayLight}#1}}
\long\def\ColorGray  #1{{\color{black!30!white}#1}}
\long\def\ColorBrown #1{{\color{Brown}#1}}
\long\def\ColorBrown #1{{\color{brown}#1}}

\long\def\ColorShort #1{{\color{SpringGreen4}#1}}
\long\def\ColorLong  #1{{\color{Red1}#1}}

\def\frown{\ensuremath{{=}{(}}}
\def\True {\mathbf{V}}
\def\False{\mathbf{F}}

\def\D{\displaystyle}

\def\co#1{{%
  \def\\{\char92}%
  \tt#1%
  }}

\def\drafturl{http://angg.twu.net/LATEX/2020-1-C2.pdf}
\def\drafturl{http://angg.twu.net/2020.1-C2.html}
\def\draftfooter{\tiny \href{\drafturl}{\jobname{}} \ColorBrown{\shorttoday{} \hours}}


%  _____ _ _   _                               
% |_   _(_) |_| | ___   _ __   __ _  __ _  ___ 
%   | | | | __| |/ _ \ | '_ \ / _` |/ _` |/ _ \
%   | | | | |_| |  __/ | |_) | (_| | (_| |  __/
%   |_| |_|\__|_|\___| | .__/ \__,_|\__, |\___|
%                      |_|          |___/      
%
% «title»  (to ".title")
% (c3m201derpsp 1 "title")
% (c3m201derpsa   "title")

\thispagestyle{empty}

\begin{center}

\vspace*{1.2cm}

{\bf \Large Cálculo 3 - 2020.1}

\bsk

Aula 13: Derivadas parciais

\bsk

Eduardo Ochs - RCN/PURO/UFF

\url{http://angg.twu.net/2020.1-C3.html}

\end{center}

\newpage

% «exercicio-1»  (to ".exercicio-1")
% (c3m201derpsp 2 "exercicio-1")
% (c3m201derps    "exercicio-1")

Ou últimos exercícios da aula passada -- link:

\ssk

\url{http://angg.twu.net/LATEX/2020-1-C3-superficies-2.pdf}

\ssk

\noindent eram uma preparação pra gente começar a entender
\ColorRed{derivadas parciais} e o início do capítulo 5 do Bortolossi.

\msk

As nossas duas primeiras definições vão ser estas aqui:

$$\begin{array}{rcl}
  \D \frac{∂F}{∂x}(x_0,y_0) &=& \D \lim_{Δx→0} \frac{F(x_0+Δx,y_0)-F(x_0,y_0)}{Δx} \\[10pt]
  \D \frac{∂F}{∂y}(x_0,y_0) &=& \D \lim_{Δy→0} \frac{F(x_0,y_0+Δy)-F(x_0,y_0)}{Δy} \\
  \end{array}
$$

\bsk

{\bf Exercício 1.} Descubra como transformar a definição 5.1 do
Bortolossi (p.170) nas fórmulas acima.

\newpage

Obs: em Português a gente chama o `$∂$' de ``derrom''. Em Francês acho
que ele se chama ```$d$' rond'', e devem ter pego a pronúncia disso
e aportuguesado. Em \LaTeX{} o `$∂$' é `\co{\\partial}'.

% (find-bortolossi5page (+ -161 162) "5. Derivadas parciais")
% (find-bortolossi5page (+ -161 162) "5.1. Lembrando Cálculo 1")
% (find-bortolossi5page (+ -162 164) "5.2. Definições e exemplos")
% (find-bortolossi5page (+ -162 165)   "Fig. 5.2: Interpretação geométrica")
% (find-bortolossi5page (+ -162 167)   "Exemplo 5.1: Cobb-Douglas")
% (find-bortolossi5page (+ -162 170)   "Definição 5.1: derivada parcial")
% (find-bortolossi5page (+ -162 171)   "a notação D_1 f é a mais clara")
% (find-bortolossi5page (+ -162 172)   "omitir os pontos onde as parciais são calculadas")

\bsk

Nós vamos usar estas quatro \ColorRed{fórmulas para aproximações}:

$$\begin{array}{lrcl}
  1) & \D \frac{∂F}{∂x}(x_0,y_0) &≈& \D \frac{F(x_0+Δx,y_0)-F(x_0,y_0)}{Δx} \\[10pt]
  2) & \D \frac{∂F}{∂y}(x_0,y_0) &≈& \D \frac{F(x_0,y_0+Δy)-F(x_0,y_0)}{Δy} \\[12pt]
  3) & \D          F(x_0+Δx,y_0) &≈& \D F(x_0,y_0) + \frac{∂F}{∂x}(x_0,y_0) Δx \\[10pt]
  4) & \D          F(x_0,y_0+Δy) &≈& \D F(x_0,y_0) + \frac{∂F}{∂y}(x_0,y_0) Δy \\
  \end{array}
$$


\newpage

% «exercicio-2»  (to ".exercicio-2")
% (c3m201derpsp 4 "exercicio-2")
% (c3m201derps    "exercicio-2")

{\bf Exercício 2.}

\ssk

% (c3m201sups2p 3 "exercicio-1")
% (c3m201sups2    "exercicio-1")

Pegue o diagrama de numerozinhos que você fez para a função
%
$$ F(x,y) =
\begin{cases}
  \sqrt{5^2 - x^2 - y^2} & \text{quando $5^2 - x^2 - y^2≥0$}, \\
  0 & \text{quando $5^2 - x^2 - y^2<0$,} \\
 \end{cases}
$$

na aula passada -- link:

\ssk

\url{http://angg.twu.net/LATEX/2020-1-C3-superficies-2.pdf}

\ssk

e use-o para calcular algumas aproximações para
$\frac{∂F}{∂x}(x_0,y_0)$ e $\frac{∂F}{∂y}(x_0,y_0)$ usando as fórmulas
1 e 2 do slide anterior. Mais precisamente: sejam $x_0=2$, $y_0=4$, e:

\msk

a) calcule a aproximação para $\frac{∂F}{∂x}(x_0,y_0)$ usando $Δx=1$.

b) calcule a aproximação para $\frac{∂F}{∂x}(x_0,y_0)$ usando $Δx=-1$.

c) calcule a aproximação para $\frac{∂F}{∂y}(x_0,y_0)$ usando $Δy=1$.

d) calcule a aproximação para $\frac{∂F}{∂y}(x_0,y_0)$ usando $Δy=-1$.


\newpage

% «tipos»  (to ".tipos")
% (c3m201derpsp 5 "tipos")
% (c3m201derps    "tipos")

Dica: {\bf TUDO} que nós estamos fazendo agora pode ser {\sl
  visualizado} e {\sl tipado}. Você já viu um pouco de tipos em {\tt
  C} e em Física; em Física os ``tipos'' são parcialmente determinados
pelas unidades --- metros são distância, segundos são tempo,
metros/segundo é uma unidade de velocidade, e assim por diante...

Aqui a gente pode pensar que $x_0$ e $x_1$ são posições no eixo
horizontal, $y_0$ e $y_1$ são posições no eixo vertical, $Δx$ é uma
distância na horizontal, $Δy$ é uma distância na vertical,
$\frac{Δy}{Δx}$ é uma {\sl inclinação} (qual? Do quê?), e assim por
diante.

\newpage

% «exercicio-3»  (to ".exercicio-3")
% (c3m201derpsp 6 "exercicio-3")
% (c3m201derps    "exercicio-3")

{\bf Exercício 3.}

\ssk

Veja se você consegue ``tipar'' (no sentido acima) cada subexpressão
de cada uma das contas que você fez no Exercício 2. Dica: use chaves
sob as subexpressões deste modo aqui,

\def\rq{\ColorRed{?}}
\def\undq#1{\underbrace{#1}_{\rq}}

$$\undq{
  \undq{(\undq{F(\undq{\undq{\undq{x_0} + \undq{Δx}},\undq{y_0}})}
        - \undq{F(\undq{\undq{x_0},\undq{y_0}})})} / \undq{Δx}
  }
$$

e escreva os seus tipos nos lugares em que eu pus as `$\rq$'s. Use
Português onde quiser e improvise o quanto precisar.



%\printbibliography

\end{document}

% «video»  (to ".video")
# (find-ssr-links "2020_C3_tipos" "grad")
# (find-es "pulseaudio" "pulseaudio-kill")

 (eepitch-shell)
 (eepitch-kill)
 (eepitch-shell)
cd /tmp/
cp -v  ~/LATEX/2020-1-C3-derivs-parciais.pdf /tmp/
xournalpp /tmp/2020-1-C3-derivs-parciais.pdf


%  __  __       _        
% |  \/  | __ _| | _____ 
% | |\/| |/ _` | |/ / _ \
% | |  | | (_| |   <  __/
% |_|  |_|\__,_|_|\_\___|
%                        
% <make>

 (eepitch-shell)
 (eepitch-kill)
 (eepitch-shell)
# (find-LATEXfile "2019planar-has-1.mk")
make -f 2019.mk STEM=2020-1-C3-derivs-parciais veryclean
make -f 2019.mk STEM=2020-1-C3-derivs-parciais pdf

% Local Variables:
% coding: utf-8-unix
% ee-tla: "c3m201derps"
% End:
