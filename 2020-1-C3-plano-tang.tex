% (find-LATEX "2020-1-C3-plano-tang.tex")
% (defun c () (interactive) (find-LATEXsh "lualatex -record 2020-1-C3-plano-tang.tex" :end))
% (defun D () (interactive) (find-pdf-page      "~/LATEX/2020-1-C3-plano-tang.pdf"))
% (defun d () (interactive) (find-pdftools-page "~/LATEX/2020-1-C3-plano-tang.pdf"))
% (defun e () (interactive) (find-LATEX "2020-1-C3-plano-tang.tex"))
% (defun u () (interactive) (find-latex-upload-links "2020-1-C3-plano-tang"))
% (defun v () (interactive) (find-2a '(e) '(d)) (g))
% (find-pdf-page   "~/LATEX/2020-1-C3-plano-tang.pdf")
% (find-sh0 "cp -v  ~/LATEX/2020-1-C3-plano-tang.pdf /tmp/")
% (find-sh0 "cp -v  ~/LATEX/2020-1-C3-plano-tang.pdf /tmp/pen/")
%   file:///home/edrx/LATEX/2020-1-C3-plano-tang.pdf
%               file:///tmp/2020-1-C3-plano-tang.pdf
%           file:///tmp/pen/2020-1-C3-plano-tang.pdf
% http://angg.twu.net/LATEX/2020-1-C3-plano-tang.pdf
% (find-LATEX "2019.mk")
% (find-C3-aula-links "2020-1-C3-plano-tang" "14" "pltan")

% «.defs»		(to "defs")
% «.title»		(to "title")
% «.danilo-pereira»	(to "danilo-pereira")
% «.exercicio-1»	(to "exercicio-1")
% «.exercicio-2»	(to "exercicio-2")
% «.derivada»		(to "derivada")
% «.exercicio-3»	(to "exercicio-3")
% «.exercicio-4»	(to "exercicio-4")
% «.mini-teste-1»	(to "mini-teste-1")
% «.miniteste-regras»	(to "miniteste-regras")
% «.video»		(to "video")

\documentclass[oneside,12pt]{article}
\usepackage[colorlinks,citecolor=DarkRed,urlcolor=DarkRed]{hyperref} % (find-es "tex" "hyperref")
\usepackage{amsmath}
\usepackage{amsfonts}
\usepackage{amssymb}
\usepackage{pict2e}
\usepackage[x11names,svgnames]{xcolor} % (find-es "tex" "xcolor")
%\usepackage{colorweb}                 % (find-es "tex" "colorweb")
%\usepackage{tikz}
%
% (find-dn6 "preamble6.lua" "preamble0")
%\usepackage{proof}   % For derivation trees ("%:" lines)
%\input diagxy        % For 2D diagrams ("%D" lines)
%\xyoption{curve}     % For the ".curve=" feature in 2D diagrams
%
\usepackage{edrx15}               % (find-LATEX "edrx15.sty")
\input edrxaccents.tex            % (find-LATEX "edrxaccents.tex")
\input edrxchars.tex              % (find-LATEX "edrxchars.tex")
\input edrxheadfoot.tex           % (find-LATEX "edrxheadfoot.tex")
\input edrxgac2.tex               % (find-LATEX "edrxgac2.tex")
%
%\usepackage[backend=biber,
%   style=alphabetic]{biblatex}            % (find-es "tex" "biber")
%\addbibresource{catsem-slides.bib}        % (find-LATEX "catsem-slides.bib")
%
% (find-es "tex" "geometry")
\usepackage[a6paper, landscape,
            top=1.5cm, bottom=.25cm, left=1cm, right=1cm, includefoot
           ]{geometry}
%
\begin{document}

\catcode`\^^J=10
\directlua{dofile "dednat6load.lua"}  % (find-LATEX "dednat6load.lua")

% %L dofile "edrxtikz.lua"  -- (find-LATEX "edrxtikz.lua")
% %L dofile "edrxpict.lua"  -- (find-LATEX "edrxpict.lua")
% \pu

% «defs»  (to ".defs")
% (find-LATEX "edrx15.sty" "colors-2019")
\long\def\ColorRed   #1{{\color{Red1}#1}}
\long\def\ColorViolet#1{{\color{MagentaVioletLight}#1}}
\long\def\ColorViolet#1{{\color{Violet!50!black}#1}}
\long\def\ColorGreen #1{{\color{SpringDarkHard}#1}}
\long\def\ColorGreen #1{{\color{SpringGreenDark}#1}}
\long\def\ColorGreen #1{{\color{SpringGreen4}#1}}
\long\def\ColorGray  #1{{\color{GrayLight}#1}}
\long\def\ColorGray  #1{{\color{black!30!white}#1}}
\long\def\ColorBrown #1{{\color{Brown}#1}}
\long\def\ColorBrown #1{{\color{brown}#1}}

\long\def\ColorShort #1{{\color{SpringGreen4}#1}}
\long\def\ColorLong  #1{{\color{Red1}#1}}

\def\frown{\ensuremath{{=}{(}}}
\def\True {\mathbf{V}}
\def\False{\mathbf{F}}

\def\drafturl{http://angg.twu.net/LATEX/2020-1-C2.pdf}
\def\drafturl{http://angg.twu.net/2020.1-C2.html}
\def\draftfooter{\tiny \href{\drafturl}{\jobname{}} \ColorBrown{\shorttoday{} \hours}}


%  _____ _ _   _                               
% |_   _(_) |_| | ___   _ __   __ _  __ _  ___ 
%   | | | | __| |/ _ \ | '_ \ / _` |/ _` |/ _ \
%   | | | | |_| |  __/ | |_) | (_| | (_| |  __/
%   |_| |_|\__|_|\___| | .__/ \__,_|\__, |\___|
%                      |_|          |___/      
%
% «title»  (to ".title")
% (c3m201pltanp 1 "title")
% (c3m201pltana   "title")

\thispagestyle{empty}

\begin{center}

\vspace*{1.2cm}

{\bf \Large Cálculo 3 - 2020.1}

\bsk

Aula 14 e 15: Introdução a planos tangentes

(e à derivada --- \ColorRed{e mini-teste 1})

\bsk

Eduardo Ochs - RCN/PURO/UFF

\url{http://angg.twu.net/2020.1-C3.html}

\end{center}

\newpage

%\printbibliography

% «danilo-pereira»  (to ".danilo-pereira")
% (find-youtubedl-links "/sda5/videos/" "Calculo II - Derivada Direcional e Vetor Gradiente (1 de 2)" "nmZ1Wmk7wcY" ".mp4" "dpddvg1")
% (code-video "dpddvg1video" "/sda5/videos/Calculo II - Derivada Direcional e Vetor Gradiente (1 de 2)-nmZ1Wmk7wcY.mp4")
% (find-dpddvg1video "0:00")

Na aula passada nós começamos a ver derivadas parciais, mas num caso
complicado, e eu pedi pra vocês assistirem este vídeo aqui, do Danilo
Pereira,

\ssk

\url{http://www.youtube.com/watch?v=nmZ1Wmk7wcY}

\ssk

\noindent
chamado ``Cálculo II - Derivada Direcional e Vetor Gradiente (1 de
2)'', que também começa direto em casos bem complicados.

Hoje nós vamos ver algo bem mais simples: {\sl planos}.

\newpage

% «exercicio-1»  (to ".exercicio-1")
% (c3m201pltanp 3 "exercicio-1")
% (c3m201pltan    "exercicio-1")

{\bf Definição.} Vou dizer que uma função $F: \R^2 → \R$ {\sl é de
  primeira ordem} quando existirem $a,b,c∈\R$ tais que $F$ é da forma:
%
$$F(x,y) \;\; = \;\; ax + by +c$$

\bsk

{\bf Exercício 1.}

Para cada uma das funções abaixo converta-a para a forma $F(x,y) =
ax+by+c$ e diga ``quem são o $a$, o $b$ e o $c$ dela''.

\ssk

a) $F(x,y) = 2(x+y) + 3 (x-4y) - 20$

b) $F(x,y) = (x+y) + (x-y) + 42$

\newpage

% «exercicio-2»  (to ".exercicio-2")
% (c3m201pltanp 4 "exercicio-2")
% (c3m201pltan    "exercicio-2")

{\bf Exercício 2.}

Seja $F(x,y) = 3 + 2x +y$.

a) Desenhe o diagrama de numerozinhos da $F$.

b) Calcule $F(0,2)$.

c) Desenhe a curva de nível de $z=F(0,2)$.

d) Desenhe a curva de nível de $z=F(2,2)$.

e) Desenhe a curva de nível de $z=9$.

f) Desenhe a curva de nível de $z=3$.

g) Sempre que uma $F:\R^2 → \R$ for uma função de primeira ordem as
suas curvas de nível vão ser ``equiespaçadas''. Use isto para
descobrir como desenhar as curvas de nível da $F$ para $z=0$, $z=1$,
$z=2$, $z=3$, $z=4$. Escreva do lado de cada uma delas ``$z=0$'',
``$z=1$'', etc, pra você não se perder.

\newpage

{\bf Exercício 2 (continuação).}

\ssk

h) Escolha algum vetor $\vv$ não-nulo que seja paralelo às curvas de
nível que você acabou de desenhar. Diga as componentes dele.

i) Calcule as derivadas parciais da $F$.

j) Calcule o gradiente da $F$. Dica: onde o vídeo do Danilo Pereira
define gradiente? Ele usa a mesma notação que o Bortolossi usa no
capítulo 8 do livro dele?

k) O gradiente da $F$ é ortogonal ao vetor $\vv$ do item h? Calcule o
produto interno deles.

% (find-bortolossi8page (+ -290 291) "8. Derivadas direcionais e o vetor gradiente")

\newpage

Seja $F(x,y) = 3 + 2x +y$,

Slogan: ``o vetor gradiente de $F$ diz a direção

pra onde a $F$ cresce mais rápido''.

Sejam $(x_0,y_0) = (2,1)$, $C = \setofst{(x_0,y_0) + \vv}{||\vv||=1}$.

Veja este vídeo:

\ssk

\url{http://angg.twu.net/eev-videos/2020_vetor_gradiente.mp4} 


\newpage

% «derivada»  (to ".derivada")
% (c3m201pltanp 7 "derivada")
% (c3m201pltan    "derivada")

{\bf Derivada}

\ssk

Se $F$ é um função de $\R^m$ em $R^n$ a derivada de $F$ vai ser uma
{\sl matriz}... isto é bem complicado e vai ficar pra segunda parte do
curso --- mas hoje vamos ver um caso particular, no qual a derivada
$Df$ é uma matriz {\sl horizontal}, e este caso vai nos ajudar a
entender algumas coisas do video do Danilo Pereira.

\msk

{\bf Definição} (temporária):

Se $F:\R^2 → \R$ e $p_0$ é um ponto de $\R^2$ então
%
$$\begin{array}{rcl}
  DF(p_0) &=& \pmat{\frac{∂F}{∂x}(p_0) & \frac{∂F}{∂y}(p_0)} \\
  DF      &=& \pmat{\frac{∂F}{∂x}      & \frac{∂F}{∂y}} \\
  \end{array}
$$

Obs: ela é ``temporária'' porque depois que nós entendermos as páginas
252 até 263 do Bortolossi a nossa definição ``de verdade'' da derivada
em várias dimensões vai ser algo bem mais geral, e as equações acima
vão ser só consequências ``óbvias'' da definição mais geral.

% (c3m192p 5 "ponto-base")
% (c3m192    "ponto-base")
% (find-LATEX "2019-2-C3-material.tex" "ponto-base")
% (find-LATEX "2019-2-C3-material.tex" "ponto-base" "jacobiana")

\newpage

% «exercicio-3»  (to ".exercicio-3")

{\bf Exercício 3.}

\ssk

Slogan: ``Funções de primeira ordem têm derivada constante''.

Seja $F(x,y) = 3 + 2x + y$.

\ssk

Calcule:

a) $DF(10, 20)$

b) $DF(42, 99)$

\msk

Seja $G(p_1) = F(p_1) + DF(p_0)(p_1 - p_0)$,

onde $F$ continua a mesma.

Digamos que $p_0 = (1,4)$ e $p_1 = (2,5)$.

c) Segundo as nossas convenções quem são $x_0$, $y_0$, $x_1$, $y_1$?

d) Interpretando $(p_1 - p_0)$ como um vetor vertical, isto é,

$(p_1 - p_0) = \psm{x_1-x_0 \\ y_1-y_0}$, Calcule $G(p_1)$.

e) Calcule $F(p_1)$ e compare o seu valor com $G(p_1)$.


\newpage

% «exercicio-4»  (to ".exercicio-4")
% (c3m201pltanp 9 "exercicio-4")
% (c3m201pltan    "exercicio-4")

No exercício da página anterior você aproximou uma função $F$ de
primeira ordem por uma função $G$ que também era de primeira ordem, e
você deve ter descoberto que no ponto $p_1$ que eu dei a gente tinha
$G(p_1) = F(p_1)$... agora vamos tentar fazer algo \ColorRed{parecido}
começando com uma $F$ que não é de primeira ordem.

\bsk

{\bf Exercício 4.}

\ssk

Seja $F(x,y) = y(y-x)$.

a) Faça o diagrama de numerozinhos para esta $F$.

b) Desenhe nele a curva de nível de $F(x,y)=0$.

c) Calcule $F(4,1)$.

b) Desenhe a curva de nível de $F(x,y)=F(4,1)$.

e) Calcule $DF(4,1)$.

f) Seja $G(p_1) = F(p_0) + DF(p_0) (p_1 - p_0)$. Ponha esta $G$ na
forma $G(x,y) = ax + by + c$. Quem são $a$, $b$, e $c$?

\newpage

{\bf Exercício 4 (continuação).}

\ssk

g) Calcule $DG$ e $DG(4,1)$.

h) Desenhe a curva de nível de $G(x,y)=G(4,1)$.

i) Essa curva de nível é, ou parece ser, tangente à curva de nível da
$F$ que você obteve no item (b)?

\bsk

j) Refaça os itens anteriores mas agora usando $p_0 = (3,2)$ ao invés
de $p_0 = (4,1)$.



\newpage

% «mini-teste-1»  (to ".mini-teste-1")
% (c3m201pltanp 11 "mini-teste-1")
% (c3m201pltan     "mini-teste-1")

{\bf Mini-teste 1}

\ssk

Sejam $F(x,y) = (x-2)(x+y)$ e $p_0 = (2,1)$.

a) Faça o diagrama de numerozinhos para esta $F$.

b) Desenhe nele as curvas de nível de $F(x,y)=0$ e $F(x,y) = F(p_0)$.

c) Calcule $DF(p_0)$.

d) Seja $G(p_1) = F(p_0) + DF(p_0) (p_1 - p_0)$. Ponha esta $G$ na
forma $G(x,y) = ax + by + c$. Quem são $a$, $b$, e $c$?

e) Desenhe a curva de nível de $G(x,y) = G(p_0)$ \ColorRed{sobre o
  gráfico do item b}.


\newpage

% «miniteste-regras»  (to ".miniteste-regras")
% (c3m201pltanp 12 "miniteste-regras")
% (c3m201pltan     "miniteste-regras")
% (c2m201mt1p 7 "miniteste-regras")
% (c2m201mt1    "miniteste-regras")

{\bf Regras:}

As questões do mini-teste serão disponibilizadas às 14:00 da
sexta-feira 13/nov/2020 e você deverá entregar as respostas
\ColorRed{escritas à mão} até as 22:00 do sábado 14/nov/2020 na
plataforma Classroom. Se o Classroom der algum problema mande também
para este endereço de e-mail:

\ssk

\ColorRed{eduardoochs@gmail.com}

\ssk

Mini-testes entregues após este horário não serão considerados.

Durante as 24 horas do mini-teste o professor não responderá perguntas
sobre os assuntos do mini-teste mas você pode discutir com os seus
colegas --- inclusive no grupo da turma.

Este mini-teste vale 0.5 pontos extras na P1.

\newpage

{\bf Regras (cont.):}

\ssk

Os alunos que cumprirem uma série de condições (ainda não divulguei a
lista delas...) poderão compensar as suas questões erradas na P2
fazendo vídeos explicando passo a passo como resolvê-las na semana
seguinte à prova. \ColorRed{Uma das condições é ter feito todos os
  mini-testes, então não deixe de fazer e entregar este mini-teste!}





\end{document}

% «video»  (to ".video")
# (find-ssr-links "2020_vetor_gradiente" "grad")
# (find-ssr-links "2020_derivada_1" "deriv1")
# (find-es "pulseaudio" "pulseaudio-kill")

 (eepitch-shell)
 (eepitch-kill)
 (eepitch-shell)
cd /tmp/
cp -v  ~/LATEX/2020-1-C3-plano-tang.pdf /tmp/
xournalpp 2020-1-C3-plano-tang.pdf



%  __  __       _        
% |  \/  | __ _| | _____ 
% | |\/| |/ _` | |/ / _ \
% | |  | | (_| |   <  __/
% |_|  |_|\__,_|_|\_\___|
%                        
% <make>

 (eepitch-shell)
 (eepitch-kill)
 (eepitch-shell)
# (find-LATEXfile "2019planar-has-1.mk")
make -f 2019.mk STEM=2020-1-C3-plano-tang veryclean
make -f 2019.mk STEM=2020-1-C3-plano-tang pdf

% Local Variables:
% coding: utf-8-unix
% ee-tla: "c3m201pltan"
% End:
