% (find-LATEX "2020-1-C3-superficies-2.tex")
% (defun c () (interactive) (find-LATEXsh "lualatex -record 2020-1-C3-superficies-2.tex" :end))
% (defun D () (interactive) (find-pdf-page      "~/LATEX/2020-1-C3-superficies-2.pdf"))
% (defun d () (interactive) (find-pdftools-page "~/LATEX/2020-1-C3-superficies-2.pdf"))
% (defun e () (interactive) (find-LATEX "2020-1-C3-superficies-2.tex"))
% (defun u () (interactive) (find-latex-upload-links "2020-1-C3-superficies-2"))
% (defun v () (interactive) (find-2a '(e) '(d)) (g))
% (find-pdf-page   "~/LATEX/2020-1-C3-superficies-2.pdf")
% (find-sh0 "cp -v  ~/LATEX/2020-1-C3-superficies-2.pdf /tmp/")
% (find-sh0 "cp -v  ~/LATEX/2020-1-C3-superficies-2.pdf /tmp/pen/")
%   file:///home/edrx/LATEX/2020-1-C3-superficies-2.pdf
%               file:///tmp/2020-1-C3-superficies-2.pdf
%           file:///tmp/pen/2020-1-C3-superficies-2.pdf
% http://angg.twu.net/LATEX/2020-1-C3-superficies-2.pdf
% (find-LATEX "2019.mk")
% (find-C3-aula-links "2020-1-C3-superficies-2" "10" "sups2")


% «.defs»		(to "defs")
% «.title»		(to "title")
% «.exercicio-1»	(to "exercicio-1")
% «.exerc-1-conjuntos»	(to "exerc-1-conjuntos")
% «.outros-cortes»	(to "outros-cortes")

\documentclass[oneside,12pt]{article}
\usepackage[colorlinks,citecolor=DarkRed,urlcolor=DarkRed]{hyperref} % (find-es "tex" "hyperref")
\usepackage{amsmath}
\usepackage{amsfonts}
\usepackage{amssymb}
\usepackage{pict2e}
\usepackage[x11names,svgnames]{xcolor} % (find-es "tex" "xcolor")
%\usepackage{colorweb}                 % (find-es "tex" "colorweb")
%\usepackage{tikz}
%
% (find-dn6 "preamble6.lua" "preamble0")
%\usepackage{proof}   % For derivation trees ("%:" lines)
%\input diagxy        % For 2D diagrams ("%D" lines)
%\xyoption{curve}     % For the ".curve=" feature in 2D diagrams
%
\usepackage{edrx15}               % (find-LATEX "edrx15.sty")
\input edrxaccents.tex            % (find-LATEX "edrxaccents.tex")
\input edrxchars.tex              % (find-LATEX "edrxchars.tex")
\input edrxheadfoot.tex           % (find-LATEX "edrxheadfoot.tex")
\input edrxgac2.tex               % (find-LATEX "edrxgac2.tex")
%
%\usepackage[backend=biber,
%   style=alphabetic]{biblatex}            % (find-es "tex" "biber")
%\addbibresource{catsem-slides.bib}        % (find-LATEX "catsem-slides.bib")
%
% (find-es "tex" "geometry")
\usepackage[a6paper, landscape,
            top=1.5cm, bottom=.25cm, left=1cm, right=1cm, includefoot
           ]{geometry}
%
\begin{document}

\catcode`\^^J=10
\directlua{dofile "dednat6load.lua"}  % (find-LATEX "dednat6load.lua")

% %L dofile "edrxtikz.lua"  -- (find-LATEX "edrxtikz.lua")
% %L dofile "edrxpict.lua"  -- (find-LATEX "edrxpict.lua")
% \pu


% «defs»  (to ".defs")
% (find-LATEX "edrx15.sty" "colors-2019")
\long\def\ColorRed   #1{{\color{Red1}#1}}
\long\def\ColorViolet#1{{\color{MagentaVioletLight}#1}}
\long\def\ColorViolet#1{{\color{Violet!50!black}#1}}
\long\def\ColorGreen #1{{\color{SpringDarkHard}#1}}
\long\def\ColorGreen #1{{\color{SpringGreenDark}#1}}
\long\def\ColorGreen #1{{\color{SpringGreen4}#1}}
\long\def\ColorGray  #1{{\color{GrayLight}#1}}
\long\def\ColorGray  #1{{\color{black!30!white}#1}}
\long\def\ColorBrown #1{{\color{Brown}#1}}
\long\def\ColorBrown #1{{\color{brown}#1}}

\long\def\ColorShort #1{{\color{SpringGreen4}#1}}
\long\def\ColorLong  #1{{\color{Red1}#1}}

\def\frown{\ensuremath{{=}{(}}}
\def\True {\mathbf{V}}
\def\False{\mathbf{F}}

\def\drafturl{http://angg.twu.net/LATEX/2020-1-C2.pdf}
\def\drafturl{http://angg.twu.net/2020.1-C2.html}
\def\draftfooter{\tiny \href{\drafturl}{\jobname{}} \ColorBrown{\shorttoday{} \hours}}


%  _____ _ _   _                               
% |_   _(_) |_| | ___   _ __   __ _  __ _  ___ 
%   | | | | __| |/ _ \ | '_ \ / _` |/ _` |/ _ \
%   | | | | |_| |  __/ | |_) | (_| | (_| |  __/
%   |_| |_|\__|_|\___| | .__/ \__,_|\__, |\___|
%                      |_|          |___/      
%
% «title»  (to ".title")
% (c3m201sups2p 1 "title")
% (c3m201sups2    "title")

\thispagestyle{empty}

\begin{center}

\vspace*{1.2cm}

{\bf \Large Cálculo 3 - 2020.1}

\bsk

Aulas 11 e 12: Alguns truques para visualizar superfícies

\bsk

Eduardo Ochs - RCN/PURO/UFF

\url{http://angg.twu.net/2020.1-C3.html}

\end{center}

\newpage

A última aula terminou com um exercício bem importante -- o exercício
2 -- que acho que ninguém conseguiu fazer durante a aula, todo mundo
deixou pra depois... link:

% (c3m201sups1p 11 "exercicio-2")
% (c3m201sups1     "exercicio-2")

\ssk

\url{http://angg.twu.net/LATEX/2020-1-C3-superficies-1.pdf}

\msk

As expressões nos `$\ldots$' em
%
$$dz = \ldots dx + \ldots dy$$
%
vão poder ser interpretadas de vários jeitos -- por exemplo como
derivadas parciais (cap.5 do Bortolossi), como derivadas direcionais
(cap.8), ou como coeficientes do plano tangente (seção 7.2)... ou
seja, estamos estudando exemplos que vão nos preparar pra entender um
monte de conceitos importantes.

Vamos começar a aula de hoje vendo mais técnicas que vão nos ajudar a
visualizar superfícies e que também vão nos ajudar com esses conceitos
que vão vir depois.

\newpage

% «exercicio-1»  (to ".exercicio-1")
% (c3m201sups2p 3 "exercicio-1")
% (c3m201sups2    "exercicio-1")

{\bf Exercício 1}

\ssk

Sejam:

$ F(x,y) =
\begin{cases}
  \sqrt{5^2 - x^2 - y^2} & \text{quando $5^2 - x^2 - y^2≥0$}, \\
  0 & \text{quando $5^2 - x^2 - y^2<0$,} \\
 \end{cases}
$

e $(x_0,y_0)=(2,4)$.

\msk

a) Faça o diagrama de numerozinhos para esta função $F$, com $x$ e $y$
assumindo todos os valores inteiros de $-7$ até $7$. Use uma
calculadora pra aproximar o $z$ com precisão de dois dígitos -- por
exemplo, $F(1,0) = \sqrt{24} ≅ 4.90$.

\msk

{\sl Obs: quando você descobrir certas simetrias você vai ver que você
  vai precisar calcular no máximo 12 raízes quadradas.}

\newpage

% «exerc-1-conjuntos»  (to ".exerc-1-conjuntos")
% (c3m201sups2p 4 "exerc-1-conjuntos")
% (c3m201sups2    "exerc-1-conjuntos")

% «outros-cortes»  (to ".outros-cortes")
% (c3m192p 1 "outros-cortes")
% (c3m192    "outros-cortes")

{\bf Exercício 1 (continuação)}

\ssk

Sejam:
%
$$\begin{array}{ccl}
    S      &=& \setofxyzst{z = F(x,y)} \\
    A_3    &=& \setofxyzst{z = F(x,y), \; z=3} \\
    A_4    &=& \setofxyzst{z = F(x,y), \; z=4} \\
    A_5    &=& \setofxyzst{z = F(x,y), \; z=5} \\
    A_0    &=& \setofxyzst{z = F(x,y), \; z=0} \\
    A_{-1} &=& \setofxyzst{z = F(x,y), \; z=-1} \\
    B      &=& \setofxyzst{z = F(x,y), \; x=x_0} \\
    B'     &=& \setofst   {(y,z)∈\R^2} {z=F(x_0,y)} \\
    C      &=& \setofxyzst{z = F(x,y), \; y=y_0} \\
    C'     &=& \setofst   {(x,z)∈\R^2} {z=F(x,y_0)} \\
    D_3    &=& \setofxyst {F(x,y) = 3} \\
    D_4    &=& \setofxyst {F(x,y) = 4} \\
  \end{array}
$$


\newpage

{\bf Exercício 1 (continuação)}

\ssk

b) O que são os conjuntos $S, \ldots, D_4$ definidos na página
anterior? Para cada um deles visualize-o e depois tente desenhá-lo em
2D ou 3D de um jeito que ajudaria os seus colegas a entender que
conjunto é este. Inspire-se nos desenhos das páginas 83 a 98 do
capítulo 3 do Bortolossi. Complemente-os com explicações em português
quando você achar que os desenhos sozinhos não são claros o
suficiente.

\msk

c) Quais dos conjuntos $S, \ldots, D_4$ são funções de $\R$ em $\R$?

\msk

d) Quais dos conjuntos $S, \ldots, D_4$ são curvas de nível?

\msk

e) Qual é a derivada da função $z=F(x_0,y)$ em $y=y_0$?

\msk

f) Qual é a derivada da função $z=F(x,y_0)$ em $x=x_0$?





% (find-books "__analysis/__analysis.el" "bortolossi")
% (find-bortolossi3page (+ -78  81) "3.2. Funções de duas variáveis")
% (find-bortolossi3page (+ -78  83)   "Definição 3.2. Funções de duas variáveis")
% (find-bortolossi3page (+ -78  86)   "Vamos tentar outros cortes. (Figs: pp.90-95)")
% (find-bortolossi3page (+ -78  93)   "Exemplo 3.2. Sela de cavalo.")

% (find-bortolossi5page (+ -161 162) "5. Derivadas parciais")

% (find-bortolossi7page (+ -238 239) "7. Aproximação linear e a regra da cadeia")
% (find-bortolossi7page (+ -238 242) "7.2. Aproximação linear em Cálculo 2")
% (find-bortolossi7page (+ -238 243)    "plano tangente")
% (find-bortolossi7page (+ -238 252)    "Teorema 7.3")
% (find-bortolossi7page (+ -238 261)    "Teorema 7.6")

% (find-bortolossi8page (+ -290 291) "8. Derivadas direcionais e o vetor gradiente")



%\printbibliography



\end{document}

%  __  __       _        
% |  \/  | __ _| | _____ 
% | |\/| |/ _` | |/ / _ \
% | |  | | (_| |   <  __/
% |_|  |_|\__,_|_|\_\___|
%                        
% <make>

 (eepitch-shell)
 (eepitch-kill)
 (eepitch-shell)
# (find-LATEXfile "2019planar-has-1.mk")
make -f 2019.mk STEM=2020-1-C3-superficies-2 veryclean
make -f 2019.mk STEM=2020-1-C3-superficies-2 pdf

% Local Variables:
% coding: utf-8-unix
% ee-tla: "c3m201sups2"
% End:
