% (find-LATEX "2020-1-C2-escadas.tex")
% (defun c () (interactive) (find-LATEXsh "lualatex -record 2020-1-C2-escadas.tex" :end))
% (defun D () (interactive) (find-pdf-page      "~/LATEX/2020-1-C2-escadas.pdf"))
% (defun d () (interactive) (find-pdftools-page "~/LATEX/2020-1-C2-escadas.pdf"))
% (defun e () (interactive) (find-LATEX "2020-1-C2-escadas.tex"))
% (defun u () (interactive) (find-latex-upload-links "2020-1-C2-escadas"))
% (defun v () (interactive) (find-2a '(e) '(d)) (g))
% (find-pdf-page   "~/LATEX/2020-1-C2-escadas.pdf")
% (find-sh0 "cp -v  ~/LATEX/2020-1-C2-escadas.pdf /tmp/")
% (find-sh0 "cp -v  ~/LATEX/2020-1-C2-escadas.pdf /tmp/pen/")
%   file:///home/edrx/LATEX/2020-1-C2-escadas.pdf
%               file:///tmp/2020-1-C2-escadas.pdf
%           file:///tmp/pen/2020-1-C2-escadas.pdf
% http://angg.twu.net/LATEX/2020-1-C2-escadas.pdf
% (find-LATEX "2019.mk")

% «.defs»		(to "defs")
% «.title»		(to "title")
% «.funcoes-escada»	(to "funcoes-escada")
% «.exercicio-1»	(to "exercicio-1")
% «.spoiler»		(to "spoiler")

\documentclass[oneside,12pt]{article}
\usepackage[colorlinks,citecolor=DarkRed,urlcolor=DarkRed]{hyperref} % (find-es "tex" "hyperref")
\usepackage{amsmath}
\usepackage{amsfonts}
\usepackage{amssymb}
\usepackage{pict2e}
\usepackage[x11names,svgnames]{xcolor} % (find-es "tex" "xcolor")
\usepackage{colorweb}                 % (find-es "tex" "colorweb")
%\usepackage{tikz}
%
% (find-dn6 "preamble6.lua" "preamble0")
%\usepackage{proof}   % For derivation trees ("%:" lines)
%\input diagxy        % For 2D diagrams ("%D" lines)
%\xyoption{curve}     % For the ".curve=" feature in 2D diagrams
%
\usepackage{edrx15}               % (find-LATEX "edrx15.sty")
\input edrxaccents.tex            % (find-LATEX "edrxaccents.tex")
\input edrxchars.tex              % (find-LATEX "edrxchars.tex")
\input edrxheadfoot.tex           % (find-LATEX "edrxheadfoot.tex")
\input edrxgac2.tex               % (find-LATEX "edrxgac2.tex")
%
%\usepackage[backend=biber,
%   style=alphabetic]{biblatex}            % (find-es "tex" "biber")
%\addbibresource{catsem-slides.bib}        % (find-LATEX "catsem-slides.bib")
%
% (find-es "tex" "geometry")
\usepackage[a6paper, landscape,
            top=1.5cm, bottom=.25cm, left=1cm, right=1cm, includefoot
           ]{geometry}
%
\begin{document}

\catcode`\^^J=10
\directlua{dofile "dednat6load.lua"}  % (find-LATEX "dednat6load.lua")

%L dofile "edrxtikz.lua"  -- (find-LATEX "edrxtikz.lua")
%L dofile "edrxpict.lua"  -- (find-LATEX "edrxpict.lua")
\pu

%\printbibliography

% «defs»  (to ".defs")
% (find-LATEX "edrx15.sty" "colors-2019")
\long\def\ColorRed   #1{{\color{Red1}#1}}
\long\def\ColorViolet#1{{\color{MagentaVioletLight}#1}}
\long\def\ColorViolet#1{{\color{Violet!50!black}#1}}
\long\def\ColorGreen #1{{\color{SpringDarkHard}#1}}
\long\def\ColorGreen #1{{\color{SpringGreenDark}#1}}
\long\def\ColorGreen #1{{\color{SpringGreen4}#1}}
\long\def\ColorGray  #1{{\color{GrayLight}#1}}
\long\def\ColorGray  #1{{\color{black!30!white}#1}}
\long\def\ColorBrown #1{{\color{Brown}#1}}
\long\def\ColorBrown #1{{\color{brown}#1}}

\long\def\ColorShort #1{{\color{SpringGreen4}#1}}
\long\def\ColorLong  #1{{\color{Red1}#1}}

\def\frown{\ensuremath{{=}{(}}}
\def\True {\mathbf{V}}
\def\False{\mathbf{F}}

\def\drafturl{http://angg.twu.net/LATEX/2020-1-C2.pdf}
\def\drafturl{http://angg.twu.net/2020.1-C2.html}
\def\draftfooter{\tiny \href{\drafturl}{\jobname{}} \ColorBrown{\shorttoday{} \hours}}

% (find-angg ".emacs" "c2q192")


%  _____ _ _   _                               
% |_   _(_) |_| | ___   _ __   __ _  __ _  ___ 
%   | | | | __| |/ _ \ | '_ \ / _` |/ _` |/ _ \
%   | | | | |_| |  __/ | |_) | (_| | (_| |  __/
%   |_| |_|\__|_|\___| | .__/ \__,_|\__, |\___|
%                      |_|          |___/      
%
% «title»  (to ".title")
% (c2m201escadasp 1 "title")
% (c2m201escadas    "title")

\thispagestyle{empty}

\begin{center}

\vspace*{1.2cm}

{\bf \Large Cálculo 2 - 2020.1}

\bsk

Aulas 7 e 8: funções escada e como integrá-las

\bsk

Eduardo Ochs - RCN/PURO/UFF

\url{http://angg.twu.net/2020.1-C2.html}

\end{center}

\newpage

% «funcoes-escada»  (to ".funcoes-escada")
% (c2m201escadasp 2 "funcoes-escada")
% (c2m201escadas    "funcoes-escada")

\unitlength=10pt

Uma {\bf função escada} é uma cujo gráfico é composto por um número
finito de segmentos horizontais e um número finito -- talvez zero --
de pontos isolados. Por exemplo:
%
$$
 f(x) \;\; = \;\;
 \vcenter{\hbox{%
 \beginpicture(0,-2)(6,5)
   \pictgrid%
   \pictpiecewise{(0,3)--(1,3)o (1,4)c (1,2)o--(3,2)c (3,-1)o--(6,-1)}%
   \pictaxes%
 \end{picture}%
 }}
$$


No exercício 4 da aula passada -- links:

\ssk

\url{http://angg.twu.net/LATEX/2020-1-C2-def-integral.pdf}

\ssk

\noindent a função $g$ era uma função escada -- e você deve ter sacado
que $\lim_{k→∞} D_k = 0$, e portanto ela é integrável. Aliás, você
deve ter sacado que \ColorRed{qualquer} função escada é integrável...

\newpage

...e a integral de uma função escada $f(x)$ em qualquer intervalo,
calculada pelos limites da aula passada, dá exatamente a área sob a
curva dela naquele intervalo, que é fácil de calcular -- até no
olhômetro! -- somando as áreas dos seus retângulos. Por exemplo, para
a $f$ do slide anterior temos:
%
$$\begin{array}{rcl}
  \Intx{0}{3}{f(x)} &=& 3·(1-0) + 2·(3-1) \\
                    &=& 3 + 4 \\
                    &=& 7, \\
  \Intx{3}{6}{f(x)} &=& -1·(6-3) \\
                    &=& -3. \\
  \end{array}
$$

\newpage

Os teoremas que vão nos permitir calcular integrais bem rápido -- por
exemplo, quanto é $\Intx{0}{4}{4-(x-2)^2}$? Até agora ainda não
sabemos! -- são os dois ``Teoremas Fundamentais do Cálculo'', que a
gente chama de TFC1 e TFC2. O TFC1 diz, a grossíssimo modo, que ``a
derivada da integral de $f$ é a própria $f$'', mas os detalhes dele
são bem difíceis de entender, e o melhor modo de entendê-los é
começando com funções escada.

\newpage

% «exercicio-1»  (to ".exercicio-1")
% (c2m201escadasp 5 "exercicio-1")
% (c2m201escadas    "exercicio-1")

{\bf Exercício 1.}

Seja:
%
$$g(x) =
 \begin{cases}
  1 & \text{quando $x<2$}, \\
  0 & \text{quando $x=2$}, \\
  2 & \text{quando $2<x<4$}, \\
  0 & \text{quando $4≤x<6$}, \\
  -1 & \text{quando $6≤x≤8$}, \\
  1 & \text{quando $8<x$}, \\
 \end{cases}
$$

a) Faça o gráfico da função $g$.

b) Calcule $\Intx{1}{7}{g(x)}$.

c) Calcule $\Intx{1}{7.2}{g(x)}$.

\newpage

d) Seja $G(b) = \Intx{1}{b}{g(x)}$. Encontre uma fórmula para calcular
$G(b)$ que valha para todos os valores de $b$ no intervalo $(6,8)$.
Verifique se esta fórmula é compatível com os seus itens (b) e (c).

\msk

e) Calcule $\Intx{1}{3}{g(x)}$.

f) Calcule $\Intx{1}{3.2}{g(x)}$.

g) Encontre uma fórmula para calcular $G(b)$ que valha para todos os
valores de $b$ no intervalo $(2,4)$. Verifique se esta fórmula é
compatível com os seus itens (b) e (c).

\msk

h) Encontre uma fórmula para calcular $G(b)$ que valha para todos os
valores de $b$ no intervalo $(1,2)$.

i) Encontre uma fórmula para calcular $G(b)$ que valha para todos os
valores de $b$ no intervalo $(4,6)$.

j) Encontre uma fórmula para calcular $G(b)$ que valha para todos os
valores de $b$ no intervalo $(8,+∞)$.

\newpage

k) Junte os resultados que você obteve nos itens (d), (g), (h), (i),
(j), numa definição para $G$ por casos -- ou seja, complete:

$$G(x) =
 \begin{cases}
  \ldots & \text{quando $1<x<2$}, \\
  \ldots & \text{quando $2<x<4$}, \\
  \ldots & \text{quando $4<x<6$}, \\
  \ldots & \text{quando $6<x<8$}. \\
  \ldots & \text{quando $8<x<+∞$}. \\
 \end{cases}
$$

l) Faça um gráfico para esta função $G$. Obs: o domínio dela é

o conjunto $(1,2)∪(2,4)∪(4,6)∪(6,8)∪(8,+∞)$.

m) Os limites laterais da $G$ em $x→2$ são iguais? E em $x→4$?

E em $x→6$? E em $x→8$?

\newpage

n) Repare que o gráfico dessa sua $G$ é formado por {\sl segmentos

de retas}. Descubra, \ColorRed{olhando as inclinações dessas retas no
  gráfico,}

os valores de $G'(1.5)$, $G'(3)$, $G'(5)$, $G'(7)$, e compare-os com

$g(1.5)$, $g(3)$, $g(5)$, $g(7)$.


\newpage

% «spoiler»  (to ".spoiler")
% (c2m201escadasp 9 "spoiler")
% (c2m201escadas    "spoiler")

{\bf Spoiler:}

$$
 g(x) \;\; = \;\;
 \vcenter{\hbox{%
 \beginpicture(0,-2)(10,6)
   \pictgrid%
   \pictpiecewise{(0,1)--(2,1)o (2,0)c
                  (2,2)o--(4,2)o
                  (4,0)c--(6,0)o
                  (6,-1)c--(8,-1)c
                  (8,1)o--(10,1)}%
   \pictaxes%
 \end{picture}%
 }}
 %
 \qquad
 %
 G(x) \;\; = \;\;
 \vcenter{\hbox{%
 \beginpicture(0,-2)(10,6)
   \pictgrid%
   \pictpiecewise{(1,0)o--(2,1)o--(4,5)o--(6,5)o--(8,3)o--(10,5)}%
   \pictaxes%
 \end{picture}%
 }}
$$



\end{document}



% (c2m201defintp 12 "exercicio-4")
% (c2m201defint     "exercicio-4")




%  __  __       _        
% |  \/  | __ _| | _____ 
% | |\/| |/ _` | |/ / _ \
% | |  | | (_| |   <  __/
% |_|  |_|\__,_|_|\_\___|
%                        
% <make>

 (eepitch-shell)
 (eepitch-kill)
 (eepitch-shell)
# (find-LATEXfile "2019planar-has-1.mk")
make -f 2019.mk STEM=2020-1-C2-escadas veryclean
make -f 2019.mk STEM=2020-1-C2-escadas pdf

% Local Variables:
% coding: utf-8-unix
% ee-tla: "c2m201escadas"
% End:
