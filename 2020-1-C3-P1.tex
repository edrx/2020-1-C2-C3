% (find-LATEX "2020-1-C3-P1.tex")
% (defun c () (interactive) (find-LATEXsh "lualatex -record 2020-1-C3-P1.tex" :end))
% (defun D () (interactive) (find-pdf-page      "~/LATEX/2020-1-C3-P1.pdf"))
% (defun d () (interactive) (find-pdftools-page "~/LATEX/2020-1-C3-P1.pdf"))
% (defun e () (interactive) (find-LATEX "2020-1-C3-P1.tex"))
% (defun u () (interactive) (find-latex-upload-links "2020-1-C3-P1"))
% (defun v () (interactive) (find-2a '(e) '(d)) (g))
% (find-pdf-page   "~/LATEX/2020-1-C3-P1.pdf")
% (find-sh0 "cp -v  ~/LATEX/2020-1-C3-P1.pdf /tmp/")
% (find-sh0 "cp -v  ~/LATEX/2020-1-C3-P1.pdf /tmp/pen/")
%   file:///home/edrx/LATEX/2020-1-C3-P1.pdf
%               file:///tmp/2020-1-C3-P1.pdf
%           file:///tmp/pen/2020-1-C3-P1.pdf
% http://angg.twu.net/LATEX/2020-1-C3-P1.pdf
% (find-LATEX "2019.mk")

% «.regras»	(to "regras")
% «.questao-1»	(to "questao-1")
% «.questao-2»	(to "questao-2")

\documentclass[oneside,12pt]{article}
\usepackage[colorlinks,citecolor=DarkRed,urlcolor=DarkRed]{hyperref} % (find-es "tex" "hyperref")
\usepackage{amsmath}
\usepackage{amsfonts}
\usepackage{amssymb}
\usepackage{pict2e}
\usepackage[x11names,svgnames]{xcolor} % (find-es "tex" "xcolor")
%\usepackage{colorweb}                 % (find-es "tex" "colorweb")
%\usepackage{tikz}
%
% (find-dn6 "preamble6.lua" "preamble0")
%\usepackage{proof}   % For derivation trees ("%:" lines)
%\input diagxy        % For 2D diagrams ("%D" lines)
%\xyoption{curve}     % For the ".curve=" feature in 2D diagrams
%
\usepackage{edrx15}               % (find-LATEX "edrx15.sty")
\input edrxaccents.tex            % (find-LATEX "edrxaccents.tex")
\input edrxchars.tex              % (find-LATEX "edrxchars.tex")
\input edrxheadfoot.tex           % (find-LATEX "edrxheadfoot.tex")
\input edrxgac2.tex               % (find-LATEX "edrxgac2.tex")
%
%\usepackage[backend=biber,
%   style=alphabetic]{biblatex}            % (find-es "tex" "biber")
%\addbibresource{catsem-slides.bib}        % (find-LATEX "catsem-slides.bib")
%
% (find-es "tex" "geometry")
\usepackage[a6paper, landscape,
            top=1.5cm, bottom=.25cm, left=1cm, right=1cm, includefoot
           ]{geometry}
%
\begin{document}

\catcode`\^^J=10
\directlua{dofile "dednat6load.lua"}  % (find-LATEX "dednat6load.lua")

% %L dofile "edrxtikz.lua"  -- (find-LATEX "edrxtikz.lua")
% %L dofile "edrxpict.lua"  -- (find-LATEX "edrxpict.lua")
% \pu

% «defs»  (to ".defs")
% (find-LATEX "edrx15.sty" "colors-2019")
\long\def\ColorRed   #1{{\color{Red1}#1}}
\long\def\ColorViolet#1{{\color{MagentaVioletLight}#1}}
\long\def\ColorViolet#1{{\color{Violet!50!black}#1}}
\long\def\ColorGreen #1{{\color{SpringDarkHard}#1}}
\long\def\ColorGreen #1{{\color{SpringGreenDark}#1}}
\long\def\ColorGreen #1{{\color{SpringGreen4}#1}}
\long\def\ColorGray  #1{{\color{GrayLight}#1}}
\long\def\ColorGray  #1{{\color{black!30!white}#1}}
\long\def\ColorBrown #1{{\color{Brown}#1}}
\long\def\ColorBrown #1{{\color{brown}#1}}

\long\def\ColorShort #1{{\color{SpringGreen4}#1}}
\long\def\ColorLong  #1{{\color{Red1}#1}}

\def\frown{\ensuremath{{=}{(}}}
\def\True {\mathbf{V}}
\def\False{\mathbf{F}}

\def\drafturl{http://angg.twu.net/LATEX/2020-1-C2.pdf}
\def\drafturl{http://angg.twu.net/2020.1-C2.html}
\def\draftfooter{\tiny \href{\drafturl}{\jobname{}} \ColorBrown{\shorttoday{} \hours}}


%  _____ _ _   _                               
% |_   _(_) |_| | ___   _ __   __ _  __ _  ___ 
%   | | | | __| |/ _ \ | '_ \ / _` |/ _` |/ _ \
%   | | | | |_| |  __/ | |_) | (_| | (_| |  __/
%   |_| |_|\__|_|\___| | .__/ \__,_|\__, |\___|
%                      |_|          |___/      
%
% «title»  (to ".title")
% (c3m201p1p 1 "title")
% (c3m201p1a   "title")

\thispagestyle{empty}

\begin{center}

\vspace*{1.2cm}

{\bf \Large Cálculo 3 - 2020.1}

\bsk

P1 (Primeira prova)

\bsk

Eduardo Ochs - RCN/PURO/UFF

\url{http://angg.twu.net/2020.1-C3.html}

\end{center}

\newpage

% «regras»  (to ".regras")
% (c3m201p1p 2 "regras")
% (c3m201p1    "regras")

% (c2m201p1p 2 "regras")
% (c2m201p1    "regras")

{\bf Regras para a P1:}

\ssk

As questões da P1 serão disponibilizadas às 16:45 da
sexta-feira 27/nov/2020 e você deverá entregar as respostas
\ColorRed{escritas à mão} até as 16:45 do sábado 28/nov/2020 na
plataforma Classroom. Se o Classroom der algum problema mande também
para este endereço de e-mail:


\ssk

\ColorRed{eduardoochs@gmail.com}

\ssk

Provas entregues após este horário não serão considerados.

Durante as 24 horas do mini-teste o professor não responderá perguntas
sobre os assuntos do mini-teste, mas você pode discutir com os seus
colegas... \ColorRed{só que as respostas devem ser individuais}.


\newpage

% «questao-1»  (to ".questao-1")
% (c3m201p1p 3 "questao-1")
% (c3m201p1    "questao-1")
% (find-es "ipython" "2020.1-C3-P1" "Questao 1")
% (c3m201dp1p 2 "dicas")
% (c3m201dp1    "dicas")


{\bf Questão 1}

(Total: 2.0 pts)

(Baseada no exercício 2 da aula 13)

\ssk

Sejam $F(x,y) = x·y$ e $(x_0,y_0) = (4,2)$.

a) Calcule $F_x$ e $F_y$ no ponto $(x_0,y_0)$.

b) calcule a aproximação para $F_x(x_0,y_0)$ usando $Δx=0.1$.

c) calcule a aproximação para $F_x(x_0,y_0)$ usando $Δx=-0.1$.

d) calcule a aproximação para $F_y(x_0,y_0)$ usando $Δy=0.1$.

e) calcule a aproximação para $F_y(x_0,y_0)$ usando $Δy=-0.1$.

\msk

Cada item vale 0.4 pts.


% (c3m201dp1p 3 "aula-13")
% (c3m201dp1    "aula-13")
% (c3m201derpsp 4 "exercicio-2")
% (c3m201derps    "exercicio-2")


\newpage


% «questao-2»  (to ".questao-2")
% (c3m201p1p 4 "questao-2")
% (c3m201p1    "questao-2")
% (find-es "ipython" "2020.1-C3-P1" "Questao 2")
% (c3m201dp1p 2 "aulas-7-e-8")
% (c3m201dp1    "aulas-7-e-8")
% (c3m201taylor3p 5 "exercicio-1")
% (c3m201taylor3    "exercicio-1")
% (c3m201taylor3p 9 "exercicio-4")
% (c3m201taylor3    "exercicio-4")


{\bf Questão 2}

(Baseada no material das aulas 7 e 8)

(Valor total da questão: 8.0 pts.)

(Total nesta página: 3.0 pts.)

\ssk

Sejam $G(x,y) = x^2 + 4y^2$. $H(x,y) = \sqrt{x^2 + 4y^2}$.

a) (1.0 pts) Desenhe pelo menos quatro curvas de nível

de $z=G(x,y)$.

b) (1.0 pts) Desenhe pelo menos quatro curvas de nível

de $z=H(x,y)$.

c) (0.1 pts) Calcule $∇G$.

d) (0.2 pts) Calcule $∇G(3,1)$.

e) (0.3 pts) Calcule $∇H$.

f) (0.4 pts) Calcule $∇H(3,1)$.

\msk


\newpage

{\bf Questão 2 (continuação)}

(Total nesta página: 2.5 pts.)

\ssk

g) (1.5 pts) Digamos que $z=H(x,y)$. Faça as contas com diferenciais e encontre
as expressões que só dependem de $x$ e $y$ --- não de $dx$, $dy$,
$\frac{dx}{dy}$, etc --- que você pode pôr nas lacunas da igualdade
abaixo para torná-la verdadeira:
%
$$dz = ▁▁▁dx + ▁▁▁dy$$

h) (1.0 pts) O que acontece na (g) quando $x=3$ e $y=1$? O resultado
que você obteve na (g) no ponto $(3,1)$ é compatível com o resultado
que você obteve na (f)? Explique.


\newpage

{\bf Questão 2 (continuação)}

(Total nesta página: 2.5 pts.)

\ssk

i) (1.5 pts) Digamos que $z=H(x,y)$, $y=f(x)$, e que esta $f$
``percorre uma curva de nível da $H$'' --- ou seja, $\frac{dz}{dx}=0$.
Encontre uma expressão que só depende de $x$ e $y$, isto é, não de
$dx$, $dy$, $\frac{dx}{dy}$, $z$, etc, que você pode pôr na lacuna da
igualdade abaixo para torná-la verdadeira:
%
$$\frac{dy}{dx} = ▁▁▁ .$$

j) (1.0 pts) O que acontece na (i) quando $x=3$ e $y=1$? O resultado
que você obteve na (i) no ponto $(3,1)$ é compatível com o resultado
que você obteve na (f)? Explique.







%\printbibliography

\end{document}

%  __  __       _        
% |  \/  | __ _| | _____ 
% | |\/| |/ _` | |/ / _ \
% | |  | | (_| |   <  __/
% |_|  |_|\__,_|_|\_\___|
%                        
% <make>

 (eepitch-shell)
 (eepitch-kill)
 (eepitch-shell)
# (find-LATEXfile "2019planar-has-1.mk")
make -f 2019.mk STEM=2020-1-C3-P1 veryclean
make -f 2019.mk STEM=2020-1-C3-P1 pdf

% Local Variables:
% coding: utf-8-unix
% ee-tla: "c3m201p1"
% End:
